@c -----------------------------------------------------------------------------
@node A User Manual For GRX2, , Top,Top
@unnumbered GRX2 User's Manual

@menu
* Credits::
* Hello world::
* Data types and function declarations::
* Setting the graphics driver::
* Setting video modes::
* Graphics contexts::
* Context use::
* Color management::
* Portable use of a few colors::
* Graphics primitives::
* Non-clipping graphics primitives::
* Customized line drawing::
* Pattern filled graphics primitives::
* Patterned line drawing::
* Image manipulation::
* Text drawing::
* Drawing in user coordinates::
* Graphics cursors::
* Keyboard input::
* Mouse event handling::
* Writing/reading PNM graphics files::
* BGI interface::
* Test examples::
* Includes::
@end menu


@node Credits, Hello world, , A User Manual For GRX2
@unnumberedsec Credits

 
@example
                       GRX 2.3.4 User's Manual

             A graphics library for DOS, Linux and X11

@end example

Based on the original doc written by: Csaba Biegl on August 10, 1992 
Updated by: Mariano Alvarez Fernandez on August 17, 2000 

2.3.4 update: January 28, 2001.
______________________________________________________________________________

@example
                             Abstract

@end example

LIBGRX is a graphics library for DOS (DJGPP v2 and Turbo C/Borland C++ 
versions), Linux and X11. On DOS it supports VGA (32768, 256 or 16 colors), EGA 
(16 colors), and VESA compliant cards (up to 16M colors). On Linux console it 
uses svgalib. On X11 it must work on any X11R5 (or later). 


@c -----------------------------------------------------------------------------
@node Hello world, Data types and function declarations, Credits, A User Manual For GRX2
@unnumberedsec Hello world
The next program draws a double frame around the screen and writes "Hello, GRX 
world" centered. Then it waits after a key is pressed. 

@example
#include <string.h>
#include <grx20.h>
#include <grxkeys.h>

int main()
@{
  char *message = "Hello, GRX world";
  int x, y;
  GrTextOption grt;

  GrSetMode( GR_default_graphics );

  grt.txo_font = &GrDefaultFont;
  grt.txo_fgcolor.v = GrWhite();
  grt.txo_bgcolor.v = GrBlack();
  grt.txo_direct = GR_TEXT_RIGHT;
  grt.txo_xalign = GR_ALIGN_CENTER;
  grt.txo_yalign = GR_ALIGN_CENTER;
  grt.txo_chrtype = GR_BYTE_TEXT;

  GrBox( 0,0,GrMaxX(),GrMaxY(),GrWhite() );
  GrBox( 4,4,GrMaxX()-4,GrMaxY()-4,GrWhite() );

  x = GrMaxX()/2;
  y = GrMaxY()/2;
  GrDrawString( message,strlen( message ),x,y,&grt );

  GrKeyRead();

  return 0;
@}

@end example

@c -----------------------------------------------------------------------------
@node Data types and function declarations, Setting the graphics driver, Hello world, A User Manual For GRX2
@unnumberedsec Data types and function declarations
All public data structures and graphics primitives meant for usage by the 
application program are declared/prototyped in the header files (in the 
'include' sub-directory):

@example
   * grdriver.h   graphics driver format specifications 
   * grfontdv.h   format of a font when loaded into memory 
   * grx20.h      drawing-related structures and functions 
   * grxkeys.h    platform independent key definitions

User programs normally only include @pxref{grx20.h} and @pxref{grxkeys.h}
@end example

@c -----------------------------------------------------------------------------
@node Setting the graphics driver, Setting video modes, Data types and function declarations, A User Manual For GRX2
@unnumberedsec Setting the graphics driver
The graphics driver is normally set by the final user by the environment 
variable GRX20DRV, but a program can set it using: 

@example
int GrSetDriver(char *drvspec);
@end example

The drvspec string has the same format as the environment variable: 

@example
<driver> gw <width> gh <height> nc <colors>
@end example

Available drivers are for: 

@example
* DOS => herc, stdvga, stdega, et4000, cl5426, mach64, ati28800, s3, VESA, memory 
* Linux => svgalib, memory 
* X11 => xwin, memory 
@end example

The optionals gw, gh and nc parameters set the desired default graphics mode. 
Normal values for 'nc' are 2, 16, 256, 64K and 16M. The current driver name can 
be obtained from: 

@example
GrCurrentVideoDriver()->name
@end example

@c -----------------------------------------------------------------------------
@node Setting video modes, Graphics contexts, Setting the graphics driver, A User Manual For GRX2
@unnumberedsec Setting video modes

Before a program can do any graphics drawing it has to configure the graphics 
driver for the desired graphics mode. It is done with the GrSetMode function as 
follows: 

@example
int GrSetMode(int which,...); 
@end example

On succes it returns non-zero (TRUE). The which parameter can be one of the
following constants, declared in grx20.h:

@example
typedef enum _GR_graphicsModes @{
  GR_80_25_text,
  GR_default_text,
  GR_width_height_text,
  GR_biggest_text,
  GR_320_200_graphics,
  GR_default_graphics,
  GR_width_height_graphics,
  GR_biggest_noninterlaced_graphics,
  GR_biggest_graphics,
  GR_width_height_color_graphics,
  GR_width_height_color_text,
  GR_custom_graphics,
  GR_width_height_bpp_graphics,
  GR_width_height_bpp_text,
  GR_custom_bpp_graphics,
  GR_NC_80_25_text,
  GR_NC_default_text,
  GR_NC_width_height_text,
  GR_NC_biggest_text,
  GR_NC_320_200_graphics,
  GR_NC_default_graphics,
  GR_NC_width_height_graphics,
  GR_NC_biggest_noninterlaced_graphics,
  GR_NC_biggest_graphics,
  GR_NC_width_height_color_graphics,
  GR_NC_width_height_color_text,
  GR_NC_custom_graphics,
  GR_NC_width_height_bpp_graphics,
  GR_NC_width_height_bpp_text,
  GR_NC_custom_bpp_graphics,
@} GrGraphicsMode;
@end example

The GR_width_height_text and GR_width_height_graphics modes require the two 
size arguments: int width and int height. 

The GR_width_height_color_graphics and GR_width_height_color_text modes 
require three arguments: int width, int height and GrColor colors. 

The GR_width_height_bpp_graphics and GR_width_height_bpp_text modes require 
three arguments: int width, int height and int bpp (bits per plane instead 
number of colors). 

The GR_custom_graphics and GR_custom_bpp_graphics modes require five 
arguments: int width, int height, GrColor colors or int bpp, int vx and int vy. 
Using this modes you can set a virtual screen of vx by vy size. 

A call with any other mode does not require any arguments. 

The GR_NC_... modes are equivalent to the GR_.. ones, but they don't clear the 
video memory. 

Graphics drivers can provide info of the supported graphics modes, use the 
next code skeleton to colect the data: 

@example
@{
  GrFrameMode fm;
  const GrVideoMode *mp;
  for(fm =GR_firstGraphicsFrameMode; fm <= GR_lastGraphicsFrameMode; fm++) @{
    mp = GrFirstVideoMode(fm);
    while( mp != NULL ) @{
      ..
      .. use the mp info
      ..
      mp = GrNextVideoMode(mp))
    @}
  @}
@}
@end example

Don't worry if you don't understand it, normal user programs don't need to 
know about FrameModes. The GrVideoMode structure has the following fields: 

@example
typedef struct _GR_videoMode GrVideoMode;

struct _GR_videoMode @{
  char    present;                    /* is it really available? */
  char    bpp;                        /* log2 of # of colors */
  short   width,height;               /* video mode geometry */
  short   mode;                       /* BIOS mode number (if any) */
  int     lineoffset;                 /* scan line length */
  int     privdata;                   /* driver can use it for anything */
  struct _GR_videoModeExt *extinfo;   /* extra info (maybe shared) */
@};
@end example

The width, height and bpp members are the useful information if you are 
interested in set modes other than the GR_default_graphics.

A user-defined function can be invoked every time the video mode is changed 
(i.e. GrSetMode is called). This function should not take any parameters and 
don't return any value. It can be installed (for all subsequent GrSetMode calls) 
with the: 

@example
void GrSetModeHook(void (*hookfunc)(void));
@end example

function. The current graphics mode (one of the valid mode argument values for 
GrSetMode) can be obtained with the: 

@example
GrGraphicsMode GrCurrentMode(void);
@end example

function, while the type of the installed graphics adapter can be determined 
with the: 

@example
GrVideoAdapter GrAdapterType(void);
@end example

function. GrAdapterType returns the type of the adapter as one of the following 
symbolic constants (defined in grx20.h): 

@example
typedef enum _GR_videoAdapters @{
  GR_UNKNOWN = (-1),     /* not known (before driver set) */
  GR_VGA,                /* VGA adapter */
  GR_EGA,                /* EGA adapter */
  GR_HERC,               /* Hercules mono adapter */
  GR_8514A,              /* 8514A or compatible */
  GR_S3,                 /* S3 graphics accelerator */
  GR_XWIN,               /* X11 driver */
  GR_MEM                 /* memory only driver */
@} GrVideoAdapter;
@end example

Note that the VESA driver return GR_VGA here. 

See the @pxref{modetest.c} example.

@c -----------------------------------------------------------------------------
@node Graphics contexts, Context use, Setting video modes, A User Manual For GRX2
@unnumberedsec Graphics contexts


The library supports a set of drawing regions called contexts (the GrContext 
structure). These can be in video memory or in system memory. Contexts in system 
memory always have the same memory organization as the video memory. When 
GrSetMode is called, a default context is created which maps to the whole 
graphics screen. Contexts are described by the GrContext data structure: 

@example
typedef struct _GR_context GrContext;

struct _GR_context @{
  struct _GR_frame    gc_frame;       /* frame buffer info */
  struct _GR_context *gc_root;        /* context which owns frame */
  int    gc_xmax;                     /* max X coord (width  - 1) */
  int    gc_ymax;                     /* max Y coord (height - 1) */
  int    gc_xoffset;                  /* X offset from root's base */
  int    gc_yoffset;                  /* Y offset from root's base */
  int    gc_xcliplo;                  /* low X clipping limit */
  int    gc_ycliplo;                  /* low Y clipping limit */
  int    gc_xcliphi;                  /* high X clipping limit */
  int    gc_ycliphi;                  /* high Y clipping limit */
  int    gc_usrxbase;                 /* user window min X coordinate */
  int    gc_usrybase;                 /* user window min Y coordinate */
  int    gc_usrwidth;                 /* user window width  */
  int    gc_usrheight;                /* user window height */
# define gc_baseaddr                  gc_frame.gf_baseaddr
# define gc_selector                  gc_frame.gf_selector
# define gc_onscreen                  gc_frame.gf_onscreen
# define gc_memflags                  gc_frame.gf_memflags
# define gc_lineoffset                gc_frame.gf_lineoffset
# define gc_driver                    gc_frame.gf_driver
@};
@end example

The following four functions return information about the layout of and memory 
occupied by a graphics context of size width by height in the current graphics 
mode (as set up by GrSetMode): 

@example
int GrLineOffset(int width); 
int GrNumPlanes(void); 
long GrPlaneSize(int w,int h); 
long GrContextSize(int w,int h); 
@end example

GrLineOffset always returns the offset between successive pixel rows of the 
context in bytes. GrNumPlanes returns the number of bitmap planes in the current 
graphics mode. GrContextSize calculates the total amount of memory needed by a 
context, while GrPlaneSize calculates the size of a bitplane in the context. The 
function: 

@example
GrContext *GrCreateContext(int w,int h,char far *memory[4],GrContext *where);
@end example

can be used to create a new context in system memory. The NULL pointer is also 
accepted as the value of the memory and where arguments, in this case the 
library allocates the necessary amount of memory internally. It is a general 
convention in the library that functions returning pointers to any LIBGRX 
specific data structure have a last argument (most of the time named where in 
the prototypes) which can be used to pass the address of the data structure 
which should be filled with the result. If this where pointer has the value of 
NULL, then the library allocates space for the data structure internally. 

The memory argument is really a 4 pointer array, each pointer must point to 
space to handle GrPlaneSize(w,h) bytes, really only GrNumPlanes() pointers must 
be malloced, the rest can be NULL. Nevertheless the normal use (see below) is

@example
gc = GrCreateContext(w,h,NULL,NULL);
@end example

so yo don't need to care about. 

The function: 
@example

GrContext *GrCreateSubContext(int x1,int y1,int x2,int y2,
                              const GrContext *parent,GrContext *where);
@end example

creates a new sub-context which maps to a part of an existing context. The 
coordinate arguments (x1 through y2) are interpreted relative to the parent 
context's limits. Pixel addressing is zero-based even in sub-contexts, i.e. the 
address of the top left pixel is (0,0) even in a sub-context which has been 
mapped onto the interior of its parent context. 

Sub-contexts can be resized, but not their parents (i.e. anything returned by 
GrCreateContext or set up by GrSetMode cannot be resized -- because this could 
lead to irrecoverable "loss" of drawing memory. The following function can be 
used for this purpose: 

@example
void GrResizeSubContext(GrContext *context,int x1,int y1,int x2,int y2); 
@end example

The current context structure is stored in a static location in the library. 
(For efficiency reasons -- it is used quite frequently, and this way no pointer 
dereferencing is necessary.) The context stores all relevant information about 
the video organization, coordinate limits, etc... The current context can be set 
with the: 

@example
void GrSetContext(const GrContext *context);
@end example

function. This function will reset the current context to the full graphics 
screen if it is passed the NULL pointer as argument. The value of the current 
context can be saved into a GrContext structure pointed to by where using: 

@example
GrContext *GrSaveContext(GrContext *where); 
@end example

(Again, if where is NULL, the library allocates the space.) The next two 
functions: 

@example
const GrContext *GrCurrentContext(void);
const GrContext *GrScreenContext(void);
@end example

return the current context and the screen context respectively. Contexts can be 
destroyed with: 

@example
void GrDestroyContext(GrContext *context);
@end example

This function will free the memory occupied by the context only if it was 
allocated originally by the library. The next three functions set up and query 
the clipping limits associated with the current context: 

@example
void GrSetClipBox(int x1,int y1,int x2,int y2); 
void GrGetClipBox(int *x1p,int *y1p,int *x2p,int *y2p); 
void GrResetClipBox(void);
@end example

GrResetClipBox sets the clipping limits to the limits of context. These are 
the limits set up initially when a context is created. There are three similar 
functions to sets/gets the clipping limits of any context: 

@example
void  GrSetClipBoxC(GrContext *c,int x1,int y1,int x2,int y2);
void  GrGetClipBoxC(const GrContext *c,int *x1p,int *y1p,int *x2p,int *y2p);
void  GrResetClipBoxC(GrContext *c);
@end example

The limits of the current context can be obtained using the following 
functions: 

@example
int GrMaxX(void); 
int GrMaxY(void); 
int GrSizeX(void); 
int GrSizeY(void);
@end example

The Max functions return the biggest valid coordinate, while the Size 
functions return a value one higher. The limits of the graphics screen 
(regardless of the current context) can be obtained with: 

@example
int GrScreenX(void); 
int GrScreenY(void); 
@end example

If you had set a virtual screen (using a custom graphics mode), the limits of 
the virtual screen can be fetched with: 

@example
int GrVirtualX(void);
int GrVirtualY(void);
@end example

The routine: 

@example
int GrScreenIsVirtual(void);
@end example

returns non zero if a virtual screen is set. The rectangle showed in the real 
screen can be set with: 

@example
int GrSetViewport(int xpos,int ypos);
@end example

and the current viewport position can be obtained by: 

@example
int GrViewportX(void);
int GrViewportY(void);
@end example

@c -----------------------------------------------------------------------------
@node Context use, Color management, Graphics contexts, A User Manual For GRX2
@unnumberedsec Context use


Here is a example of normal context use:  

@example
GrContext *grc;

if( (grc = GrCreateContext( w,h,NULL,NULL )) == NULL )@{
  ...process the error
  @}
else @{
  GrSetContext( grc );
  ...do some drawing
  ...and probably bitblt to the screen context
  GrSetContext( NULL ); /* the screen context! */
  GrDestroyContext( grc );
  @}

@end example

But if you have a GrContext variable (not a pointer) you want to use (probably 
because is static to some routines) you can do: 

@example
static GrContext grc; /* not a pointer!! */

if( GrCreateContext( w,h,NULL,&grc )) == NULL ) @{
  ...process the error
 @}
else @{
  GrSetContext( &grc );
  ...do some drawing
  ...and probably bitblt to the screen context
  GrSetContext( NULL ); /* the screen context! */
  GrDestroyContext( &grc );
  @}

@end example

Note that GrDestoryContext knows if grc was automatically malloced or not!! 

Only if you don't want GrCreateContext use malloc at all, you must allocate 
the memory buffers and pass it to GrCreateContext. 

Using GrCreateSubContext is the same, except it doesn't need the buffer, 
because it uses the parent buffer. 

See the @pxref{winclip.c} and @pxref{wintest.c} examples.

@c -----------------------------------------------------------------------------
@node Color management, Portable use of a few colors, Context use, A User Manual For GRX2
@unnumberedsec Color management

GRX defines the type GrColor for color variables. GrColor it's a 32 bits 
integer. The 8 left bits are reserved for the write mode (see below). The 24 
bits right are the color value. 

The library supports two models for color management. In the 'indirect' (or 
color table) model, color values are indices to a color table. The color table 
slots will be allocated with the highest resolution supported by the hardware 
(EGA: 2 bits, VGA: 6 bits) with respect to the component color intensities. In 
the 'direct' (or RGB) model, color values map directly into component color 
intensities with non-overlapping bitfields of the color index representing the 
component colors. 

Color table model is supported until 256 color modes. The RGB model is 
supported in 256 color and up color modes. 

In RGB model the color index map to component color intensities depend on the 
video mode set, so it can't be assumed the component color bitfields (but if you 
are curious check the GrColorInfo global structure in grx20.h). 

After the first GrSetMode call two colors are always defined: black and white. 
The color values of these two colors are returned by the functions: 

@example
GrColor GrBlack(void); 
GrColor GrWhite(void);
@end example

GrBlack() is guaranteed to be 0. 

The library supports five write modes (a write mode descibes the operation 
between the actual bit color and the one to be set): write, XOR, logical OR, 
logical AND and IMAGE. These can be selected with OR-ing the color value with 
one of the following constants declared in grx20.h : 

@example
#define GrWRITE       0UL            /* write color */
#define GrXOR         0x01000000UL   /* to "XOR" any color to the screen */
#define GrOR          0x02000000UL   /* to "OR" to the screen */
#define GrAND         0x03000000UL   /* to "AND" to the screen */
#define GrIMAGE       0x04000000UL   /* blit: write, except given color */
@end example

The GrIMAGE write mode only works with the bitblt function. 
By convention, the no-op color is obtained by combining color value 0 (black) 
with the XOR operation. This no-op color has been defined in grx20.h as: 

@example
#define GrNOCOLOR     (GrXOR | 0)    /* GrNOCOLOR is used for "no" color */
@end example

The write mode part and the color value part of a GrColor variable can be 
obtained OR-ing it with one of the following constants declared in grx20.h: 

@example
#define GrCVALUEMASK  0x00ffffffUL   /* color value mask */
#define GrCMODEMASK   0xff000000UL   /* color operation mask */
@end example

The number of colors in the current graphics mode is returned by the: 

@example
GrColor GrNumColors(void);
@end example

function, while the number of unused, available color can be obtained by 
calling: 

@example
GrColor GrNumFreeColors(void); 
@end example

Colors can be allocated with the: 

@example
GrColor GrAllocColor(int r,int g,int b);
@end example

function (component intensities can range from 0 to 255), or with the: 

@example
GrColor GrAllocCell(void);
@end example

function. In the second case the component intensities of the returned color can 
be set with: 

@example
void GrSetColor(GrColor color,int r,int g,int b); 
@end example

In the color table model both Alloc functions return GrNOCOLOR if there are no 
more free colors available. In the RGB model GrNumFreeColors returns 0 and 
GrAllocCell always returns GrNOCOLOR, as colors returned by GrAllocCell are 
meant to be changed -- what is not supposed to be done in RGB mode. Also note 
that GrAllocColor operates much more efficiently in RGB mode, and that it never 
returns GrNOCOLOR in this case. 

Color table entries can be freed (when not in RGB mode) by calling: 

@example
void GrFreeColor(GrColor color);
@end example

The component intensities of any color can be queried using the function: 

@example
void GrQueryColor(GrColor c,int *r,int *g,int *b); 
@end example

Initially the color system is in color table (indirect) model if there are 256 
or less colors. 256 color modes can be put into the RGB model by calling: 

@example
void GrSetRGBcolorMode(void);
@end example

The color system can be reset (i.e. put back into color table model if 
possible, all colors freed except for black and white) by calling: 

@example
void GrResetColors(void);
@end example

The function: 

@example
void GrRefreshColors(void);
@end example

reloads the currently allocated color values into the video hardware. This 
function is not needed in typical applications, unless the display adapter is 
programmed directly by the application. 

This functions: 

@example
GrColor GrAllocColorID(int r,int g,int b);
void GrQueryColorID(GrColor c,int *r,int *g,int *b);
@end example

are inlined versions (except if you compile GRX with GRX_SKIP_INLINES defined) 
to be used in the RGB model (in the color table model they call the normal 
routines). 

See the @pxref{rgbtest.c} and @pxref{colorops.c} examples.


@c -----------------------------------------------------------------------------
@node Portable use of a few colors, Graphics primitives, Color management, A User Manual For GRX2
@unnumberedsec Portable use of a few colors

People that only want to use a few colors find the GRX color handling a bit 
confusing, but it gives the power to manage a lot of color deeps and two color 
models. Here are some guidelines to easily use the famous 16 ega colors in GRX 
programs. We need this GRX function: 

@example
GrColor *GrAllocEgaColors(void);
@end example

it returns a 16 GrColor array with the 16 ega colors alloced (really it's a 
trivial function, read the source src/setup/colorega.c). We can use a 
construction like that: 

First, in your C code make a global pointer, and init it after set the 
graphics mode: 

@example
GrColor *egacolors;
....
int your_setup_function( ... )
@{
  ...
  GrSetMode( ... )
  ...
  egacolors = GrAllocEgaColors();
  ...
@}

@end example

Next, add this to your main include file: 

@example
extern GrColor *egacolors;
#define BLACK        egacolors[0]
#define BLUE         egacolors[1]
#define GREEN        egacolors[2]
#define CYAN         egacolors[3]
#define RED          egacolors[4]
#define MAGENTA      egacolors[5]
#define BROWN        egacolors[6]
#define LIGHTGRAY    egacolors[7]
#define DARKGRAY     egacolors[8]
#define LIGHTBLUE    egacolors[9]
#define LIGHTGREEN   egacolors[10]
#define LIGHTCYAN    egacolors[11]
#define LIGHTRED     egacolors[12]
#define LIGHTMAGENTA egacolors[13]
#define YELLOW       egacolors[14]
#define WHITE        egacolors[15]
@end example

Now you can use the defined colors in your code. Note that if you are in color 
table model in a 16 color mode, you have exhausted the color table. Note too 
that this don't work to initialize static variables with a color, because 
egacolors is not initialized. 


@c -----------------------------------------------------------------------------
@node Graphics primitives, Non-clipping graphics primitives, Portable use of a few colors, A User Manual For GRX2
@unnumberedsec Graphics primitives

The screen, the current context or the current clip box can be cleared (i.e. 
set to a desired background color) by using one of the following three 
functions: 

@example
void GrClearScreen(GrColor bg); 
void GrClearcontext(GrColor bg); 
void GrClearClipBox(GrColor bg);
@end example

Thanks to the special GrColor definition, you can do more than simple clear 
with this functions, by example with: 

@example
GrClearScreen( GrWhite()|GrXOR );
@end example

the graphics screen is negativized, do it again and the screen is restored. 

The following line drawing graphics primitives are supported by the library: 

@example
void GrPlot(int x,int y,GrColor c);
void GrLine(int x1,int y1,int x2,int y2,GrColor c); 
void GrHLine(int x1,int x2,int y,GrColor c); 
void GrVLine(int x,int y1,int y2,GrColor c); 
void GrBox(int x1,int y1,int x2,int y2,GrColor c); 
void GrCircle(int xc,int yc,int r,GrColor c); 
void GrEllipse(int xc,int yc,int xa,int ya,GrColor c); 
void GrCircleArc(int xc,int yc,int r,int start,int end,int style,GrColor c); 
void GrEllipseArc(int xc,int yc,int xa,int ya,
                  int start,int end,int style,GrColor c);
void GrPolyLine(int numpts,int points[][2],GrColor c); 
void GrPolygon(int numpts,int points[][2],GrColor c); 
@end example

All primitives operate on the current graphics context. The last argument of 
these functions is always the color to use for the drawing. The HLine and VLine 
primitives are for drawing horizontal and vertical lines. They have been 
included in the library because they are more efficient than the general line 
drawing provided by GrLine. The ellipse primitives can only draw ellipses with 
their major axis parallel with either the X or Y coordinate axis. They take the 
half X and Y axis length in the xa and ya arguments. The arc (circle and 
ellipse) drawing functions take the start and end angles in tenths of degrees 
(i.e. meaningful range: 0 ... 3600). The angles are interpreted 
counter-clockwise starting from the positive X axis. The style argument can be 
one of this defines from grx20.h: 

@example
#define GR_ARC_STYLE_OPEN       0
#define GR_ARC_STYLE_CLOSE1     1
#define GR_ARC_STYLE_CLOSE2     2
@end example

GR_ARC_STYLE_OPEN draws only the arc, GR_ARC_STYLE_CLOSE1 closes the arc with 
a line between his start and end point, GR_ARC_STYLE_CLOSE1 draws the typical 
cake slice. This routine: 

@example
void GrLastArcCoords(int *xs,int *ys,int *xe,int *ye,int *xc,int *yc);
@end example

can be used to retrieve the start, end, and center points used by the last arc 
drawing functions. 

See the @pxref{circtest.c} and @pxref{arctest.c} examples.

The polyline and polygon primitives take the address of an n by 2 coordinate 
array. The X values should be stored in the elements with 0 second index, and 
the Y values in the elements with a second index value of 1. Coordinate arrays 
passed to the polygon primitive can either contain or omit the closing edge of 
the polygon -- the primitive will append it to the list if it is missing. 

See the @pxref{polytest.c} example.

Because calculating the arc points it's a very time consuming operation, there 
are two functions to pre-calculate the points, that can be used next with 
polyline and polygon primitives: 

@example
int  GrGenerateEllipse(int xc,int yc,int xa,int ya,
                       int points[GR_MAX_ELLIPSE_POINTS][2]);
int  GrGenerateEllipseArc(int xc,int yc,int xa,int ya,int start,int end,
                          int points[GR_MAX_ELLIPSE_POINTS][2]);
@end example

The following filled primitives are available: 

@example
void GrFilledBox(int x1,int y1,int x2,int y2,GrColor c); 
void GrFramedBox(int x1,int y1,int x2,int y2,int wdt,GrFBoxColors *c); 
void GrFilledCircle(int xc,int yc,int r,GrColor c); 
void GrFilledEllipse(int xc,int yc,int xa,int ya,GrColor c); 
void GrFilledCircleArc(int xc,int yc,int r,
                       int start,int end,int style,GrColor c);
void GrFilledEllipseArc(int xc,int yc,int xa,int ya,
                        int start,int end,int style,GrColor c);
void GrFilledPolygon(int numpts,int points[][2],GrColor c); 
void GrFilledConvexPolygon(int numpts,int points[][2],GrColor c); 
@end example

Similarly to the line drawing, all of the above primitives operate on the 
current graphics context. The GrFramedBox primitive can be used to draw 
motif-like shaded boxes and "ordinary" framed boxes as well. The x1 through y2 
coordinates specify the interior of the box, the border is outside this area, 
wdt pixels wide. The primitive uses five different colors for the interior and 
four borders of the box which are specified in the GrFBoxColors structure: 

@example
typedef struct @{ 
  GrColor fbx_intcolor; 
  GrColor fbx_topcolor; 
  GrColor fbx_rightcolor; 
  GrColor fbx_bottomcolor; 
  GrColor fbx_leftcolor; 
@} GrFBoxColors;
@end example

The GrFilledConvexPolygon primitive can be used to fill convex polygons. It 
can also be used to fill some concave polygons whose boundaries do not intersect 
any horizontal scan line more than twice. All other concave polygons have to be 
filled with the (somewhat less efficient) GrFilledPolygon primitive. This 
primitive can also be used to fill several disjoint non�overlapping polygons in 
a single operation. 

The function: 

@example
void GrFloodFill(int x, int y, GrColor border, GrColor c);
@end example

flood-fills the area bounded by the color border using x, y like the starting 
point. 

The current color value of any pixel in the current context can be obtained 
with: 

@example
GrColor GrPixel(int x,int y);
@end example

and: 

@example
GrColor GrPixelC(GrContext *c,int x,int y);
@end example

do the same for any context. 

Rectangular areas can be transferred within a context or between contexts by 
calling: 

@example
void GrBitBlt(GrContext *dest,int x,int y,GrContext *source,
              int x1,int y1,int x2,int y2,GrColor op);
@end example

x, y is the position in the destination context, and x1, y1, x2, y2 the area 
from the source context to be transfered. The op argument should be one of 
supported color write modes (GrWRITE, GrXOR, GrOR, GrAND, GrIMAGE), it will 
control how the pixels from the source context are combined with the pixels in 
the destination context (the GrIMAGE op must be ored with the color value to be 
handled as transparent). If either the source or the destination context 
argument is the NULL pointer then the current context is used for that argument. 

See the @pxref{blittest.c} example.

A efficient form to get/put pixels from/to a context can be achieved using the 
next functions: 

@example
const GrColor *GrGetScanline(int x1,int x2,int yy);
const GrColor *GrGetScanlineC(GrContext *ctx,int x1,int x2,int yy);
void GrPutScanline(int x1,int x2,int yy,const GrColor *c, GrColor op);
@end example

The Get functions return a pointer to a static GrColor pixel array (or NULL if 
they fail) with the color values of a row (yy) segment (x1 to x2). GrGetScanline 
uses the current context. GrGestScanlineC uses the context ctx (that can be NULL 
to refer to the current context). Note that the output is only valid until the 
next GRX call. 

GrPutScanline puts the GrColor pixel array c on the yy row segmet defined by 
x1 to x2 in the current context using the op operation. op can be any of 
GrWRITE, GrXOR, GrOR, GrAND or GrIMAGE. Data in c must fit GrCVALUEMASK 
otherwise the results are implementation dependend. So you can't supply 
operation code with the pixel data!. 

@c -----------------------------------------------------------------------------
@node Non-clipping graphics primitives, Customized line drawing, Graphics primitives, A User Manual For GRX2
@unnumberedsec Non-clipping graphics primitives

There is a non-clipping version of some of the elementary primitives. These 
are somewhat more efficient than the regular versions. These are to be used only 
in situations when it is absolutely certain that no drawing will be performed 
beyond the boundaries of the current context. Otherwise the program will almost 
certainly crash! The reason for including these functions is that they are 
somewhat more efficient than the regular, clipping versions. ALSO NOTE: These 
function do not check for conflicts with the mouse cursor. (See the explanation 
about the mouse cursor handling later in this document.) The list of the 
supported non-clipping primitives: 

@example
void GrPlotNC(int x,int y,GrColor c); 
void GrLineNC(int x1,int y1,int x2,int y2,GrColor c); 
void GrHLineNC(int x1,int x2,int y,GrColor c); 
void GrVLineNC(int x,int y1,int y2,GrColor c); 
void GrBoxNC(int x1,int y1,int x2,int y2,GrColor c); 
void GrFilledBoxNC(int x1,int y1,int x2,int y2,GrColor c); 
void GrFramedBoxNC(int x1,int y1,int x2,int y2,int wdt,GrFBoxColors *c); 
void grbitbltNC(GrContext *dst,int x,int y,GrContext *src,
                int x1,int y1,int x2,int y2,GrColor op);
GrColor GrPixelNC(int x,int y);
GrColor GrPixelCNC(GrContext *c,int x,int y);
@end example

@c -----------------------------------------------------------------------------
@node Customized line drawing, Pattern filled graphics primitives, Non-clipping graphics primitives, A User Manual For GRX2
@unnumberedsec Customized line drawing

The basic line drawing graphics primitives described previously always draw 
continuous lines which are one pixel wide. There is another group of line 
drawing functions which can be used to draw wide and/or patterned lines. These 
functions have similar parameter passing conventions as the basic ones with one 
difference: instead of the color value a pointer to a structure of type 
GrLineOption has to be passed to them. The definition of the GrLineOption 
structure: 

@example
typedef struct @{
  GrColor lno_color;             /* color used to draw line */
  int     lno_width;             /* width of the line */
  int     lno_pattlen;           /* length of the dash pattern */
  unsigned char *lno_dashpat;    /* draw/nodraw pattern */
@} GrLineOption;
@end example

The lno_pattlen structure element should be equal to the number of alternating 
draw -- no draw section length values in the array pointed to by the lno_dashpat 
element. The dash pattern array is assumed to begin with a drawn section. If the 
pattern length is equal to zero a continuous line is drawn. 

Example, a white line 3 bits wide (thick) and pattern 6 bits draw, 4 bits nodraw:

@example
GrLineOption mylineop;
...
mylineop.lno_color = GrWhite();
mylineop.lno_width = 3;
mylineop.lno_pattlen = 2;
mylineop.lno_dashpat = "\x06\x04";
@end example

The available custom line drawing primitives: 

@example
void GrCustomLine(int x1,int y1,int x2,int y2,GrLineOption *o); 
void GrCustomBox(int x1,int y1,int x2,int y2,GrLineOption *o); 
void GrCustomCircle(int xc,int yc,int r,GrLineOption *o); 
void GrCustomEllipse(int xc,int yc,int xa,int ya,GrLineOption *o); 
void GrCustomCircleArc(int xc,int yc,int r,
                       int start,int end,int style,GrLineOption *o);
void GrCustomEllipseArc(int xc,int yc,int xa,int ya,
                        int start,int end,int style,GrLineOption *o);
void GrCustomPolyLine(int numpts,int points[][2],GrLineOption *o); 
void GrCustomPolygon(int numpts,int points[][2],GrLineOption *o); 
@end example

See the @pxref{linetest.c} example.

@c -----------------------------------------------------------------------------
@node Pattern filled graphics primitives, Patterned line drawing, Customized line drawing, A User Manual For GRX2
@unnumberedsec Pattern filled graphics primitives

The library also supports a pattern filled version of the basic filled 
primitives described above. These functions have similar parameter passing 
conventions as the basic ones with one difference: instead of the color value a 
pointer to an union of type 'GrPattern' has to be passed to them. The GrPattern 
union can contain either a bitmap or a pixmap fill pattern. The first integer 
slot in the union determines which type it is. Bitmap fill patterns are 
rectangular arrays of bits, each set bit representing the foreground color of 
the fill operation, and each zero bit representing the background. Both the 
foreground and background colors can be combined with any of the supported 
logical operations. Bitmap fill patterns have one restriction: their width must 
be eight pixels. Pixmap fill patterns are very similar to contexts. The relevant 
structure declarations (from grx20.h): 

@example
/*
 * BITMAP: a mode independent way to specify a fill pattern of two
 *   colors. It is always 8 pixels wide (1 byte per scan line), its
 *   height is user-defined. SET THE TYPE FLAG TO ZERO!!!
 */
typedef struct _GR_bitmap @{
  int     bmp_ispixmap;          /* type flag for pattern union */
  int     bmp_height;            /* bitmap height */
  char   *bmp_data;              /* pointer to the bit pattern */
  GrColor bmp_fgcolor;           /* foreground color for fill */
  GrColor bmp_bgcolor;           /* background color for fill */
  int     bmp_memflags;          /* set if dynamically allocated */
@} GrBitmap;

/*
 * PIXMAP: a fill pattern stored in a layout identical to the video RAM
 *   for filling using 'bitblt'-s. It is mode dependent, typically one
 *   of the library functions is used to build it. KEEP THE TYPE FLAG
 *   NONZERO!!!
 */
typedef struct _GR_pixmap @{
  int     pxp_ispixmap;          /* type flag for pattern union */
  int     pxp_width;             /* pixmap width (in pixels)  */
  int     pxp_height;            /* pixmap height (in pixels) */
  GrColor pxp_oper;              /* bitblt mode (SET, OR, XOR, AND, IMAGE) */
  struct _GR_frame pxp_source;   /* source context for fill */
@} GrPixmap;

/*
 * Fill pattern union -- can either be a bitmap or a pixmap
 */
typedef union _GR_pattern @{
  int      gp_ispixmap;          /* nonzero for pixmaps */
  GrBitmap gp_bitmap;            /* fill bitmap */
  GrPixmap gp_pixmap;            /* fill pixmap */
@} GrPattern;

@end example

This define group (from grx20.h) help to acces the GrPattern menbers: 

@example
#define gp_bmp_data                     gp_bitmap.bmp_data
#define gp_bmp_height                   gp_bitmap.bmp_height
#define gp_bmp_fgcolor                  gp_bitmap.bmp_fgcolor
#define gp_bmp_bgcolor                  gp_bitmap.bmp_bgcolor

#define gp_pxp_width                    gp_pixmap.pxp_width
#define gp_pxp_height                   gp_pixmap.pxp_height
#define gp_pxp_oper                     gp_pixmap.pxp_oper
#define gp_pxp_source                   gp_pixmap.pxp_source
@end example

Bitmap patterns can be easily built from initialized character arrays and 
static structures by the C compiler, thus no special support is included in the 
library for creating them. The only action required from the application program 
might be changing the foreground and background colors as needed. Pixmap 
patterns are more difficult to build as they replicate the layout of the video 
memory which changes for different video modes. For this reason the library 
provides three functions to create pixmap patterns in a mode-independent way: 

@example
GrPattern *GrBuildPixmap(char *pixels,int w,int h,GrColorTableP colors);
GrPattern *GrBuildPixmapFromBits(char *bits,int w,int h,
                                 GrColor fgc,GrColor bgc);
GrPattern *GrConvertToPixmap(GrContext *src);
@end example

GrBuildPixmap build a pixmap from a two dimensional (w by h) array of 
characters. The elements in this array are used as indices into the color table 
specified with the argument colors. (This means that pixmaps created this way 
can use at most 256 colors.) The color table pointer: 

@example
typedef GrColor *GrColorTableP;
@end example

should point to an array of integers with the first element being the number of 
colors in the table and the color values themselves starting with the second 
element. NOTE: any color modifiers (GrXOR, GrOR, GrAND) OR-ed to the elements of 
the color table are ignored. 

The GrBuildPixmapFromBits function builds a pixmap fill pattern from bitmap 
data. It is useful if the width of the bitmap pattern is not eight as such 
bitmap patterns can not be used to build a GrBitmap structure. 

The GrConvertToPixmap function converts a graphics context to a pixmap fill 
pattern. It is useful when the pattern can be created with graphics drawing 
operations. NOTE: the pixmap pattern and the original context share the drawing 
RAM, thus if the context is redrawn the fill pattern changes as well. Fill 
patterns which were built by library routines can be destroyed when no longer 
needed (i.e. the space occupied by them can be freed) by calling: 

@example
void GrDestroyPattern(GrPattern *p);
@end example

NOTE: when pixmap fill patterns converted from contexts are destroyed, the 
drawing RAM is not freed. It is freed when the original context is destroyed. 
Fill patterns built by the application have to be destroyed by the application 
as well (if this is needed). 

The list of supported pattern filled graphics primitives is shown below. These 
functions are very similar to their solid filled counterparts, only their last 
argument is different: 

@example
void GrPatternFilledPlot(int x,int y,GrPattern *p);
void GrPatternFilledLine(int x1,int y1,int x2,int y2,GrPattern *p);
void GrPatternFilledBox(int x1,int y1,int x2,int y2,GrPattern *p);
void GrPatternFilledCircle(int xc,int yc,int r,GrPattern *p);
void GrPatternFilledEllipse(int xc,int yc,int xa,int ya,GrPattern *p);
void GrPatternFilledCircleArc(int xc,int yc,int r,int start,int end,
                              int style,GrPattern *p);
void GrPatternFilledEllipseArc(int xc,int yc,int xa,int ya,int start,int end,
                               int style,GrPattern *p);
void GrPatternFilledConvexPolygon(int numpts,int points[][2],GrPattern *p);
void GrPatternFilledPolygon(int numpts,int points[][2],GrPattern *p);
void GrPatternFloodFill(int x, int y, GrColor border, GrPattern *p);
@end example

Strictly speaking the plot and line functions in the above group are not 
filled, but they have been included here for convenience. 

@c -----------------------------------------------------------------------------
@node Patterned line drawing, Image manipulation, Pattern filled graphics primitives, A User Manual For GRX2
@unnumberedsec Patterned line drawing

The custom line drawing functions introduced above also have a version when 
the drawn sections can be filled with a (pixmap or bitmap) fill pattern. To 
achieve this these functions must be passed both a custom line drawing option 
(GrLineOption structure) and a fill pattern (GrPattern union). These two have 
been combined into the GrLinePattern structure: 

@example
typedef struct @{
  GrPattern     *lnp_pattern;    /* fill pattern */
  GrLineOption  *lnp_option;     /* width + dash pattern */
@} GrLinePattern;

@end example

All patterned line drawing functions take a pointer to this structure as their 
last argument. The list of available functions: 

@example
void GrPatternedLine(int x1,int y1,int x2,int y2,GrLinePattern *lp);
void GrPatternedBox(int x1,int y1,int x2,int y2,GrLinePattern *lp);
void GrPatternedCircle(int xc,int yc,int r,GrLinePattern *lp);
void GrPatternedEllipse(int xc,int yc,int xa,int ya,GrLinePattern *lp);
void GrPatternedCircleArc(int xc,int yc,int r,int start,int end,
                          int style,GrLinePattern *lp);
void GrPatternedEllipseArc(int xc,int yc,int xa,int ya,int start,int end,
                           int style,GrLinePattern *lp);
void GrPatternedPolyLine(int numpts,int points[][2],GrLinePattern *lp);
void GrPatternedPolygon(int numpts,int points[][2],GrLinePattern *lp);

@end example

@c -----------------------------------------------------------------------------
@node Image manipulation, Text drawing, Patterned line drawing, A User Manual For GRX2
@unnumberedsec Image manipulation

GRX defines the GrImage type like a GrPixmap synonym: 

@example
#define GrImage GrPixmap
@end example

nevertheless the GrImage type enforces the image character of this object, so 
for compatibility with future GRX versions use the next functions if you need to 
convert between GrImage and GrPixmap objects: 

@example
GrImage *GrImageFromPattern(GrPattern *p);
GrPattern *GrPatternFromImage(GrImage *p);
@end example

the GrImageFromPattern function returns NULL if the GrPattern given is not a 
GrPixmap. 

Like pixmaps patterns images are dependent of the actual video mode set. So 
the library provides functions to create images in a mode-independent way: 

@example
GrImage *GrImageBuild(char *pixels,int w,int h,GrColorTableP colors);
GrImage *GrImageFromContext(GrContext *c);
@end example

these functions work like the GrBuildPixmap and GrConvertToPixmap ones. 
Remember: the image and the original context share the drawing RAM. 

There are a number of functions to display all or part of an image in the 
current context: 

@example
void GrImageDisplay(int x,int y, GrImage *i);
void GrImageDisplayExt(int x1,int y1,int x2,int y2, GrImage *i);
void GrImageFilledBoxAlign(int xo,int yo,int x1,int y1,int x2,int y2,
                           GrImage *p);
void GrImageHLineAlign(int xo,int yo,int x,int y,int width,GrImage *p);
void GrImagePlotAlign(int xo,int yo,int x,int y,GrImage *p);
@end example

GrImageDisplay display the whole image using x, y like the upper left corner 
in the current context. GrImageDisplayExt display as much as it can (repiting 
the image if necesary) in the rectangle defined by x1, y1 and x2, y2. 

GrImageFilledBoxAlign is a most general funtion (really the later two call it) 
display as much as it can in the defined rectangle using xo, yo like the align 
point, it is the virtual point in the destination context (it doesn't need to be 
into the rectangle) with that the upper left image corner is aligned. 

GrImageHLineAlign and GrImagePlotAlign display a row segment or a point of the 
image at x y position using the xo, yo allign point. 

The most usefull image funtions are these:

@example
GrImage *GrImageInverse(GrImage *p,int flag);
GrImage *GrImageStretch(GrImage *p,int nwidth,int nheight);
@end example

GrImageInverse creates a new image object, flipping p left-right or top-down 
as indicated by flag that can be: 

@example
#define GR_IMAGE_INVERSE_LR  0x01  /* inverse left right */
#define GR_IMAGE_INVERSE_TD  0x02  /* inverse top down */
@end example

GrImageStretch creates a new image stretching p to nwidth by nheight. 

To destroy a image objet when you don't need it any more use: 

@example
void GrImageDestroy(GrImage *i);
@end example

See the @pxref{imgtest.c} example.

@c -----------------------------------------------------------------------------
@node Text drawing, Drawing in user coordinates, Image manipulation, A User Manual For GRX2
@unnumberedsec Text drawing

The library supports loadable fonts. When in memory they are bit-mapped (i.e. 
not scalable!) fonts. A driver design allow GRX to load different font formats, 
the last GRX release come with drivers to load the GRX own font format and the 
BGI Borland format for all platforms supported, the X11 version can load X11 
fonts too. 

The GRX distribution come with a font collection in the GRX own format. Some 
of these fonts were converted from VGA fonts. These fonts have all 256 
characters from the PC-437 codepage. Some additional fonts were converted from 
fonts in the MIT X11 distribution. Most of these are ISO-8859-1 coded. Fonts 
also have family names. The following font families are included: 

@example
Font file name       Family  Description
pc<W>x<H>[t].fnt     pc      VGA font, fixed 
xm<W>x<H>[b][i].fnt  X_misc  X11, fixed, miscellaneous group 
char<H>[b][i].fnt    char    X11, proportional, charter family 
cour<H>[b][i].fnt    cour    X11, fixed, courier 
helve<H>[b][i].fnt   helve   X11, proportional, helvetica 
lucb<H>[b][i].fnt    lucb    X11, proportional, lucida bright 
lucs<H>[b][i].fnt    lucs    X11, proportional, lucida sans serif 
luct<H>[b][i].fnt    luct    X11, fixed, lucida typewriter 
ncen<H>[b][i].fnt    ncen    X11, proportional, new century schoolbook 
symb<H>.fnt          symbol  X11, proportional, greek letters, symbols 
tms<H>[b][i].fnt     times   X11, proportional, times
@end example

In the font names <W> means the font width, <H> the font height. Many font 
families have bold and/or italic variants. The files containing these fonts 
contain a 'b' and/or 'i' character in their name just before the extension. 
Additionally, the strings "_bold" and/or "_ital" are appended to the font family 
names. Some of the pc VGA fonts come in thin formats also, these are denoted by 
a 't' in their file names and the string "_thin" in their family names. 

The GrFont structure hold a font in memory. A number of 'pc' fonts are 
built-in to the library and don't need to be loaded: 

@example
extern  GrFont          GrFont_PC6x8;
extern  GrFont          GrFont_PC8x8;
extern  GrFont          GrFont_PC8x14;
extern  GrFont          GrFont_PC8x16;
@end example

Other fonts must be loaded with the GrLoadFont function. If the font file name 
starts with any path separator character or character sequence (':', '/' or '\') 
then it is loaded from the specified directory, otherwise the library try load 
the font first from the current directory and next from the default font path. 
The font path can be set up with the GrSetFontPath function. If the font path is 
not set then the value of the 'GRXFONT' environment variable is used as the font 
path. If GrLoadFont is called again with the name of an already loaded font then 
it will return a pointer to the result of the first loading. Font loading 
routines return NULL if the font was not found. When not needed any more, fonts 
can be unloaded (i.e. the storage occupied by them freed) by calling 
GrUnloadFont. 

The prototype declarations for these functions: 

@example
GrFont *GrLoadFont(char *name);
void GrUnloadFont(GrFont *font);
void GrSetFontPath(char *path_list);
@end example

Using these functions: 

@example
GrFont *GrLoadConvertedFont(char *name,int cvt,int w,int h,
                            int minch,int maxch);
GrFont *GrBuildConvertedFont(GrFont *from,int cvt,int w,int h,
                             int minch,int maxch);
@end example

a new font can be generated from a file font or a font in memory, the 'cvt' 
argument direct the conversion or-ing the desired operations from these defines: 

@example
/*
 * Font conversion flags for 'GrLoadConvertedFont'. OR them as desired.
 */
#define GR_FONTCVT_NONE         0     /* no conversion */
#define GR_FONTCVT_SKIPCHARS    1     /* load only selected characters */
#define GR_FONTCVT_RESIZE       2     /* resize the font */
#define GR_FONTCVT_ITALICIZE    4     /* tilt font for "italic" look */
#define GR_FONTCVT_BOLDIFY      8     /* make a "bold"(er) font  */
#define GR_FONTCVT_FIXIFY       16    /* convert prop. font to fixed wdt */
#define GR_FONTCVT_PROPORTION   32    /* convert fixed font to prop. wdt */
@end example

GR_FONTCVT_SKIPCHARS needs 'minch' and 'maxch' arguments. 

GR_FONTCVT_RESIZE needs 'w' and 'h' arguments. 

The function: 

@example
void GrDumpFont(GrFont *f,char *CsymbolName,char *fileName);
@end example

writes a font to a C source code file, so it can be compiled and linked with a 
user program. GrDumpFont would not normally be used in a release program because 
its purpose is to produce source code. When the source code is compiled and 
linked into a program distributing the font file with the program in not 
necessary, avoiding the possibility of the font file being deleted or corrupted. 

You can use the premade fnt2c.c program (see the source, it's so simple) to 
dump a selected font to source code, by example: 

"fnt2c helv15 myhelv15 nyhelv15.c"

Next, if this line is included in your main include file: 

@example
extern GrFont myhelv15
@end example

and "myhelv15.c" compiled and linked with your project, you can use 'myhelv15' 
in every place a GrFont is required. 

This simple function: 

@example
void GrTextXY(int x,int y,char *text,GrColor fg,GrColor bg);
@end example

draw text in the current context in the standard direction, using the 
GrDefaultFont (mapped in the grx20.h file to the GrFont_PC8x14 font) with x, y 
like the upper left corner and the foreground and background colors given (note 
that bg equal to GrBlack() | GrOR make the background transparent). 

For other functions the GrTextOption structure specifies how to draw a 
character string: 

@example
typedef struct _GR_textOption @{      /* text drawing option structure */
  struct _GR_font     *txo_font;      /* font to be used */
  union  _GR_textColor txo_fgcolor;   /* foreground color */
  union  _GR_textColor txo_bgcolor;   /* background color */
  char    txo_chrtype;                /* character type (see above) */
  char    txo_direct;                 /* direction (see above) */
  char    txo_xalign;                 /* X alignment (see above) */
  char    txo_yalign;                 /* Y alignment (see above) */
@} GrTextOption;

typedef union _GR_textColor @{        /* text color union */
  GrColor       v;                    /* color value for "direct" text */
  GrColorTableP p;                    /* color table for attribute text */
@} GrTextColor;

@end example

The text can be rotated in increments of 90 degrees (txo_direct), alignments 
can be set in both directions (txo_xalign and txo_yalign), and separate fore and 
background colors can be specified. The accepted text direction values: 

@example
#define GR_TEXT_RIGHT           0       /* normal */
#define GR_TEXT_DOWN            1       /* downward */
#define GR_TEXT_LEFT            2       /* upside down, right to left */
#define GR_TEXT_UP              3       /* upward */
#define GR_TEXT_DEFAULT         GR_TEXT_RIGHT
@end example

The accepted horizontal and vertical alignment option values: 

@example
#define GR_ALIGN_LEFT           0       /* X only */
#define GR_ALIGN_TOP            0       /* Y only */
#define GR_ALIGN_CENTER         1       /* X, Y   */
#define GR_ALIGN_RIGHT          2       /* X only */
#define GR_ALIGN_BOTTOM         2       /* Y only */
#define GR_ALIGN_BASELINE       3       /* Y only */
#define GR_ALIGN_DEFAULT        GR_ALIGN_LEFT
@end example

Text strings can be of three different types: one character per byte (i.e. the 
usual C character string, this is the default), one character per 16-bit word 
(suitable for fonts with a large number of characters), and a PC-style 
character-attribute pair. In the last case the GrTextOption structure must 
contain a pointer to a color table of size 16 (fg color bits in attrib) or 8 (bg 
color bits). (The color table format is explained in more detail in the previous 
section explaining the methods to build fill patterns.) The supported text 
types: 

@example
#define GR_BYTE_TEXT            0       /* one byte per character */
#define GR_WORD_TEXT            1       /* two bytes per character */
#define GR_ATTR_TEXT            2       /* chr w/ PC style attribute byte */
@end example

The PC-style attribute text uses the same layout (first byte: character, 
second: attributes) and bitfields as the text mode screen on the PC. The only 
difference is that the 'blink' bit is not supported (it would be very time 
consuming -- the PC text mode does it with hardware support). This bit is used 
instead to control the underlined display of characters. For convenience the 
following attribute manipulation macros have been declared in grx20.h: 

@example
#define GR_BUILD_ATTR(fg,bg,ul) \
        (((fg) & 15) | (((bg) & 7) << 4) | ((ul) ? 128 : 0))
#define GR_ATTR_FGCOLOR(attr)   (((attr)     ) &  15)
#define GR_ATTR_BGCOLOR(attr)   (((attr) >> 4) &   7)
#define GR_ATTR_UNDERLINE(attr) (((attr)     ) & 128)
@end example

Text strings of the types GR_BYTE_TEXT and GR_WORD_TEXT can also be drawn 
underlined. This is controlled by OR-ing the constant GR_UNDERLINE_TEXT to the 
foreground color value: 

@example
#define GR_UNDERLINE_TEXT       (GrXOR << 4)
@end example

After the application initializes a text option structure with the desired 
values it can call one of the following two text drawing functions: 

@example
void GrDrawChar(int chr,int x,int y,GrTextOption *opt);
void GrDrawString(void *text,int length,int x,int y,GrTextOption *opt);
@end example

NOTE: text drawing is fastest when it is drawn in the 'normal' direction, and 
the character does not have to be clipped. It this case the library can use the 
appropriate low-level video RAM access routine, while in any other case the text 
is drawn pixel-by-pixel by the higher-level code. 

There are pattern filed versions too: 

@example
void GrPatternDrawChar(int chr,int x,int y,GrTextOption *opt,GrPattern *p);
void GrPatternDrawString(void *text,int length,int x,int y,GrTextOption *opt,
                         GrPattern *p);
void GrPatternDrawStringExt(void *text,int length,int x,int y,
                            GrTextOption *opt,GrPattern *p);
@end example

The size of a font, a character or a text string can be obtained by calling 
one of the following functions. These functions also take into consideration the 
text direction specified in the text option structure passed to them. 

@example
int  GrFontCharPresent(GrFont *font,int chr);
int  GrFontCharWidth(GrFont *font,int chr);
int  GrFontCharHeight(GrFont *font,int chr);
int  GrFontCharBmpRowSize(GrFont *font,int chr);
int  GrFontCharBitmapSize(GrFont *font,int chr);
int  GrFontStringWidth(GrFont *font,void *text,int len,int type);
int  GrFontStringHeight(GrFont *font,void *text,int len,int type);
int  GrProportionalTextWidth(GrFont *font,void *text,int len,int type);
int  GrCharWidth(int chr,GrTextOption *opt);
int  GrCharHeight(int chr,GrTextOption *opt);
void GrCharSize(int chr,GrTextOption *opt,int *w,int *h);
int  GrStringWidth(void *text,int length,GrTextOption *opt);
int  GrStringHeight(void *text,int length,GrTextOption *opt);
void GrStringSize(void *text,int length,GrTextOption *opt,int *w,int *h);
@end example

  The GrTextRegion structure and its associated functions can be used to 
implement a fast (as much as possible in graphics modes) rectangular text window 
using a fixed font. Clipping for such windows is done in character size 
increments instead of pixels (i.e. no partial characters are drawn). Only fixed 
fonts can be used in their natural size. GrDumpText will cache the code of the 
drawn characters in the buffer pointed to by the 'backup' slot (if it is 
non-NULL) and will draw a character only if the previously drawn character in 
that grid element is different. 

This can speed up text scrolling significantly in graphics modes. The 
supported text types are the same as above. 

@example
typedef struct @{                     /* fixed font text window desc. */
  struct _GR_font     *txr_font;      /* font to be used */
  union  _GR_textColor txr_fgcolor;   /* foreground color */
  union  _GR_textColor txr_bgcolor;   /* background color */
  void   *txr_buffer;                 /* pointer to text buffer */
  void   *txr_backup;                 /* optional backup buffer */
  int     txr_width;                  /* width of area in chars */
  int     txr_height;                 /* height of area in chars */
  int     txr_lineoffset;             /* offset in buffer(s) between rows */
  int     txr_xpos;                   /* upper left corner X coordinate */
  int     txr_ypos;                   /* upper left corner Y coordinate */
  char    txr_chrtype;                /* character type (see above) */
@} GrTextRegion;

void GrDumpChar(int chr,int col,int row,GrTextRegion *r);
void GrDumpText(int col,int row,int wdt,int hgt,GrTextRegion *r);
void GrDumpTextRegion(GrTextRegion *r);

@end example

The GrDumpTextRegion function outputs the whole text region, while GrDumpText 
draws only a user-specified part of it. GrDumpChar updates the character in the 
buffer at the specified location with the new character passed to it as argument 
and then draws the new character on the screen as well. With these functions you 
can simulate a text mode window, write chars directly to the txr_buffer and call 
GrDumpTextRegion when you want to update the window (or GrDumpText if you know 
the area to update is small). 

See the @pxref{fonttest.c} example.

@c -----------------------------------------------------------------------------
@node Drawing in user coordinates, Graphics cursors, Text drawing, A User Manual For GRX2
@unnumberedsec Drawing in user coordinates

 
There is a second set of the graphics primitives which operates in user 
coordinates. Every context has a user to screen coordinate mapping associated 
with it. An application specifies the user window by calling the GrSetUserWindow 
function. 

@example
void GrSetUserWindow(int x1,int y1,int x2,int y2);
@end example

A call to this function it in fact specifies the virtual coordinate limits 
which will be mapped onto the current context regardless of the size of the 
context. For example, the call: 

@example
GrSetUserWindow(0,0,11999,8999);
@end example

tells the library that the program will perform its drawing operations in a 
coordinate system X:0...11999 (width = 12000) and Y:0...8999 (height = 9000). 
This coordinate range will be mapped onto the total area of the current context. 
The virtual coordinate system can also be shifted. For example: 

@example
GrSetUserWindow(5000,2000,16999,10999);
@end example

The user coordinates can even be used to turn the usual left-handed coordinate 
system (0:0 corresponds to the upper left corner) to a right handed one (0:0 
corresponds to the bottom left corner) by calling: 

@example
GrSetUserWindow(0,8999,11999,0); 
@end example

The library also provides three utility functions for the query of the current 
user coordinate limits and for converting user coordinates to screen coordinates 
and vice versa. 

@example
void GrGetUserWindow(int *x1,int *y1,int *x2,int *y2);
void GrGetScreenCoord(int *x,int *y);
void GrGetUserCoord(int *x,int *y);
@end example

If an application wants to take advantage of the user to screen coordinate 
mapping it has to use the user coordinate version of the graphics primitives. 
These have exactly the same parameter passing conventions as their screen 
coordinate counterparts. NOTE: the user coordinate system is not initialized by 
the library! The application has to set up its coordinate mapping before calling 
any of the use coordinate drawing functions -- otherwise the program will almost 
certainly exit (in a quite ungraceful fashion) with a 'division by zero' error. 
The list of supported user coordinate drawing functions: 

@example
void GrUsrPlot(int x,int y,GrColor c);
void GrUsrLine(int x1,int y1,int x2,int y2,GrColor c);
void GrUsrHLine(int x1,int x2,int y,GrColor c);
void GrUsrVLine(int x,int y1,int y2,GrColor c);
void GrUsrBox(int x1,int y1,int x2,int y2,GrColor c);
void GrUsrFilledBox(int x1,int y1,int x2,int y2,GrColor c);
void GrUsrFramedBox(int x1,int y1,int x2,int y2,int wdt,GrFBoxColors *c);
void GrUsrCircle(int xc,int yc,int r,GrColor c);
void GrUsrEllipse(int xc,int yc,int xa,int ya,GrColor c);
void GrUsrCircleArc(int xc,int yc,int r,int start,int end,
                    int style,GrColor c);
void GrUsrEllipseArc(int xc,int yc,int xa,int ya,int start,int end,
                     int style,GrColor c);
void GrUsrFilledCircle(int xc,int yc,int r,GrColor c);
void GrUsrFilledEllipse(int xc,int yc,int xa,int ya,GrColor c);
void GrUsrFilledCircleArc(int xc,int yc,int r,int start,int end,
                          int style,GrColor c);
void GrUsrFilledEllipseArc(int xc,int yc,int xa,int ya,int start,int end,
                           int style,GrColor c);
void GrUsrPolyLine(int numpts,int points[][2],GrColor c);
void GrUsrPolygon(int numpts,int points[][2],GrColor c);
void GrUsrFilledConvexPolygon(int numpts,int points[][2],GrColor c);
void GrUsrFilledPolygon(int numpts,int points[][2],GrColor c);
void GrUsrFloodFill(int x, int y, GrColor border, GrColor c);
GrColor GrUsrPixel(int x,int y);
GrColor GrUsrPixelC(GrContext *c,int x,int y);
void GrUsrCustomLine(int x1,int y1,int x2,int y2,GrLineOption *o);
void GrUsrCustomBox(int x1,int y1,int x2,int y2,GrLineOption *o);
void GrUsrCustomCircle(int xc,int yc,int r,GrLineOption *o);
void GrUsrCustomEllipse(int xc,int yc,int xa,int ya,GrLineOption *o);
void GrUsrCustomCircleArc(int xc,int yc,int r,int start,int end,
                          int style,GrLineOption *o);
void GrUsrCustomEllipseArc(int xc,int yc,int xa,int ya,int start,int end,
                           int style,GrLineOption *o);
void GrUsrCustomPolyLine(int numpts,int points[][2],GrLineOption *o);
void GrUsrCustomPolygon(int numpts,int points[][2],GrLineOption *o);
void GrUsrPatternedLine(int x1,int y1,int x2,int y2,GrLinePattern *lp);
void GrUsrPatternedBox(int x1,int y1,int x2,int y2,GrLinePattern *lp);
void GrUsrPatternedCircle(int xc,int yc,int r,GrLinePattern *lp);
void GrUsrPatternedEllipse(int xc,int yc,int xa,int ya,GrLinePattern *lp);
void GrUsrPatternedCircleArc(int xc,int yc,int r,int start,int end,
                             int style,GrLinePattern *lp);
void GrUsrPatternedEllipseArc(int xc,int yc,int xa,int ya,int start,int end,
                              int style,GrLinePattern *lp);
void GrUsrPatternedPolyLine(int numpts,int points[][2],GrLinePattern *lp);
void GrUsrPatternedPolygon(int numpts,int points[][2],GrLinePattern *lp);
void GrUsrPatternFilledPlot(int x,int y,GrPattern *p);
void GrUsrPatternFilledLine(int x1,int y1,int x2,int y2,GrPattern *p);
void GrUsrPatternFilledBox(int x1,int y1,int x2,int y2,GrPattern *p);
void GrUsrPatternFilledCircle(int xc,int yc,int r,GrPattern *p);
void GrUsrPatternFilledEllipse(int xc,int yc,int xa,int ya,GrPattern *p);
void GrUsrPatternFilledCircleArc(int xc,int yc,int r,int start,int end,int style,GrPattern *p);
void GrUsrPatternFilledEllipseArc(int xc,int yc,int xa,int ya,int start,int end,int style,GrPattern *p);
void GrUsrPatternFilledConvexPolygon(int numpts,int points[][2],GrPattern *p);
void GrUsrPatternFilledPolygon(int numpts,int points[][2],GrPattern *p);
void GrUsrPatternFloodFill(int x, int y, GrColor border, GrPattern *p);
void GrUsrDrawChar(int chr,int x,int y,GrTextOption *opt);
void GrUsrDrawString(char *text,int length,int x,int y,GrTextOption *opt);
void GrUsrTextXY(int x,int y,char *text,GrColor fg,GrColor bg);
@end example

@c -----------------------------------------------------------------------------
@node Graphics cursors, Keyboard input, Drawing in user coordinates, A User Manual For GRX2
@unnumberedsec Graphics cursors

The library provides support for the creation and usage of an unlimited number 
of graphics cursors. An application can use these cursors for any purpose. 
Cursors always save the area they occupy before they are drawn. When moved or 
erased they restore this area. As a general rule of thumb, an application should 
erase a cursor before making changes to an area it occupies and redraw the 
cursor after finishing the drawing. Cursors are created with the GrBuildCursor 
function: 

@example
GrCursor *GrBuildCursor(char far *pixels,int pitch,int w,int h,
                        int xo,int yo,GrColorTableP c);
@end example

The pixels, w (=width), h (=height) and c (= color table) arguments are 
similar to the arguments of the pixmap building library function GrBuildPixmap 
(see that paragraph for a more detailed explanation.), but with two differences. 
First, is not assumed that the pixels data is w x h sized, the pitch argument 
set the offset between rows. Second, the pixmap data is interpreted slightly 
differently, any pixel with value zero is taken as a "transparent" pixel, i.e. 
the background will show through the cursor pattern at that pixel. A pixmap data 
byte with value = 1 will refer to the first color in the table, and so on. 

The xo (= X offset) and yo (= Y offset) arguments specify the position (from 
the top left corner of the cursor pattern) of the cursor's "hot point". 

The GrCursor data structure: 

@example
typedef struct _GR_cursor @{
  struct _GR_context work;            /* work areas (4) */
  int     xcord,ycord;                /* cursor position on screen */
  int     xsize,ysize;                /* cursor size */
  int     xoffs,yoffs;                /* LU corner to hot point offset */
  int     xwork,ywork;                /* save/work area sizes */
  int     xwpos,ywpos;                /* save/work area position on screen */
  int     displayed;                  /* set if displayed */
@} GrCursor;
@end example

is typically not used (i.e. read or changed) by the application program, it 
should just pass pointers to these structures to the appropriate library 
functions. Other cursor manipulation functions: 

@example
void GrDisplayCursor(GrCursor *cursor);
void GrEraseCursor(GrCursor *cursor);
void GrMoveCursor(GrCursor *cursor,int x,int y);
void GrDestroyCursor(GrCursor *cursor);
@end example

See the @pxref{curstest.c} example.

@c -----------------------------------------------------------------------------
@node Keyboard input, Mouse event handling, Graphics cursors, A User Manual For GRX2
@unnumberedsec Keyboard input

GRX can handle platform independant key input. The file grxkeys.h defines the 
keys to be used in the user's program. This is an extract: 

@example
#define GrKey_Control_A            0x0001
#define GrKey_Control_B            0x0002
#define GrKey_Control_C            0x0003
...
#define GrKey_A                    0x0041
#define GrKey_B                    0x0042
#define GrKey_C                    0x0043
...
#define GrKey_F1                   0x013b
#define GrKey_F2                   0x013c
#define GrKey_F3                   0x013d
...
#define GrKey_Alt_F1               0x0168
#define GrKey_Alt_F2               0x0169
#define GrKey_Alt_F3               0x016a
@end example

But you can be confident that the standard ASCII is right maped. 
The GrKeyType type is defined to store keycodes: 

@example
typedef unsigned short GrKeyType;
@end example

This function: 

@example
int GrKeyPressed(void);
@end example

returns non zero if there are any keycode waiting, that can be read with: 

@example
GrKeyType GrKeyRead(void);
@end example

The function: 

@example
int GrKeyStat(void);
@end example

returns a keyboard status word, or-ing it with the next defines it can be known 
the status of some special keys: 

@example
#define GR_KB_RIGHTSHIFT    0x01      /* right shift key depressed */
#define GR_KB_LEFTSHIFT     0x02      /* left shift key depressed */
#define GR_KB_CTRL          0x04      /* CTRL depressed */
#define GR_KB_ALT           0x08      /* ALT depressed */
#define GR_KB_SCROLLOCK     0x10      /* SCROLL LOCK active */
#define GR_KB_NUMLOCK       0x20      /* NUM LOCK active */
#define GR_KB_CAPSLOCK      0x40      /* CAPS LOCK active */
#define GR_KB_INSERT        0x80      /* INSERT state active */
#define GR_KB_SHIFT         (GR_KB_LEFTSHIFT | GR_KB_RIGHTSHIFT)
@end example

See the @pxref{keys.c} example.

@c -----------------------------------------------------------------------------
@node Mouse event handling, Writing/reading PNM graphics files, Keyboard input, A User Manual For GRX2
@unnumberedsec Mouse event handling

All mouse services need the presence of a mouse. An application can test 
whether a mouse is available by calling the function: 

@example
int  GrMouseDetect(void);
@end example

which will return zero if no mouse (or mouse driver) is present, non-zero 
otherwise. The mouse must be initialized by calling one (and only one) of these 
functions: 

@example
void GrMouseInit(void);
void GrMouseInitN(int queue_size);
@end example

GrMouseInit sets a event queue (see below) size to GR_M_QUEU_SIZE (128). A 
user supply event queue size can be set calling GrMouseInitN instead. 

It is a good practice to call GrMouseUnInit before exiting the program. This 
will restore any interrupt vectors hooked by the program to their original 
values. 

@example
void GrMouseUnInit(void);
@end example

The mouse can be controlled with the following functions: 

@example
void GrMouseSetSpeed(int spmult,int spdiv);
void GrMouseSetAccel(int thresh,int accel);
void GrMouseSetLimits(int x1,int y1,int x2,int y2);
void GrMouseGetLimits(int *x1,int *y1,int *x2,int *y2);
void GrMouseWarp(int x,int y);
@end example

The library calculates the mouse position only from the mouse mickey counters. 
(To avoid the limit and 'rounding to the next multiple of eight' problem with 
some mouse driver when it finds itself in a graphics mode unknown to it.) The 
parameters to the GrMouseSetSpeed function specify how coordinate changes are 
obtained from mickey counter changes, multipling by spmult and dividing by 
spdiv. In high resolution graphics modes the value of one just works fine, in 
low resolution modes (320x200 or similar) it is best set the spdiv to two or 
three. (Of course, it also depends on the sensitivity the mouse.) The 
GrMouseSetAccel function is used to control the ballistic effect: if a mouse 
coordinate changes between two samplings by more than the thresh parameter, the 
change is multiplied by the accel parameter. NOTE: some mouse drivers perform 
similar calculations before reporting the coordinates in mickeys. In this case 
the acceleration done by the library will be additional to the one already 
performed by the mouse driver. The limits of the mouse movement can be set 
(passed limits will be clipped to the screen) with GrMouseSetLimits (default is 
the whole screen) and the current limits can be obtained with GrMouseGetLimits. 
GrMouseWarp sets the mouse cursor to the specified position. 

As typical mouse drivers do not know how to draw mouse cursors in high 
resolution graphics modes, the mouse cursor is drawn by the library. The mouse 
cursor can be set with: 

@example
void GrMouseSetCursor(GrCursor *cursor);
void GrMouseSetColors(GrColor fg,GrColor bg);
@end example

GrMouseSetColors uses an internal arrow pattern, the color fg will be used as 
the interior of it and bg will be the border. The current mouse cursor can be 
obtained with: 

@example
GrCursor *GrMouseGetCursor(void);
@end example

The mouse cursor can be displayed/erased with: 

@example
void GrMouseDisplayCursor(void);
void GrMouseEraseCursor(void);
@end example

The mouse cursor can be left permanently displayed. All graphics primitives 
except for the few non-clipping functions check for conflicts with the mouse 
cursor and erase it before the drawing if necessary. Of course, it may be more 
efficient to erase the cursor manually before a long drawing sequence and redraw 
it after completion. The library provides an alternative pair of calls for this 
purpose which will erase the cursor only if it interferes with the drawing: 

@example
int  GrMouseBlock(GrContext *c,int x1,int y1,int x2,int y2);
void GrMouseUnBlock(int return_value_from_GrMouseBlock);
@end example

GrMouseBlock should be passed the context in which the drawing will take place 
(the usual convention of NULL meaning the current context is supported) and the 
limits of the affected area. It will erase the cursor only if it interferes with 
the drawing. When the drawing is finished GrMouseUnBlock must be called with the 
argument returned by GrMouseBlock. 

The status of the mouse cursor can be obtained with calling 
GrMouseCursorIsDisplayed. This function will return non-zero if the cursor is 
displayed, zero if it is erased. 

@example
int  GrMouseCursorIsDisplayed(void);
@end example

The library supports (beside the simple cursor drawing) three types of 
"rubberband" attached to the mouse cursor. The GrMouseSetCursorMode function is 
used to select the cursor drawing mode. 

@example
void GrMouseSetCursorMode(int mode,...);
@end example

The parameter mode can have the following values: 

@example
#define GR_M_CUR_NORMAL   0    /* MOUSE CURSOR modes: just the cursor */
#define GR_M_CUR_RUBBER   1    /* rect. rubber band (XOR-d to the screen) */
#define GR_M_CUR_LINE     2    /* line attached to the cursor */
#define GR_M_CUR_BOX      3    /* rectangular box dragged by the cursor */
@end example

GrMouseSetCursorMode takes different parameters depending on the cursor 
drawing mode selected. The accepted call formats are: 

@example
GrMouseSetCursorMode(M_CUR_NORMAL); 
GrMouseSetCursorMode(M_CUR_RUBBER,xanchor,yanchor,GrColor); 
GrMouseSetCursorMode(M_CUR_LINE,xanchor,yanchor,GrColor); 
GrMouseSetCursorMode(M_CUR_BOX,dx1,dy1,dx2,dy2,GrColor); 
@end example

The anchor parameters for the rubberband and rubberline modes specify a fixed 
screen location to which the other corner of the primitive is bound. The dx1 
through dy2 parameters define the offsets of the corners of the dragged box from 
the hotpoint of the mouse cursor. The color value passed is always XOR-ed to the 
screen, i.e. if an application wants the rubberband to appear in a given color 
on a given background then it has to pass the XOR of these two colors to 
GrMouseSetCursorMode. 

The GrMouseGetEvent function is used to obtain the next mouse or keyboard 
event. It takes a flag with various bits encoding the type of event needed. It 
returns the event in a GrMouseEvent structure. The relevant declarations from 
grx20.h: 

@example
void GrMouseGetEvent(int flags,GrMouseEvent *event);

typedef struct _GR_mouseEvent @{    /* mouse event buffer structure */
  int  flags;                       /* event type flags (see above) */
  int  x,y;                         /* mouse coordinates */
  int  buttons;                     /* mouse button state */
  int  key;                         /* key code from keyboard */
  int  kbstat;                      /* keybd status (ALT, CTRL, etc..) */
  long dtime;                       /* time since last event (msec) */
@} GrMouseEvent;
@end example

The event structure has been extended with a keyboard status word (thus a 
program can check for combinations like ALT-<left mousebutton press>) and a time 
stamp which can be used to check for double clicks, etc... The following macros 
have been defined in grx20.h to help in creating the control flag for 
GrMouseGetEvent and decoding the various bits in the event structure: 

@example
#define GR_M_MOTION         0x001         /* mouse event flag bits */
#define GR_M_LEFT_DOWN      0x002
#define GR_M_LEFT_UP        0x004
#define GR_M_RIGHT_DOWN     0x008
#define GR_M_RIGHT_UP       0x010
#define GR_M_MIDDLE_DOWN    0x020
#define GR_M_MIDDLE_UP      0x040
#define GR_M_BUTTON_DOWN    (GR_M_LEFT_DOWN | GR_M_MIDDLE_DOWN | \
                             GR_M_RIGHT_DOWN)
#define GR_M_BUTTON_UP      (GR_M_LEFT_UP   | GR_M_MIDDLE_UP   | \
                             GR_M_RIGHT_UP)
#define GR_M_BUTTON_CHANGE  (GR_M_BUTTON_UP | GR_M_BUTTON_DOWN )

#define GR_M_LEFT           1             /* mouse button index bits */
#define GR_M_RIGHT          2
#define GR_M_MIDDLE         4

#define GR_M_KEYPRESS       0x080        /* other event flag bits */
#define GR_M_POLL           0x100
#define GR_M_NOPAINT        0x200
#define GR_M_EVENT          (GR_M_MOTION | GR_M_KEYPRESS | \
                             GR_M_BUTTON_DOWN | GR_M_BUTTON_UP)
@end example

GrMouseGetEvent will display the mouse cursor if it was previously erased and 
the GR_M_NOPAINT bit is not set in the flag passed to it. In this case it will 
also erase the cursor after an event has been obtained. 

GrMouseGetEvent block until a event is produced, except if the GR_M_POLL bit 
is set in the falg passed to it. 

Another version of GetEvent: 

@example
void GrMouseGetEventT(int flags,GrMouseEvent *event,long timout_msecs);
@end example

can be istructed to wait timout_msec for the presence of an event. Note that 
event->dtime is only valid if any event occured (event->flags != 0) otherwise 
it's set as -1. Additionally event timing is real world time even in X11 && 
Linux. 

If there are one or more events waiting the function: 

@example
int  GrMousePendingEvent(void);
@end example

returns non-zero value. 

The generation of mouse and keyboard events can be individually enabled or 
disabled (by passing a non-zero or zero, respectively, value in the 
corresponding enable_XX parameter) by calling: 

@example
void MouseEventEnable(int enable_kb,int enable_ms);
@end example

Note that GrMouseInit set both by default. 

See the @pxref{mousetest.c} example.

@node Writing/reading PNM graphics files, BGI interface,Mouse event handling , A User Manual For GRX2
@unnumberedsec Writing/reading PNM graphics files

GRX includes functions to load/save a context from/to a PNM file. 

PNM is a group of simple graphics formats from the NetPbm (http://netpbm.sourceforge.net)
distribution. NetPbm can convert from/to PNM lots of graphics formats, 
and apply some transformations to PNM files.
(Note. You don't need the NetPbm distribution to use this functions). 

There are six PNM formats: 

@example
  P1 text PBM (bitmap) 
  P2 text PGM (gray scale) 
  P3 text PPM (real color) 
  P4 binary PBM (bitmap) 
  P5 binary PGM (gray scale) 
  P6 binary PPM (real color) 
@end example

GRX can handle the binary formats only (get the NetPbm distribution if you 
need convert text to binary formats). 

To save a context in a PNM file you have three functions: 

@example
int GrSaveContextToPbm( GrContext *grc, char *pbmfn, char *docn );
int GrSaveContextToPgm( GrContext *grc, char *pgmfn, char *docn );
int GrSaveContextToPpm( GrContext *grc, char *ppmfn, char *docn );
@end example

they work both in RGB and palette modes, grc must be a pointer to the context to 
be saved, if it is NULL the current context is saved; p-mfn is the file name to 
be created and docn is an optional text comment to be written in the file, it 
can be NULL. Three functions return 0 on succes and -1 on error. 

GrSaveContextToPbm dumps a context in a PBM file (bitmap). If the pixel color 
isn't Black it asumes White. 

GrSaveContextToPgm dumps a context in a PGM file (gray scale). The colors are 
quantized to gray scale using .299r + .587g + .114b. 

GrSaveContextToPpm dumps a context in a PPM file (real color). 
To load a PNM file in a context you must use: 

int GrLoadContextFromPnm( GrContext *grc, char *pnmfn );

it support reading PBM, PGM and PPM binary files. grc must be a pointer to the 
context to be written, is it is NULL the current context is used; p-mfn is the 
file name to be read. If context dimensions are lesser than pnm dimensions, the 
function loads as much as it can. If color mode is not in RGB mode, the routine 
allocates as much colors as it can. The function returns 0 on succes and -1 on 
error. 

Finally to query the file format, width and height of a PNM file you can use: 

int GrQueryPnm( char *ppmfn, int *width, int *height, int *maxval );

pnmfn is the name of pnm file; width returns the pnm width; height returns the 
pnm height; maxval returns the max color component value. The function returns 1 
to 6 on success (the PNM format) or -1 on error. 

@node BGI interface , Test examples,Writing/reading PNM graphics files , A User Manual For GRX2
@unnumberedsec BGI interface


From the 2.3.1 version, GRX includes the BCC2GRX library created by Hartmut 
Schirmer. The BCC2GRX was created to allow users of GRX to compile graphics 
programs written for Borland-C++ and Turbo-C graphics interface. BCC2GRX is not 
a convenient platform to develop new BGI programs. Of course you should use 
native LIBGRX in such cases! 

Read the readme.bgi file for more info. 

@node Test examples, arctest.c,BGI interface , A User Manual For GRX2

@unnumberedsec Test examples
@menu
* arctest.c::
* blittest.c::
* circtest.c::
* cliptest.c::
* colorops.c::
* curstest.c::
* fonttest.c::
* imgtest.c::
* keys.c::
* life.c::
* linetest.c::
* modetest.c::
* mousetest.c::
* pcirctst.c::
* polytest.c::
* rgbtest.c::
* scroltst.c::
* speedtst.c::
* textpatt.c::
* winclip.c::
* wintest.c::
* drawing.h::
* rand.h::
* test.h::
* arctest.dat::
* polytest.dat::
@end menu

@c -----------------------------------------------------------------------------
@node arctest.c, blittest.c,Test examples , Test examples
@unnumberedsec arctest.c

@example
/**
 ** arctest.c ---- test arc outline and filled arc drawing
 **
 ** Copyright (c) 1995 Csaba Biegl, 820 Stirrup Dr, Nashville, TN 37221
 ** The GRX graphics library is free software; you can redistribute it
 ** and/or modify it under some conditions; see the "copying.grx" file
 ** for details.
 **/

#include <string.h>
#include "test.h"  /* @pxref{test.h}  */

TESTFUNC(arctest)
@{
   char buff[300];
   int  xc,yc,xa,ya,start,end;
   FILE *fp;
   GrColor red   = GrAllocColor(255,0,0);
   GrColor green = GrAllocColor(0,255,0);
   GrColor blue  = GrAllocColor(0,0,255);

   fp = fopen("arctest.dat","r");   /* @pxref{arctest.dat}  */
   if(fp == NULL) return;
   while(fgets(buff,299,fp) != NULL) @{
       int len = strlen(buff);
       while(--len >= 0) @{
      if(buff[len] == '\n') @{ buff[len] = '\0'; continue; @}
      if(buff[len] == '\r') @{ buff[len] = '\0'; continue; @}
      break;
       @}
       if(sscanf(buff,
            " arc xc=%d yc=%d xa=%d ya=%d start=%d end=%d",
            &xc,&yc,&xa,&ya,&start,&end) == 6) @{
      GrClearScreen(GrBlack());
      GrEllipse(xc,yc,xa,ya,red);
      GrFilledEllipse(xc,yc,xa,ya,blue);
      GrEllipseArc(xc,yc,xa,ya,start,end,GR_ARC_STYLE_CLOSE2,GrWhite());
      GrTextXY(0,0,buff,GrWhite(),GrNOCOLOR);
      getch();
      GrClearScreen(GrBlack());
      GrEllipseArc(xc,yc,xa,ya,start,end,GR_ARC_STYLE_CLOSE2,red);
      GrFilledEllipseArc(xc,yc,xa,ya,start,end,GR_ARC_STYLE_CLOSE2,green);
      GrTextXY(0,0,buff,GrWhite(),GrNOCOLOR);
      getch();
       @}
   @}
   fclose(fp);
@}

@end example

@c -----------------------------------------------------------------------------
@node blittest.c, circtest.c, arctest.c, Test examples
@unnumberedsec blittest.c

@example
/**
 ** blittest.c ---- test various bitblt-s
 **
 ** Copyright (c) 1995 Csaba Biegl, 820 Stirrup Dr, Nashville, TN 37221
 ** The GRX graphics library is free software; you can redistribute it
 ** and/or modify it under some conditions; see the "copying.grx" file
 ** for details.
 **/

#include <stdlib.h>
#include <string.h>

#include "test.h"  /* @pxref{test.h}  */

#define BHH     (GrScreenY() / 10)
int     BWW =   83;

void drbox(GrContext *src,int x,int y)
@{
   GrColor c1 = GrAllocColor(0,0,255);
   GrColor c2 = GrAllocColor(255,0,0);
   int  xx;

   GrClearScreen(c1);
   GrSetContext(src);
   GrSetClipBox(x-10,y-10,x+BWW-1+10,y+BHH-1+10);
   GrClearClipBox(c2);
   GrSetClipBox(x,y,x+BWW-1,y+BHH-1);
   GrClearClipBox(GrBlack());
   GrBox(x,y,x+BWW-1,y+BHH-1,GrWhite());
   for(xx = x; xx < x+BWW; xx += 5) @{
       GrLine(xx,y,xx+BHH,y+BHH,GrWhite());
       GrLine(xx,y,xx-BHH,y+BHH,GrWhite());
   @}
   GrSetContext(NULL);
   GrResetClipBox();
@}

void doblits(GrContext *src,int x,int y)
@{
   GrColor xc = GrAllocColor(255,255,255) | GrXOR;
   int xx = (GrSizeX() - BWW)/ 2;
   int yy = 2;
   int ii;

   for(ii = 0; ii < 8; ii++) @{
       GrBitBlt(NULL,xx,yy,src,x,y,x+BWW-1,y+BHH-1,GrWRITE);
       xx++;
       yy += (BHH + 2);
   @}
/*
   getch();
   xx = (GrSizeX() - BWW)/ 2;
   yy = 2;
   for(ii = 0; ii < 8; ii++) @{
       GrFilledBox(xx,yy,xx+BWW-1,yy+BHH-1,xc);
       xx++;
       yy += (BHH + 2);
   @}
*/
@}

void bltest(GrContext *src,int x,int y)
@{
   int ii;

   for(ii = 0; ii < 8; ii++) @{
       drbox(src,x,y);
       doblits(src,x,y);
       getch();
       x++;
   @}
@}

void blxtest(void)
@{
   GrContext memc;
   int cw = (BWW + 28) & ~7;
   int ch = BHH + 20;

   bltest(NULL,GrScreenX()-BWW-8,GrScreenY()-BHH);
   bltest(NULL,0,0);
   GrCreateContext(cw,ch,NULL,&memc);
   bltest(&memc,cw-BWW-8,ch-BHH);
@}

TESTFUNC(blittest)
@{
   GrFBoxColors bcolors,ocolors,icolors;
   GrColor c,bg;
   int  x = GrSizeX();
   int  y = GrSizeY();
   int  ww = (x * 2) / 3;
   int  wh = (y * 2) / 3;
   int  ii,jj;
   int  wdt = ww / 150;
   int  bw = x / 17;
   int  bh = y / 17;
   int  bx,by;
   int  cnt;

   GrContext *save = GrCreateSubContext(0,0,GrMaxX(),GrMaxY(),NULL,NULL);
   GrContext *tile = GrCreateContext(bw,bh,NULL,NULL);

   blxtest();
   BWW = 3;
   blxtest();

   bcolors.fbx_intcolor = GrAllocColor(160,100,30);
   bcolors.fbx_topcolor = GrAllocColor(240,150,45);
   bcolors.fbx_leftcolor = GrAllocColor(240,150,45);
   bcolors.fbx_rightcolor = GrAllocColor(80,50,15);
   bcolors.fbx_bottomcolor = GrAllocColor(80,50,15);

   ocolors.fbx_intcolor = GrAllocColor(0,120,100);
   ocolors.fbx_topcolor = GrAllocColor(0,180,150);
   ocolors.fbx_leftcolor = GrAllocColor(0,180,150);
   ocolors.fbx_rightcolor = GrAllocColor(0,90,60);
   ocolors.fbx_bottomcolor = GrAllocColor(0,90,60);

   icolors.fbx_intcolor = bg = GrAllocColor(30,30,30);
   icolors.fbx_bottomcolor = GrAllocColor(0,180,150);
   icolors.fbx_rightcolor = GrAllocColor(0,180,150);
   icolors.fbx_leftcolor = GrAllocColor(0,90,60);
   icolors.fbx_topcolor = GrAllocColor(0,90,60);

   c = GrAllocColor(250,250,0);

   for(ii = 0,by = -(bh/3); ii < 19; ii++) @{
       for(jj = 0,bx = -(bw/2); jj < 19; jj++) @{
      GrFramedBox(bx+2*wdt,by+2*wdt,bx+bw-2*wdt-1,by+bh-2*wdt-1,2*wdt,&bcolors);
      bx += bw;
       @}
       by += bh;
   @}

   GrFramedBox(ww/4-5*wdt-1,wh/4-5*wdt-1,ww/4+5*wdt+ww+1,wh/4+5*wdt+wh+1,wdt,&ocolors);
   GrFramedBox(ww/4-1,wh/4-1,ww/4+ww+1,wh/4+wh+1,wdt,&icolors);

   GrSetClipBox(ww/4,wh/4,ww/4+ww,wh/4+wh);
   drawing(ww/4,wh/4,ww,wh,c,bg);
   getch();

   GrClearScreen(0);
   GrSetContext(save);

   bx = -(bw/2) + 15*bw;
   by = -(bh/3) + 15*bh;

   GrFramedBox(bx+2*wdt,by+2*wdt,bx+bw-2*wdt-1,by+bh-2*wdt-1,2*wdt,&bcolors);

   for (cnt=0; cnt<3; cnt++) @{
     for(ii = 0,by = -(bh/3); ii < 19; ii++) @{
       for(jj = 0,bx = -(bw/2); jj < 19; jj++) @{
      if((ii != 15) || (jj != 15)) @{
          GrBitBlt(save,
         bx,by,
         save,
         -(bw/2) + 15*bw,
         -(bh/3) + 15*bh,
         -(bw/2) + 15*bw + bw - 1,
         -(bh/3) + 15*bh + bh - 1,
         cnt==1 ? GrXOR : GrWRITE
          );
      @}
      bx += bw;
       @}
       by += bh;
     @}
   @}

   GrFramedBox(ww/4-5*wdt-1,wh/4-5*wdt-1,ww/4+5*wdt+ww+1,wh/4+5*wdt+wh+1,wdt,&ocolors);
   GrFramedBox(ww/4-1,wh/4-1,ww/4+ww+1,wh/4+wh+1,wdt,&icolors);

   GrSetClipBox(ww/4,wh/4,ww/4+ww,wh/4+wh);
   drawing(ww/4,wh/4,ww,wh,c,bg);
   getch();


   GrBitBlt(tile,
       0,0,
       save,
       -(bw/2) + 15*bw,
       -(bh/3) + 15*bh,
       -(bw/2) + 15*bw + bw - 1,
       -(bh/3) + 15*bh + bh - 1,
       GrWRITE
   );
   GrSetContext(tile);
   GrFramedBox(2*wdt,2*wdt,bw-2*wdt-1,bh-2*wdt-1,2*wdt,&bcolors);

   GrClearScreen(0);
   GrSetContext(save);

   for(ii = 0,by = -(bh/3); ii < 19; ii++) @{
       for(jj = 0,bx = -(bw/2); jj < 19; jj++) @{
      GrBitBlt(save,
          bx,by,
          tile,
          0,0,
          bw-1,bh-1,
          GrWRITE
      );
      bx += bw;
       @}
       by += bh;
   @}

   GrFramedBox(ww/4-5*wdt-1,wh/4-5*wdt-1,ww/4+5*wdt+ww+1,wh/4+5*wdt+wh+1,wdt,&ocolors);
   GrFramedBox(ww/4-1,wh/4-1,ww/4+ww+1,wh/4+wh+1,wdt,&icolors);

   GrSetClipBox(ww/4,wh/4,ww/4+ww,wh/4+wh);
   drawing(ww/4,wh/4,ww,wh,c,bg);

   getch();
   GrResetClipBox();
   GrBitBlt(NULL,
      60,60,
      NULL,
      20,20,
      GrSizeX() - 40,
      GrSizeY() - 40,
      GrWRITE
   );

   getch();

   GrBitBlt(NULL,
      10,10,
      NULL,
      60,60,
      GrSizeX() - 40,
      GrSizeY() - 40,
      GrWRITE
   );

   getch();

   GrSetContext(tile);
   GrClearContext(0);

   GrBitBlt(tile,
       0,0,
       save,
       -(bw/2),
       -(bh/3),
       -(bw/2) + 15*bw + bw - 1,
       -(bh/3) + 15*bh + bh - 1,
       GrWRITE
   );

   GrSetContext(save);
   GrClearScreen(0);

   for(ii = 0,by = -(bh/3); ii < 18; ii++) @{
       for(jj = 0,bx = -(bw/2); jj < 18; jj++) @{
      GrBitBlt(save,
          bx,by,
          tile,
          0,0,
          bw-1,bh-1,
          GrWRITE
      );
      bx += bw;
       @}
       by += bh;
   @}

   GrFramedBox(ww/4-5*wdt-1,wh/4-5*wdt-1,ww/4+5*wdt+ww+1,wh/4+5*wdt+wh+1,wdt,&ocolors);
   GrFramedBox(ww/4-1,wh/4-1,ww/4+ww+1,wh/4+wh+1,wdt,&icolors);

   GrSetClipBox(ww/4,wh/4,ww/4+ww,wh/4+wh);
   drawing(ww/4,wh/4,ww,wh,c,bg);

   getch();

@}

@end example

@c -----------------------------------------------------------------------------
@node circtest.c, cliptest.c, blittest.c, Test examples
@unnumberedsec circtest.c

@example
/**
 ** circtest.c ---- test circle and ellipse rendering
 **
 ** Copyright (c) 1995 Csaba Biegl, 820 Stirrup Dr, Nashville, TN 37221
 ** The GRX graphics library is free software; you can redistribute it
 ** and/or modify it under some conditions; see the "copying.grx" file
 ** for details.
 **/

#include "test.h" /* @pxref{test.h}  */
#include <math.h>

void drawellip(int xc,int yc,int xa,int ya,GrColor c1,GrColor c2,GrColor c3)
@{
   double ddx = (double)xa;
   double ddy = (double)ya;
   double R2 = ddx*ddx*ddy*ddy;
   double SQ;
   int x1,x2,y1,y2;
   int dx,dy;

   GrFilledBox(xc-xa,yc-ya,xc+xa,yc+ya,c1);
   dx = xa;
   dy = 0;
   GrPlot(xc-dx,yc,c3);
   GrPlot(xc+dx,yc,c3);
   while(++dy <= ya) @{
       SQ = R2 - (double)dy * (double)dy * ddx * ddx;
       dx = (int)(sqrt(SQ)/ddy + 0.5);
       x1 = xc - dx;
       x2 = xc + dx;
       y1 = yc - dy;
       y2 = yc + dy;
       GrPlot(x1,y1,c3);
       GrPlot(x2,y1,c3);
       GrPlot(x1,y2,c3);
       GrPlot(x2,y2,c3);
   @}
   GrEllipse(xc,yc,xa,ya,c2);
@}

TESTFUNC(circtest)
@{
   int     xc,yc;
   int     xr,yr;
   GrColor c1,c2,c3;

   c1 = GrAllocColor(64,64,255);
   c2 = GrAllocColor(255,255,64);
   c3 = GrAllocColor(255,64,64);
   xc = GrSizeX() / 2;
   yc = GrSizeY() / 2;
   xr = 1;
   yr = 1;
   while((xr < 1000) || (yr < 1000)) @{
       drawellip(xc,yc,xr,yr,c1,c2,c3);
       xr += xr/4+1;
       yr += yr/4+1;
       if(getch() == 'q') break;
   @}
   xr = 4;
   yr = 1;
   while((xr < 1000) || (yr < 1000)) @{
       drawellip(xc,yc,xr,yr,c1,c2,c3);
       yr += yr/4+1;
       xr = yr * 4;
       if(getch() == 'q') break;
   @}
   xr = 1;
   yr = 4;
   while((xr < 1000) || (yr < 1000)) @{
       drawellip(xc,yc,xr,yr,c1,c2,c3);
       xr += xr/4+1;
       yr = xr * 4;
       if(getch() == 'q') break;
   @}
@}

@end example

@c -----------------------------------------------------------------------------
@node cliptest.c, colorops.c, circtest.c, Test examples
@unnumberedsec cliptest.c

@example
/**
 ** cliptest.c ---- test clipping
 **
 ** Copyright (c) 1995 Csaba Biegl, 820 Stirrup Dr, Nashville, TN 37221
 ** The GRX graphics library is free software; you can redistribute it
 ** and/or modify it under some conditions; see the "copying.grx" file
 ** for details.
 **/

#include "test.h"   /* @pxref{test.h}  */
#include "rand.h"   /* @pxref{rand.h}  */

TESTFUNC(cliptest)
@{
	long delay;
	int x = GrSizeX();
	int y = GrSizeY();
	int ww = (x * 2) / 3;
	int wh = (y * 2) / 3;
	GrColor c;

	c = GrAllocColor(200,100,100);
	GrBox(ww/4-1,wh/4-1,ww/4+ww+1,wh/4+wh+1,GrWhite());
	GrSetClipBox(ww/4,wh/4,ww/4+ww,wh/4+wh);

	drawing(0,0,ww,wh,c,GrBlack());
	GrKeyRead();

	while(!GrKeyPressed()) @{
	    GrFilledBox(0,0,x,y,GrBlack());
	    drawing(-(RND()%(2*ww))+ww/2,
		-(RND()%(2*wh))+wh/2,
		RND()%(3*ww)+10,
		RND()%(3*wh)+10,
		c,
		GrNOCOLOR
	    );
	    for(delay = 200000L; delay > 0L; delay--);
	@}
	GrKeyRead();
@}

@end example

@c -----------------------------------------------------------------------------
@node colorops.c, curstest.c, cliptest.c, Test examples
@unnumberedsec colorops.c

@example
/**
 ** colorops.c ---- test WRITE, XOR, OR, and AND draw modes
 **
 ** Copyright (c) 1995 Csaba Biegl, 820 Stirrup Dr, Nashville, TN 37221
 ** The GRX graphics library is free software; you can redistribute it
 ** and/or modify it under some conditions; see the "copying.grx" file
 ** for details.
 **/


#include "test.h"   /* @pxref{test.h}  */
#include "rand.h"   /* @pxref{rand.h}  */

TESTFUNC(colorops)
@{
	GrFBoxColors bcolors,ocolors,icolors;
	GrColor bg,c;
	int x = GrSizeX();
	int y = GrSizeY();
	int ww = (x * 2) / 3;
	int wh = (y * 2) / 3;
	int ii,jj;
	int wdt = ww / 150;
	int bw = x / 16;
	int bh = y / 16;
	int bx,by;

	/* This won't work very well under X11 in pseudocolor
	** mode (256 colors or less) if not using a private
	** color map. The missing colors break RGB mode      */
	GrSetRGBcolorMode();

	bcolors.fbx_intcolor = GrAllocColor(160,100,30);
	bcolors.fbx_topcolor = GrAllocColor(240,150,45);
	bcolors.fbx_leftcolor = GrAllocColor(240,150,45);
	bcolors.fbx_rightcolor = GrAllocColor(80,50,15);
	bcolors.fbx_bottomcolor = GrAllocColor(80,50,15);

	ocolors.fbx_intcolor = GrAllocColor(0,120,100);
	ocolors.fbx_topcolor = GrAllocColor(0,180,150);
	ocolors.fbx_leftcolor = GrAllocColor(0,180,150);
	ocolors.fbx_rightcolor = GrAllocColor(0,90,60);
	ocolors.fbx_bottomcolor = GrAllocColor(0,90,60);

	icolors.fbx_intcolor = GrAllocColor(30,30,30);
	icolors.fbx_bottomcolor = GrAllocColor(0,180,150);
	icolors.fbx_rightcolor = GrAllocColor(0,180,150);
	icolors.fbx_leftcolor = GrAllocColor(0,90,60);
	icolors.fbx_topcolor = GrAllocColor(0,90,60);

	c  = GrAllocColor(250,250,0);
	bg = GrNOCOLOR;

	for(ii = 0,by = -(bh / 3); ii < 17; ii++) @{
	    for(jj = 0,bx = (-bw / 2); jj < 17; jj++) @{
		GrFramedBox(bx+2*wdt,by+2*wdt,bx+bw-2*wdt-1,by+bh-2*wdt-1,2*wdt,&bcolors);
		bx += bw;
	    @}
	    by += bh;
	@}

	GrFramedBox(ww/4-5*wdt-1,wh/4-5*wdt-1,ww/4+5*wdt+ww+1,wh/4+5*wdt+wh+1,wdt,&ocolors);
	GrFramedBox(ww/4-1,wh/4-1,ww/4+ww+1,wh/4+wh+1,wdt,&icolors);

	GrSetClipBox(ww/4,wh/4,ww/4+ww,wh/4+wh);

	drawing(ww/4,wh/4,ww,wh,c,bg);
	while(!GrKeyPressed()) @{
	    drawing(ww/4+(RND()%100),
		wh/4+(RND()%100),
		ww,
		wh,
		((RND() / 16) & (GrNumColors() - 1)),
		bg
	    );
	@}
	GrKeyRead();
	GrFramedBox(ww/4-1,wh/4-1,ww/4+ww+1,wh/4+wh+1,wdt,&icolors);
	drawing(ww/4,wh/4,ww,wh,c,bg);
	while(!GrKeyPressed()) @{
	    drawing(ww/4+(RND()%100),
		wh/4+(RND()%100),
		ww,
		wh,
		((RND() / 16) & (GrNumColors() - 1)) | GrXOR,
		bg
	    );
	@}
	GrKeyRead();
	GrFramedBox(ww/4-1,wh/4-1,ww/4+ww+1,wh/4+wh+1,wdt,&icolors);
	drawing(ww/4,wh/4,ww,wh,c,bg);
	while(!GrKeyPressed()) @{
	    drawing(ww/4+(RND()%100),
		wh/4+(RND()%100),
		ww,
		wh,
		((RND() / 16) & (GrNumColors() - 1)) | GrOR,
		bg
	    );
	@}
	GrKeyRead();
	GrFramedBox(ww/4-1,wh/4-1,ww/4+ww+1,wh/4+wh+1,wdt,&icolors);
	drawing(ww/4,wh/4,ww,wh,c,bg);
	while(!GrKeyPressed()) @{
	    drawing(ww/4+(RND()%100),
		wh/4+(RND()%100),
		ww,
		wh,
		((RND() / 16) & (GrNumColors() - 1)) | GrAND,
		bg
	    );
	@}
	GrKeyRead();
@}

@end example

@c -----------------------------------------------------------------------------
@node curstest.c, fonttest.c, colorops.c, Test examples
@unnumberedsec curstest.c

@example
/**
 ** curstest.c ---- test cursors
 **
 ** Copyright (c) 1995 Csaba Biegl, 820 Stirrup Dr, Nashville, TN 37221
 ** The GRX graphics library is free software; you can redistribute it
 ** and/or modify it under some conditions; see the "copying.grx" file
 ** for details.
 **/

#include "test.h"   /* @pxref{test.h}  */

char p16d[] = @{
    0,1,0,0,0,0,0,0,0,0,0,0,1,1,0,0,
    1,2,1,0,0,0,0,0,0,0,0,1,2,2,1,0,
    1,2,2,1,0,0,0,0,0,0,1,2,0,0,2,1,
    1,2,2,2,1,0,0,0,0,0,1,2,0,0,2,1,
    1,2,2,2,2,1,0,0,0,0,0,1,2,2,1,0,
    1,2,2,2,2,2,1,0,0,0,0,0,1,1,0,0,
    1,2,2,2,2,2,2,1,0,0,0,0,0,0,0,0,
    1,2,2,2,2,2,2,2,1,0,0,0,0,0,0,0,
    1,2,2,2,2,2,2,2,2,1,0,0,0,0,0,0,
    1,2,2,2,2,2,2,2,2,2,1,0,0,0,0,0,
    1,2,2,2,2,2,2,2,2,2,2,1,0,0,0,0,
    1,2,2,2,2,1,1,1,1,1,1,0,0,0,0,0,
    1,2,2,2,1,0,0,0,0,0,0,0,0,0,0,0,
    1,2,2,1,0,0,0,0,0,0,0,0,0,0,0,0,
    1,2,1,0,0,0,0,0,0,0,0,0,0,0,0,0,
    0,1,0,0,0,0,0,0,0,0,0,0,0,0,0,0,
@};

TESTFUNC(cursortest)
@{
	GrColor bgc = GrAllocColor(0,0,128);
	GrColor fgc = GrAllocColor(255,255,0);
	GrColor msc[3];
	GrCursor *cur;
	int x,y;

	msc[0] = 2;
	msc[1] = GrWhite();
	msc[2] = GrAllocColor(255,0,0);
	cur = GrBuildCursor(p16d,16,16,16,1,1,msc);
	x = GrScreenX() / 2;
	y = GrScreenY() / 2;
	GrMoveCursor(cur,x,y);
	GrClearScreen(bgc);
	GrSetColor((GrNumColors() - 1),255,255,255);
	drawing(0,0,GrSizeX(),GrSizeY(),fgc,GrNOCOLOR);
	GrFilledBox(0,0,320,120,GrAllocColor(0,255,255));
	GrTextXY( 10,90,"ANDmask",GrBlack(),GrNOCOLOR);
	GrTextXY( 90,90,"ORmask", GrBlack(),GrNOCOLOR);
	GrTextXY(170,90,"Save",   GrBlack(),GrNOCOLOR);
	GrTextXY(250,90,"Work",   GrBlack(),GrNOCOLOR);
	GrDisplayCursor(cur);
	for( ; ; ) @{
	    GrBitBlt(
		NULL,10,10,
		&cur->work,cur->xwork/2,0,cur->xwork/2+cur->xsize-1,cur->ysize-1,
		GrWRITE
	    );
	    GrBitBlt(
		NULL,90,10,
		&cur->work,0,0,cur->xsize-1,cur->ysize-1,
		GrWRITE
	    );
	    GrBitBlt(
		NULL,170,10,
		&cur->work,0,cur->ysize,cur->xwork-1,cur->ysize+cur->ywork-1,
		GrWRITE
	    );
	    GrBitBlt(
		NULL,250,10,
		&cur->work,0,cur->ysize+cur->ywork,cur->xwork-1,cur->ysize+2*cur->ywork-1,
		GrWRITE
	    );
	    GrTextXY(0,GrMaxY()-20,"Type u d l r U D L R or q to quit",GrWhite(),GrNOCOLOR);
	    switch(GrKeyRead()) @{
		case 'u': y--; break;
		case 'd': y++; break;
		case 'l': x--; break;
		case 'r': x++; break;
		case 'U': y -= 10; break;
		case 'D': y += 10; break;
		case 'L': x -= 10; break;
		case 'R': x += 10; break;
		case 'q': return;
		default:  continue;
	    @}
	    if(x < 0) x = 0;
	    if(x > GrScreenX()) x = GrScreenX();
	    if(y < 100) y = 100;
	    if(y > GrScreenY()) y = GrScreenY();
	    GrMoveCursor(cur,x,y);
	@}
@}

@end example

@c -----------------------------------------------------------------------------
@node fonttest.c, imgtest.c, curstest.c, Test examples
@unnumberedsec fonttest.c

@example

/**
 ** fonttest.c ---- test text drawing
 **
 ** Copyright (c) 1995 Csaba Biegl, 820 Stirrup Dr, Nashville, TN 37221
 ** The GRX graphics library is free software; you can redistribute it
 ** and/or modify it under some conditions; see the "copying.grx" file
 ** for details.
 **/

#include <string.h>
#include "test.h"   /* @pxref{test.h}  */


int  cx;
int  cy;
GrColor c1;
GrColor c2;
GrColor c3;
GrColor c4;

char test_text[] = @{
    "QUICK BROWN FOX JUMPS OVER THE LAZY DOG, "
    "quick brown fox jumps over the lazy dog !@@#$%^&*()1234567890"
@};

void displayfont(GrFont *font,char *text,int len)
@{
	GrTextOption opt;
	int ww,hh;
	int bx,by;
	int bw,bh;

	memset(&opt,0,sizeof(opt));
	opt.txo_font   = font;
	opt.txo_xalign = GR_ALIGN_LEFT;
	opt.txo_yalign = GR_ALIGN_TOP;
	GrFilledBox(0,0,GrSizeX(),GrSizeY(),GrBlack());
	opt.txo_direct    = GR_TEXT_RIGHT;
	opt.txo_fgcolor.v = GrBlack();
	opt.txo_bgcolor.v = c1;
	ww = GrStringWidth(text,len,&opt);
	hh = GrStringHeight(text,len,&opt);
	bw = ww+2*hh;
	bh = ww;
	bx = cx - bw/2;
	by = cy - bh/2;
	GrDrawString(text,len,bx+hh,by,&opt);
	opt.txo_direct    = GR_TEXT_DOWN;
	opt.txo_bgcolor.v = c2;
	GrDrawString(text,len,bx+bw-hh,by,&opt);
	opt.txo_direct    = GR_TEXT_LEFT;
	opt.txo_bgcolor.v = c3;
	GrDrawString(text,len,bx+bw-ww-hh,by+bh-hh,&opt);
	opt.txo_direct    = GR_TEXT_UP;
	opt.txo_bgcolor.v = c4;
	GrDrawString(text,len,bx,by+bh-ww,&opt);
	GrKeyRead();
	GrClearClipBox(GrBlack());
	opt.txo_direct    = GR_TEXT_RIGHT;
	opt.txo_fgcolor.v = c1;
	opt.txo_bgcolor.v = GrBlack();
	bx = GrSizeX() / 16;
	by = GrSizeY() / 16;
	bx = (bx + 7) & ~7;
	while(by < GrSizeY()) @{
	    GrDrawString(test_text,strlen(test_text),bx,by,&opt);
	    opt.txo_fgcolor.v ^= GR_UNDERLINE_TEXT;
	    by += hh;
	@}
	GrKeyRead();
@}

TESTFUNC(fonttest)
@{
	GrFont *f;
	int i;
	char buff[100];
	cx = GrSizeX() / 2;
	cy = GrSizeY() / 2;
	c1 = GrAllocColor(100,200,100);
	c2 = GrAllocColor(150,150,100);
	c3 = GrAllocColor(100,100,200);
	c4 = GrAllocColor(100,180,180);
	GrBox(GrSizeX()/16 - 2,
	    GrSizeY()/16 - 2,
	    GrSizeX() - GrSizeX()/16 + 1,
	    GrSizeY() - GrSizeY()/16 + 1,
	    GrAllocColor(250,100,100)
	);
	GrSetClipBox(GrSizeX()/16,
	    GrSizeY()/16,
	    GrSizeX() - GrSizeX()/16 - 1,
	    GrSizeY() - GrSizeY()/16 - 1
	);
	strcpy(buff,"Default GRX font");
	displayfont(&GrDefaultFont,buff,strlen(buff));
	strcpy(buff,"Default font scaled to 6x10");
	displayfont(
	    GrBuildConvertedFont(
		&GrDefaultFont,
		(GR_FONTCVT_SKIPCHARS | GR_FONTCVT_RESIZE),
		6,
		10,
		' ',
		'z'
	    ),
	    buff,
	    strlen(buff)
	);
	strcpy(buff,"Default font scaled to 12x24");
	displayfont(
	    GrBuildConvertedFont(
		&GrDefaultFont,
		(GR_FONTCVT_SKIPCHARS | GR_FONTCVT_RESIZE),
		12,
		24,
		' ',
		'z'
	    ),
	    buff,
	    strlen(buff)
	);
	strcpy(buff,"Default font scaled to 18x36");
	displayfont(
	    GrBuildConvertedFont(
		&GrDefaultFont,
		(GR_FONTCVT_SKIPCHARS | GR_FONTCVT_RESIZE),
		18,
		36,
		' ',
		'z'
	    ),
	    buff,
	    strlen(buff)
	);
	strcpy(buff,"Default font scaled to 10x20 proportional");
	displayfont(
	    GrBuildConvertedFont(
		&GrDefaultFont,
		(GR_FONTCVT_SKIPCHARS | GR_FONTCVT_RESIZE | GR_FONTCVT_PROPORTION),
		10,
		20,
		' ',
		'z'
	    ),
	    buff,
	    strlen(buff)
	);
	strcpy(buff,"Default font scaled to 10x20 bold");
	displayfont(
	    GrBuildConvertedFont(
		&GrDefaultFont,
		(GR_FONTCVT_SKIPCHARS | GR_FONTCVT_RESIZE | GR_FONTCVT_BOLDIFY),
		10,
		20,
		' ',
		'z'
	    ),
	    buff,
	    strlen(buff)
	);
	strcpy(buff,"Default font scaled to 10x20 italic");
	displayfont(
	    GrBuildConvertedFont(
		&GrDefaultFont,
		(GR_FONTCVT_SKIPCHARS | GR_FONTCVT_RESIZE | GR_FONTCVT_ITALICIZE),
		10,
		20,
		' ',
		'z'
	    ),
	    buff,
	    strlen(buff)
	);
	for(i = 0; i < Argc; i++) @{
	    f = GrLoadFont(Argv[i]);
	    if(f) @{
		sprintf(buff,"This is font %s",Argv[i]);
		displayfont(f,buff,strlen(buff));
	    @}
	@}
@}

@end example

@c -----------------------------------------------------------------------------
@node imgtest.c, keys.c, fonttest.c, Test examples
@unnumberedsec imgtest.c

@example
/**
 ** imgtest.c ---- test image functions mapping
 **
 ** Copyright (c) 1998 Hartmut Schirmer
 ** The GRX graphics library is free software; you can redistribute it
 ** and/or modify it under some conditions; see the "copying.grx" file
 ** for details.
 **/

#include "test.h"   /* @pxref{test.h}  */

#define PARTS 4

TESTFUNC(imgtest)
@{
	int  x = GrSizeX();
	int  y = GrSizeY();
	int  ww = (x / PARTS)-1;
	int  wh = (y / PARTS)-1;
	int m1, m2, d1, d2;
	GrColor c1, c2, c3;
	GrContext ctx;
	GrImage *img1;
	GrImage *img2;
	if (! GrCreateContext(ww,wh,NULL,&ctx)) return;

	GrSetContext(&ctx);
	c1 = GrAllocColor(255,100,0);
	c2 = GrAllocColor(0,0,(GrNumColors() >= 16 ? 180 : 63));
	c3 = GrAllocColor(0,255,0);
	drawing(0,0,ww,wh,c1,c2);
	GrBox(0,0,ww-1,wh-1,c1);

	GrSetContext(NULL);

	img1 = GrImageFromContext(&ctx);
	if (!img1) return;

	GrFilledBox(0,0,ww+1,wh+1,c3);
	GrFilledBox(ww+15,0,2*ww+16,wh+1,c3);

	GrBitBlt(NULL,1,1,&ctx,0,0,ww-1,wh-1,0);
	GrImageDisplay(ww+16,1,img1);
	GrImageDisplayExt(0,wh+4,x-1,y-1, img1);

	GrKeyRead();

	for (m1=1; m1 <= PARTS ; m1<<=1) @{
	  for (d1=1; d1 <= PARTS; d1 <<= 1) @{
	    for (m2=1; m2 <= PARTS ; m2<<=1) @{
	      for (d2=1; d2 <= PARTS; d2 <<= 1) @{
		img2 = GrImageStretch(img1,(m1*ww)/d1, (m2*wh)/d2);
		if (img2) @{
		  GrImageDisplayExt(0,0,x-1,y-1,img2);
		  GrImageDestroy(img2);
		@}
	      @}
	    @}
	  @}
	@}
	GrKeyRead();

	/* let's finish with some GrGetScanline / GrPutScanline tests */
	for (d1 = 1; d1 < 32; ++d1) @{
	  for (m1 = wh; m1 < y-wh-d1-1; ++m1) @{
	    const GrColor *cp;
	    cp = GrGetScanline(ww+1,x-ww-d1,m1+1);
	    if (cp) @{
	      GrPutScanline(ww,x-ww-d1-1,m1,cp,GrIMAGE|c2);
	    @}
	  @}
	@}

	GrKeyRead();
@}

@end example

@c -----------------------------------------------------------------------------
@node keys.c, life.c, imgtest.c, Test examples
@unnumberedsec keys.c

@example
/**
 **
 ** keys.c  
 ** Copyright (c) 1995 Csaba Biegl, 820 Stirrup Dr, Nashville, TN 37221
 ** The GRX graphics library is free software; you can redistribute it
 ** and/or modify it under some conditions; see the "copying.grx" file
 ** for details.
 **
 **/

#include <stdio.h>
#include <math.h>
#include <ctype.h>
#include "grx20.h"
#include "grxkeys.h"

#if defined(PENTIUM_CLOCK) && (!defined(__GNUC__) || !defined(__i386__))
#undef PENTIUM_CLOCK
#endif

#ifdef PENTIUM_CLOCK
/******************************************************************************
** This is modified version of keys.c that checks also work of GrKeyPressed()
** and measures time spent in this procedure (at first set divider to clock
** frequency of Your CPU (I have 90MHz): gcc -DPENTIUM_CLOCK=90.
**    A.Pavenis (pavenis@@acad.latnet.lv)
*******************  Some time measurements stuff  ***************************
** Macrodefinition RDTSC reads Pentium CPU timestamp counter. It is counting
** CPU clocks. Attempt to use it under 386 or 486 will cause an invalid
** instruction
*/
#define RDTSC(h,l) __asm__ ("rdtsc" : "=a"(l) , "=d"(h))
inline  long   rdtsc(void)  @{ long h,l; RDTSC(h,l); return l; @}
/* ***************************************************************************/
#endif /* PENTIUM_CLOCK */

#ifdef __GO32__
#include <conio.h>
#include <pc.h>
#endif

#define ISPRINT(k) (((unsigned int)(k)) <= 255 && isprint(k))

typedef struct @{ GrKeyType key;
		 char     *name; @} KeyEntry;

static KeyEntry Keys[] = @{
  @{ GrKey_NoKey              , "GrKey_NoKey" @},
  @{ GrKey_OutsideValidRange  , "GrKey_OutsideValidRange" @},
  @{ GrKey_Control_A          , "GrKey_Control_A" @},
  @{ GrKey_Control_B          , "GrKey_Control_B" @},
  @{ GrKey_Control_C          , "GrKey_Control_C" @},
  @{ GrKey_Control_D          , "GrKey_Control_D" @},
  @{ GrKey_Control_E          , "GrKey_Control_E" @},
  @{ GrKey_Control_F          , "GrKey_Control_F" @},
  @{ GrKey_Control_G          , "GrKey_Control_G" @},
  @{ GrKey_Control_H          , "GrKey_Control_H" @},
  @{ GrKey_Control_I          , "GrKey_Control_I" @},
  @{ GrKey_Control_J          , "GrKey_Control_J" @},
  @{ GrKey_Control_K          , "GrKey_Control_K" @},
  @{ GrKey_Control_L          , "GrKey_Control_L" @},
  @{ GrKey_Control_M          , "GrKey_Control_M" @},
  @{ GrKey_Control_N          , "GrKey_Control_N" @},
  @{ GrKey_Control_O          , "GrKey_Control_O" @},
  @{ GrKey_Control_P          , "GrKey_Control_P" @},
  @{ GrKey_Control_Q          , "GrKey_Control_Q" @},
  @{ GrKey_Control_R          , "GrKey_Control_R" @},
  @{ GrKey_Control_S          , "GrKey_Control_S" @},
  @{ GrKey_Control_T          , "GrKey_Control_T" @},
  @{ GrKey_Control_U          , "GrKey_Control_U" @},
  @{ GrKey_Control_V          , "GrKey_Control_V" @},
  @{ GrKey_Control_W          , "GrKey_Control_W" @},
  @{ GrKey_Control_X          , "GrKey_Control_X" @},
  @{ GrKey_Control_Y          , "GrKey_Control_Y" @},
  @{ GrKey_Control_Z          , "GrKey_Control_Z" @},
  @{ GrKey_Control_LBracket   , "GrKey_Control_LBracket" @},
  @{ GrKey_Control_BackSlash  , "GrKey_Control_BackSlash" @},
  @{ GrKey_Control_RBracket   , "GrKey_Control_RBracket" @},
  @{ GrKey_Control_Caret      , "GrKey_Control_Caret" @},
  @{ GrKey_Control_Underscore , "GrKey_Control_Underscore" @},
  @{ GrKey_Space              , "GrKey_Space" @},
  @{ GrKey_ExclamationPoint   , "GrKey_ExclamationPoint" @},
  @{ GrKey_DoubleQuote        , "GrKey_DoubleQuote" @},
  @{ GrKey_Hash               , "GrKey_Hash" @},
  @{ GrKey_Dollar             , "GrKey_Dollar" @},
  @{ GrKey_Percent            , "GrKey_Percent" @},
  @{ GrKey_Ampersand          , "GrKey_Ampersand" @},
  @{ GrKey_Quote              , "GrKey_Quote" @},
  @{ GrKey_LParen             , "GrKey_LParen" @},
  @{ GrKey_RParen             , "GrKey_RParen" @},
  @{ GrKey_Star               , "GrKey_Star" @},
  @{ GrKey_Plus               , "GrKey_Plus" @},
  @{ GrKey_Comma              , "GrKey_Comma" @},
  @{ GrKey_Dash               , "GrKey_Dash" @},
  @{ GrKey_Period             , "GrKey_Period" @},
  @{ GrKey_Slash              , "GrKey_Slash" @},
  @{ GrKey_0                  , "GrKey_0" @},
  @{ GrKey_1                  , "GrKey_1" @},
  @{ GrKey_2                  , "GrKey_2" @},
  @{ GrKey_3                  , "GrKey_3" @},
  @{ GrKey_4                  , "GrKey_4" @},
  @{ GrKey_5                  , "GrKey_5" @},
  @{ GrKey_6                  , "GrKey_6" @},
  @{ GrKey_7                  , "GrKey_7" @},
  @{ GrKey_8                  , "GrKey_8" @},
  @{ GrKey_9                  , "GrKey_9" @},
  @{ GrKey_Colon              , "GrKey_Colon" @},
  @{ GrKey_SemiColon          , "GrKey_SemiColon" @},
  @{ GrKey_LAngle             , "GrKey_LAngle" @},
  @{ GrKey_Equals             , "GrKey_Equals" @},
  @{ GrKey_RAngle             , "GrKey_RAngle" @},
  @{ GrKey_QuestionMark       , "GrKey_QuestionMark" @},
  @{ GrKey_At                 , "GrKey_At" @},
  @{ GrKey_A                  , "GrKey_A" @},
  @{ GrKey_B                  , "GrKey_B" @},
  @{ GrKey_C                  , "GrKey_C" @},
  @{ GrKey_D                  , "GrKey_D" @},
  @{ GrKey_E                  , "GrKey_E" @},
  @{ GrKey_F                  , "GrKey_F" @},
  @{ GrKey_G                  , "GrKey_G" @},
  @{ GrKey_H                  , "GrKey_H" @},
  @{ GrKey_I                  , "GrKey_I" @},
  @{ GrKey_J                  , "GrKey_J" @},
  @{ GrKey_K                  , "GrKey_K" @},
  @{ GrKey_L                  , "GrKey_L" @},
  @{ GrKey_M                  , "GrKey_M" @},
  @{ GrKey_N                  , "GrKey_N" @},
  @{ GrKey_O                  , "GrKey_O" @},
  @{ GrKey_P                  , "GrKey_P" @},
  @{ GrKey_Q                  , "GrKey_Q" @},
  @{ GrKey_R                  , "GrKey_R" @},
  @{ GrKey_S                  , "GrKey_S" @},
  @{ GrKey_T                  , "GrKey_T" @},
  @{ GrKey_U                  , "GrKey_U" @},
  @{ GrKey_V                  , "GrKey_V" @},
  @{ GrKey_W                  , "GrKey_W" @},
  @{ GrKey_X                  , "GrKey_X" @},
  @{ GrKey_Y                  , "GrKey_Y" @},
  @{ GrKey_Z                  , "GrKey_Z" @},
  @{ GrKey_LBracket           , "GrKey_LBracket" @},
  @{ GrKey_BackSlash          , "GrKey_BackSlash" @},
  @{ GrKey_RBracket           , "GrKey_RBracket" @},
  @{ GrKey_Caret              , "GrKey_Caret" @},
  @{ GrKey_UnderScore         , "GrKey_UnderScore" @},
  @{ GrKey_BackQuote          , "GrKey_BackQuote" @},
  @{ GrKey_a                  , "GrKey_a" @},
  @{ GrKey_b                  , "GrKey_b" @},
  @{ GrKey_c                  , "GrKey_c" @},
  @{ GrKey_d                  , "GrKey_d" @},
  @{ GrKey_e                  , "GrKey_e" @},
  @{ GrKey_f                  , "GrKey_f" @},
  @{ GrKey_g                  , "GrKey_g" @},
  @{ GrKey_h                  , "GrKey_h" @},
  @{ GrKey_i                  , "GrKey_i" @},
  @{ GrKey_j                  , "GrKey_j" @},
  @{ GrKey_k                  , "GrKey_k" @},
  @{ GrKey_l                  , "GrKey_l" @},
  @{ GrKey_m                  , "GrKey_m" @},
  @{ GrKey_n                  , "GrKey_n" @},
  @{ GrKey_o                  , "GrKey_o" @},
  @{ GrKey_p                  , "GrKey_p" @},
  @{ GrKey_q                  , "GrKey_q" @},
  @{ GrKey_r                  , "GrKey_r" @},
  @{ GrKey_s                  , "GrKey_s" @},
  @{ GrKey_t                  , "GrKey_t" @},
  @{ GrKey_u                  , "GrKey_u" @},
  @{ GrKey_v                  , "GrKey_v" @},
  @{ GrKey_w                  , "GrKey_w" @},
  @{ GrKey_x                  , "GrKey_x" @},
  @{ GrKey_y                  , "GrKey_y" @},
  @{ GrKey_z                  , "GrKey_z" @},
  @{ GrKey_LBrace             , "GrKey_LBrace" @},
  @{ GrKey_Pipe               , "GrKey_Pipe" @},
  @{ GrKey_RBrace             , "GrKey_RBrace" @},
  @{ GrKey_Tilde              , "GrKey_Tilde" @},
  @{ GrKey_Control_Backspace  , "GrKey_Control_Backspace" @},
  @{ GrKey_Alt_Escape         , "GrKey_Alt_Escape" @},
  @{ GrKey_Control_At         , "GrKey_Control_At" @},
  @{ GrKey_Alt_Backspace      , "GrKey_Alt_Backspace" @},
  @{ GrKey_BackTab            , "GrKey_BackTab" @},
  @{ GrKey_Alt_Q              , "GrKey_Alt_Q" @},
  @{ GrKey_Alt_W              , "GrKey_Alt_W" @},
  @{ GrKey_Alt_E              , "GrKey_Alt_E" @},
  @{ GrKey_Alt_R              , "GrKey_Alt_R" @},
  @{ GrKey_Alt_T              , "GrKey_Alt_T" @},
  @{ GrKey_Alt_Y              , "GrKey_Alt_Y" @},
  @{ GrKey_Alt_U              , "GrKey_Alt_U" @},
  @{ GrKey_Alt_I              , "GrKey_Alt_I" @},
  @{ GrKey_Alt_O              , "GrKey_Alt_O" @},
  @{ GrKey_Alt_P              , "GrKey_Alt_P" @},
  @{ GrKey_Alt_LBracket       , "GrKey_Alt_LBracket" @},
  @{ GrKey_Alt_RBracket       , "GrKey_Alt_RBracket" @},
  @{ GrKey_Alt_Return         , "GrKey_Alt_Return" @},
  @{ GrKey_Alt_A              , "GrKey_Alt_A" @},
  @{ GrKey_Alt_S              , "GrKey_Alt_S" @},
  @{ GrKey_Alt_D              , "GrKey_Alt_D" @},
  @{ GrKey_Alt_F              , "GrKey_Alt_F" @},
  @{ GrKey_Alt_G              , "GrKey_Alt_G" @},
  @{ GrKey_Alt_H              , "GrKey_Alt_H" @},
  @{ GrKey_Alt_J              , "GrKey_Alt_J" @},
  @{ GrKey_Alt_K              , "GrKey_Alt_K" @},
  @{ GrKey_Alt_L              , "GrKey_Alt_L" @},
  @{ GrKey_Alt_Semicolon      , "GrKey_Alt_Semicolon" @},
  @{ GrKey_Alt_Quote          , "GrKey_Alt_Quote" @},
  @{ GrKey_Alt_Backquote      , "GrKey_Alt_Backquote" @},
  @{ GrKey_Alt_Backslash      , "GrKey_Alt_Backslash" @},
  @{ GrKey_Alt_Z              , "GrKey_Alt_Z" @},
  @{ GrKey_Alt_X              , "GrKey_Alt_X" @},
  @{ GrKey_Alt_C              , "GrKey_Alt_C" @},
  @{ GrKey_Alt_V              , "GrKey_Alt_V" @},
  @{ GrKey_Alt_B              , "GrKey_Alt_B" @},
  @{ GrKey_Alt_N              , "GrKey_Alt_N" @},
  @{ GrKey_Alt_M              , "GrKey_Alt_M" @},
  @{ GrKey_Alt_Comma          , "GrKey_Alt_Comma" @},
  @{ GrKey_Alt_Period         , "GrKey_Alt_Period" @},
  @{ GrKey_Alt_Slash          , "GrKey_Alt_Slash" @},
  @{ GrKey_Alt_KPStar         , "GrKey_Alt_KPStar" @},
  @{ GrKey_F1                 , "GrKey_F1" @},
  @{ GrKey_F2                 , "GrKey_F2" @},
  @{ GrKey_F3                 , "GrKey_F3" @},
  @{ GrKey_F4                 , "GrKey_F4" @},
  @{ GrKey_F5                 , "GrKey_F5" @},
  @{ GrKey_F6                 , "GrKey_F6" @},
  @{ GrKey_F7                 , "GrKey_F7" @},
  @{ GrKey_F8                 , "GrKey_F8" @},
  @{ GrKey_F9                 , "GrKey_F9" @},
  @{ GrKey_F10                , "GrKey_F10" @},
  @{ GrKey_Home               , "GrKey_Home" @},
  @{ GrKey_Up                 , "GrKey_Up" @},
  @{ GrKey_PageUp             , "GrKey_PageUp" @},
  @{ GrKey_Alt_KPMinus        , "GrKey_Alt_KPMinus" @},
  @{ GrKey_Left               , "GrKey_Left" @},
  @{ GrKey_Center             , "GrKey_Center" @},
  @{ GrKey_Right              , "GrKey_Right" @},
  @{ GrKey_Alt_KPPlus         , "GrKey_Alt_KPPlus" @},
  @{ GrKey_End                , "GrKey_End" @},
  @{ GrKey_Down               , "GrKey_Down" @},
  @{ GrKey_PageDown           , "GrKey_PageDown" @},
  @{ GrKey_Insert             , "GrKey_Insert" @},
  @{ GrKey_Delete             , "GrKey_Delete" @},
  @{ GrKey_Shift_F1           , "GrKey_Shift_F1" @},
  @{ GrKey_Shift_F2           , "GrKey_Shift_F2" @},
  @{ GrKey_Shift_F3           , "GrKey_Shift_F3" @},
  @{ GrKey_Shift_F4           , "GrKey_Shift_F4" @},
  @{ GrKey_Shift_F5           , "GrKey_Shift_F5" @},
  @{ GrKey_Shift_F6           , "GrKey_Shift_F6" @},
  @{ GrKey_Shift_F7           , "GrKey_Shift_F7" @},
  @{ GrKey_Shift_F8           , "GrKey_Shift_F8" @},
  @{ GrKey_Shift_F9           , "GrKey_Shift_F9" @},
  @{ GrKey_Shift_F10          , "GrKey_Shift_F10" @},
  @{ GrKey_Control_F1         , "GrKey_Control_F1" @},
  @{ GrKey_Control_F2         , "GrKey_Control_F2" @},
  @{ GrKey_Control_F3         , "GrKey_Control_F3" @},
  @{ GrKey_Control_F4         , "GrKey_Control_F4" @},
  @{ GrKey_Control_F5         , "GrKey_Control_F5" @},
  @{ GrKey_Control_F6         , "GrKey_Control_F6" @},
  @{ GrKey_Control_F7         , "GrKey_Control_F7" @},
  @{ GrKey_Control_F8         , "GrKey_Control_F8" @},
  @{ GrKey_Control_F9         , "GrKey_Control_F9" @},
  @{ GrKey_Control_F10        , "GrKey_Control_F10" @},
  @{ GrKey_Alt_F1             , "GrKey_Alt_F1" @},
  @{ GrKey_Alt_F2             , "GrKey_Alt_F2" @},
  @{ GrKey_Alt_F3             , "GrKey_Alt_F3" @},
  @{ GrKey_Alt_F4             , "GrKey_Alt_F4" @},
  @{ GrKey_Alt_F5             , "GrKey_Alt_F5" @},
  @{ GrKey_Alt_F6             , "GrKey_Alt_F6" @},
  @{ GrKey_Alt_F7             , "GrKey_Alt_F7" @},
  @{ GrKey_Alt_F8             , "GrKey_Alt_F8" @},
  @{ GrKey_Alt_F9             , "GrKey_Alt_F9" @},
  @{ GrKey_Alt_F10            , "GrKey_Alt_F10" @},
  @{ GrKey_Control_Print      , "GrKey_Control_Print" @},
  @{ GrKey_Control_Left       , "GrKey_Control_Left" @},
  @{ GrKey_Control_Right      , "GrKey_Control_Right" @},
  @{ GrKey_Control_End        , "GrKey_Control_End" @},
  @{ GrKey_Control_PageDown   , "GrKey_Control_PageDown" @},
  @{ GrKey_Control_Home       , "GrKey_Control_Home" @},
  @{ GrKey_Alt_1              , "GrKey_Alt_1" @},
  @{ GrKey_Alt_2              , "GrKey_Alt_2" @},
  @{ GrKey_Alt_3              , "GrKey_Alt_3" @},
  @{ GrKey_Alt_4              , "GrKey_Alt_4" @},
  @{ GrKey_Alt_5              , "GrKey_Alt_5" @},
  @{ GrKey_Alt_6              , "GrKey_Alt_6" @},
  @{ GrKey_Alt_7              , "GrKey_Alt_7" @},
  @{ GrKey_Alt_8              , "GrKey_Alt_8" @},
  @{ GrKey_Alt_9              , "GrKey_Alt_9" @},
  @{ GrKey_Alt_0              , "GrKey_Alt_0" @},
  @{ GrKey_Alt_Dash           , "GrKey_Alt_Dash" @},
  @{ GrKey_Alt_Equals         , "GrKey_Alt_Equals" @},
  @{ GrKey_Control_PageUp     , "GrKey_Control_PageUp" @},
  @{ GrKey_F11                , "GrKey_F11" @},
  @{ GrKey_F12                , "GrKey_F12" @},
  @{ GrKey_Shift_F11          , "GrKey_Shift_F11" @},
  @{ GrKey_Shift_F12          , "GrKey_Shift_F12" @},
  @{ GrKey_Control_F11        , "GrKey_Control_F11" @},
  @{ GrKey_Control_F12        , "GrKey_Control_F12" @},
  @{ GrKey_Alt_F11            , "GrKey_Alt_F11" @},
  @{ GrKey_Alt_F12            , "GrKey_Alt_F12" @},
  @{ GrKey_Control_Up         , "GrKey_Control_Up" @},
  @{ GrKey_Control_KPDash     , "GrKey_Control_KPDash" @},
  @{ GrKey_Control_Center     , "GrKey_Control_Center" @},
  @{ GrKey_Control_KPPlus     , "GrKey_Control_KPPlus" @},
  @{ GrKey_Control_Down       , "GrKey_Control_Down" @},
  @{ GrKey_Control_Insert     , "GrKey_Control_Insert" @},
  @{ GrKey_Control_Delete     , "GrKey_Control_Delete" @},
  @{ GrKey_Control_Tab        , "GrKey_Control_Tab" @},
  @{ GrKey_Control_KPSlash    , "GrKey_Control_KPSlash" @},
  @{ GrKey_Control_KPStar     , "GrKey_Control_KPStar" @},
  @{ GrKey_Alt_KPSlash        , "GrKey_Alt_KPSlash" @},
  @{ GrKey_Alt_Tab            , "GrKey_Alt_Tab" @},
  @{ GrKey_Alt_Enter          , "GrKey_Alt_Enter" @},
  @{ GrKey_Alt_LAngle         , "GrKey_Alt_LAngle" @},
  @{ GrKey_Alt_RAngle         , "GrKey_Alt_RAngle" @},
  @{ GrKey_Alt_At             , "GrKey_Alt_At" @},
  @{ GrKey_Alt_LBrace         , "GrKey_Alt_LBrace" @},
  @{ GrKey_Alt_Pipe           , "GrKey_Alt_Pipe" @},
  @{ GrKey_Alt_RBrace         , "GrKey_Alt_RBrace" @},
  @{ GrKey_Print              , "GrKey_Print" @},
  @{ GrKey_Shift_Insert       , "GrKey_Shift_Insert" @},
  @{ GrKey_Shift_Home         , "GrKey_Shift_Home" @},
  @{ GrKey_Shift_End          , "GrKey_Shift_End" @},
  @{ GrKey_Shift_PageUp       , "GrKey_Shift_PageUp" @},
  @{ GrKey_Shift_PageDown     , "GrKey_Shift_PageDown" @},
  @{ GrKey_Alt_Up             , "GrKey_Alt_Up" @},
  @{ GrKey_Alt_Left           , "GrKey_Alt_Left" @},
  @{ GrKey_Alt_Center         , "GrKey_Alt_Center" @},
  @{ GrKey_Alt_Right          , "GrKey_Alt_Right" @},
  @{ GrKey_Alt_Down           , "GrKey_Alt_Down" @},
  @{ GrKey_Alt_Insert         , "GrKey_Alt_Insert" @},
  @{ GrKey_Alt_Delete         , "GrKey_Alt_Delete" @},
  @{ GrKey_Alt_Home           , "GrKey_Alt_Home" @},
  @{ GrKey_Alt_End            , "GrKey_Alt_End" @},
  @{ GrKey_Alt_PageUp         , "GrKey_Alt_PageUp" @},
  @{ GrKey_Alt_PageDown       , "GrKey_Alt_PageDown" @},
  @{ GrKey_Shift_Up           , "GrKey_Shift_Up" @},
  @{ GrKey_Shift_Down         , "GrKey_Shift_Down" @},
  @{ GrKey_Shift_Right        , "GrKey_Shift_Right" @},
  @{ GrKey_Shift_Left         , "GrKey_Shift_Left" @}
@};

#define KEYS (sizeof(Keys)/sizeof(Keys[0]))

int main(void) @{
  int spaces_count = 0;
  KeyEntry *kp;
  GrKeyType k;
  int ok;

  /* need to initialize X11 drivers before using keyboard && mouse functions */
#if (GRX_VERSION_API-0) >= 0x0229
  if ( GrGetLibrarySystem() == GRX_VERSION_GENERIC_X11)
    GrSetMode(GR_320_200_graphics);
#elif (GRX_VERSION == GRX_VERSION_GENERIC_X11)
  GrSetMode(GR_320_200_graphics);
#endif

  printf("\n\n Checking GrKey... style interface"
	   "\n Type 3 spaces to quit the test\n\n");
  while (spaces_count < 3) @{
#ifdef PENTIUM_CLOCK
    int keyPressed=0;
    do
    @{
	static double old_tm = -1.0;
	double  tm;
	unsigned s,e;
	s = rdtsc();
	keyPressed = GrKeyPressed();
	e = rdtsc();
	tm = ((double) (e-s))/(1000.0*PENTIUM_CLOCK);
	if (fabs(tm-old_tm) > 0.01) @{
	  printf ("%5.2f ",tm);
	  fflush (stdout);
	  old_tm = tm;
	@}
    @} while (!keyPressed);
#endif /* PENTIUM_CLOCK */
    k = GrKeyRead();
    if (k == ' ') ++spaces_count; else spaces_count = 0;

    ok = 0;
    for ( kp = Keys; kp < &Keys[KEYS]; ++kp ) @{
      if (k == kp->key) @{
	printf("code 0x%04x\tsymbol %s\n", (unsigned)k, kp->name);
	ok = 1;
	break;
      @}
    @}
    if (!ok)
      printf("code 0x%04x\tsymbol UNKNOWN\n", (unsigned)k);
  @}

  printf("\n\n Now checking getch()"
	   "\n Type 3 spaces to quit the test\n\n");
  spaces_count = 0;
  while (spaces_count < 3) @{
    k = getch();
    if (k == ' ') ++spaces_count; else spaces_count = 0;

    printf("code 0x%02x\tchar ", (unsigned)k);
    if (ISPRINT(k))
      printf("'%c'\n", (char)k);
    else
      printf("not printable\n");

  @}

  printf("\n\n Now checking getkey()"
	   "\n Type 3 spaces to quit the test\n\n");
  spaces_count = 0;
  while (spaces_count < 3) @{
    k = getkey();
    if (k == ' ') ++spaces_count; else spaces_count = 0;

    printf("code 0x%04x\tchar ", (unsigned)k);
    if (ISPRINT(k))
      printf("'%c'\n", (char)k);
    else
      printf("not printable\n");
  @}
  return 0;
@}

@end example

@c -----------------------------------------------------------------------------
@node life.c, linetest.c, keys.c, Test examples
@unnumberedsec life.c

@example
/**
 ** life.c ---- Conway's life program
 **
 ** Copyright (c) 1995 Csaba Biegl, 820 Stirrup Dr, Nashville, TN 37221
 ** The GRX graphics library is free software; you can redistribute it
 ** and/or modify it under some conditions; see the "copying.grx" file
 ** for details.
 **/

#include "test.h"   /* @pxref{test.h}  */
#include "rand.h"   /* @pxref{rand.h}  */

#include <malloc.h>
#include <string.h>
#include <time.h>

TESTFUNC(life)
@{
	int  W = GrSizeX();
	int  H = GrSizeY();
	char **map[2],**old,**cur;
	int  *xp,*xn,*yp,*yn;
	int  which,x,y,gen;
	GrColor c[2];
	long thresh;
	for(which = 0; which < 2; which++) @{
	    cur = malloc(H * sizeof(char *));
	    if(!cur) return;
	    map[which] = cur;
	    for(y = 0; y < H; y++) @{
		cur[y] = malloc(W);
		if(!cur[y]) return;
	    @}
	@}
	xp = malloc(W * sizeof(int));
	xn = malloc(W * sizeof(int));
	yp = malloc(H * sizeof(int));
	yn = malloc(H * sizeof(int));
	if(!xp || !xn || !yp || !yn) return;
	for(x = 0; x < W; x++) @{
	    xp[x] = (x + W - 1) % W;
	    xn[x] = (x + W + 1) % W;
	@}
	for(y = 0; y < H; y++) @{
	    yp[y] = (y + H - 1) % H;
	    yn[y] = (y + H + 1) % H;
	@}
	c[0] = GrBlack();
	c[1] = GrWhite();
	which = 0;
	old = map[which];
	cur = map[1 - which];
	srand((int)time(NULL));
	for(y = 0; y < H; y++) @{
	    for(x = 0; x < W; x++) @{
		int ii = RND() % 53;
		while(--ii >= 0) RND();
		old[y][x] = (((RND() % 131) > 107) ? 1 : 0);
		GrPlotNC(x,y,c[(int)old[y][x]]);
	    @}
	@}
	thresh = (((unsigned long)RND() << 16) + RND()) % 1003567UL;
	gen    = (Argc > 0) ? 1 : 0;
	do @{
	    for(y = 0; y < H; y++) @{
		char *prow = old[yp[y]];
		char *crow = old[y];
		char *nrow = old[yn[y]];
		char *curr = cur[y];
		for(x = 0; x < W; x++) @{
		    int  xprev = xp[x];
		    int  xnext = xn[x];
		    char live  = prow[xprev] +
				 prow[x]     +
				 prow[xnext] +
				 crow[xprev] +
				 crow[xnext] +
				 nrow[xprev] +
				 nrow[x]     +
				 nrow[xnext];
		    live = ((live | crow[x]) == 3) ? 1 : 0;
		    if(--thresh <= 0) @{
			live  ^= gen;
			thresh = (((unsigned long)RND() << 16) + RND()) % 1483567UL;
		    @}
		    curr[x] = live;
		@}
	    @}
	    for(y = 0; y < H; y++) @{
		char *curr = cur[y];
		char *oldr = old[y];
		for(x = 0; x < W; x++) @{
		    if(curr[x] != oldr[x]) GrPlotNC(x,y,c[(int)curr[x]]);
		@}
	    @}
	    which = 1 - which;
	    old = map[which];
	    cur = map[1 - which];
	@} while(!GrKeyPressed());
	while(GrKeyPressed()) GrKeyRead();
@}

@end example

@c -----------------------------------------------------------------------------
@node linetest.c, modetest.c, life.c, Test examples
@unnumberedsec linetest.c

@example
/**
 ** linetest.c ---- test wide and patterned lines
 **
 ** Copyright (c) 1995 Csaba Biegl, 820 Stirrup Dr, Nashville, TN 37221
 ** The GRX graphics library is free software; you can redistribute it
 ** and/or modify it under some conditions; see the "copying.grx" file
 ** for details.
 **/

#include "test.h"   /* @pxref{test.h}  */
#ifdef __GO32__
#include <pc.h>
#endif

TESTFUNC(test1)
@{
	GrLineOption o1,o2,o3,o4;
	int i;
	for(i = 0; i < 2; i++) @{
	    o1.lno_color   = GrAllocColor(255,0,0);
	    o1.lno_width   = 1;
	    o1.lno_pattlen = 4 * i;
	    o1.lno_dashpat = "\5\5\24\24";
	    o2.lno_color   = GrAllocColor(255,255,0);
	    o2.lno_width   = 2;
	    o2.lno_pattlen = 6 * i;
	    o2.lno_dashpat = "\5\5\24\24\2\2";
	    o3.lno_color   = GrAllocColor(0,255,255);
	    o3.lno_width   = 30;
	    o3.lno_pattlen = 8 * i;
	    o3.lno_dashpat = "\5\5\24\24\2\2\40\40";
	    o4.lno_color   = GrAllocColor(255,0,255);
	    o4.lno_width   = 4;
	    o4.lno_pattlen = 6 * i;
	    o4.lno_dashpat = "\2\2\2\2\10\10";
	    GrClearScreen(GrBlack());
	    GrCustomLine(10,10,100,100,&o1);
	    GrCustomLine(10,50,100,140,&o1);
	    GrCustomLine(10,90,100,180,&o1);
	    GrCustomLine(110,10,200,100,&o2);
	    GrCustomLine(110,50,200,140,&o2);
	    GrCustomLine(110,90,200,180,&o2);
	    GrCustomLine(210,10,300,100,&o3);
	    GrCustomLine(210,50,300,140,&o3);
	    GrCustomLine(210,90,300,180,&o3);
	    GrCustomLine(20,300,600,300,&o4);
	    GrCustomLine(20,320,600,340,&o4);
	    GrCustomLine(20,380,600,360,&o4);
	    GrCustomLine(400,100,400,300,&o4);
	    GrCustomLine(420,100,440,300,&o4);
	    GrCustomLine(480,100,460,300,&o4);
	    GrCustomLine(600,200,500,300,&o4);
	    GrKeyRead();
	    GrClearScreen(GrBlack());
	    GrCustomBox(50,50,550,350,&o3);
	    GrCustomCircle(300,200,50,&o2);
	    GrKeyRead();
	@}
@}

@end example

@c -----------------------------------------------------------------------------
@node modetest.c, mousetest.c, linetest.c, Test examples
@unnumberedsec modetest.c

@example
/**
 ** modetest.c ---- test all available graphics modes
 ** Copyright (c) 1995 Csaba Biegl, 820 Stirrup Dr, Nashville, TN 37221
 ** The GRX graphics library is free software; you can redistribute it
 ** and/or modify it under some conditions; see the "copying.grx" file
 ** for details.
 **/

#include <string.h>
#include <stdlib.h>
#include <stdio.h>
#include <ctype.h>

#ifdef __WATCOMC__
#include <conio.h>
#endif

#ifdef __GO32__
#include <pc.h>
#endif

#include "grx20.h"     
#include "drawing.h"   /* @pxref{drawing.h}  */


typedef struct @{
    int  w,h,bpp;
@} gvmode;

gvmode grmodes[200];
int  nmodes = 0;

gvmode *collectmodes(const GrVideoDriver *drv,gvmode *gp)
@{
	GrFrameMode fm;
	const GrVideoMode *mp;
	for(fm =GR_firstGraphicsFrameMode;
	      fm <= GR_lastGraphicsFrameMode; fm++) @{
	    for(mp = GrFirstVideoMode(fm); mp; mp = GrNextVideoMode(mp)) @{
		gp->w   = mp->width;
		gp->h   = mp->height;
		gp->bpp = mp->bpp;
		gp++;
	    @}
	@}
	return(gp);
@}

int vmcmp(const void *m1,const void *m2)
@{
	gvmode *md1 = (gvmode *)m1;
	gvmode *md2 = (gvmode *)m2;
	if(md1->bpp != md2->bpp) return(md1->bpp - md2->bpp);
	if(md1->w   != md2->w  ) return(md1->w   - md2->w  );
	if(md1->h   != md2->h  ) return(md1->h   - md2->h  );
	return(0);
@}

#define LINES   18
#define COLUMNS 80
void ModeText(int i, int shrt,char *mdtxt) @{
	switch (shrt) @{
	  case 2 : sprintf(mdtxt,"%2d) %dx%d ", i+1, grmodes[i].w, grmodes[i].h);
		   break;
	  case 1 : sprintf(mdtxt,"%2d) %4dx%-4d ", i+1, grmodes[i].w, grmodes[i].h);
		   break;
	  default: sprintf(mdtxt,"  %2d)  %4dx%-4d ", i+1, grmodes[i].w, grmodes[i].h);
		   break;
	@}
	mdtxt += strlen(mdtxt);

	if (grmodes[i].bpp > 20)
	  sprintf(mdtxt, "%ldM", 1L << (grmodes[i].bpp-20));
	else  if (grmodes[i].bpp > 10)
	  sprintf(mdtxt, "%ldk", 1L << (grmodes[i].bpp-10));
	else
	  sprintf(mdtxt, "%ld", 1L << grmodes[i].bpp);
	switch (shrt) @{
	  case 2 : break;
	  case 1 : strcat(mdtxt, " col"); break;
	  default: strcat(mdtxt, " colors"); break;
	@}
@}

int ColsCheck(int cols, int ml, int sep) @{
  int len;

  len = ml * cols + (cols-1) * sep + 1;
  return len <= COLUMNS;
@}

void PrintModes(void) @{
	char mdtxt[100];
	unsigned int maxlen;
	int i, n, shrt, c, cols;

	cols = (nmodes+LINES-1) / LINES;
	do @{
	  for (shrt = 0; shrt <= 2; ++shrt) @{
	    maxlen = 0;
	    for (i = 0; i < nmodes; ++i) @{
	      ModeText(i,shrt,mdtxt);
	      if (strlen(mdtxt) > maxlen) maxlen = strlen(mdtxt);
	    @}
	    n = 2;
	    if (cols>1 || shrt<2) @{
	      if (!ColsCheck(cols, maxlen, n)) continue;
	      while (ColsCheck(cols, maxlen, n+1) && n < 4) ++n;
	    @}
	    c = 0;
	    for (i = 0; i < nmodes; ++i) @{
	      if (++c == cols) c = 0;
	      ModeText(i,shrt,mdtxt);
	      printf("%*s%s", (c ? -maxlen-n : -maxlen), mdtxt, (c ? "" : "\n") );
	    @}
	    if (!c) printf("\n");
	    return;
	  @}
	  --cols;
	@} while (1);
@}

int main(void)
@{
	static int firstgr = 1;
	GrSetDriver(NULL);
	if(GrCurrentVideoDriver() == NULL) @{
	    printf("No graphics driver found\n");
	    exit(1);
	@}
	for( ; ; ) @{
	    int  i,w,h,px,py;
	    char m1[40];
	    nmodes = (int)(collectmodes(GrCurrentVideoDriver(),grmodes) - grmodes);
	    GrSetMode(GR_default_text);
	    if(nmodes == 0) @{
		printf("No graphics modes found\n");
		exit(1);
	    @}
	    qsort(grmodes,nmodes,sizeof(grmodes[0]),vmcmp);
	    printf(
		"Graphics driver: \"%s\"\n"
		"  graphics defaults: %dx%d %ld colors\n"
		"  text defaults: %dx%d %ld colors\n\n",
		GrCurrentVideoDriver()->name,
		GrDriverInfo->defgw,
		GrDriverInfo->defgh,
		(long)GrDriverInfo->defgc,
		GrDriverInfo->deftw,
		GrDriverInfo->defth,
		(long)GrDriverInfo->deftc
	    );
	    PrintModes();
	    printf("\nEnter choice #, or anything else to quit> ");
	    fflush(stdout);
	    if(!gets(m1) || (sscanf(m1,"%d",&i) != 1) || (i < 1) || (i > nmodes)) @{
		exit(0);
	    @}
	    if(firstgr) @{
		printf(
		    "When in graphics mode, press <CR> to return to menu.\n"
		    "Now press <CR> to continue..."
		);
		fflush(stdout);
		gets(m1);
		firstgr = 0;
	    @}
	    i--;
	    GrSetMode(
		GR_width_height_color_graphics,
		grmodes[i].w,
		grmodes[i].h,
		1L << grmodes[i].bpp
	    );
	    if(grmodes[i].bpp<15) @{
		w = GrScreenX() >> 1;
		h = GrScreenY() >> 1;
		px = w + 5;
		py = h + 5;
		w -= 10;
		h -= 10;
		drawing(
		    5,5,w,h,
		    GrBlack(),
		    GrWhite()
		);
		drawing(
		    px,5,w,h,
		    GrAllocColor(255,0,0),
		    GrAllocColor(0,255,0)
		);
		drawing(
		    5,py,w,h,
		    GrAllocColor(0,0,255),
		    GrAllocColor(255,255,0)
		);
		drawing(
		    px,py,w,h,
		    GrAllocColor(255,0,255),
		    GrAllocColor(0,255,255)
		);
	    @} else @{
		int y,sx;
		sx=GrScreenX()>>2;
		for(y=0;y<GrScreenY();y++) @{
		    int yy = y & 255;
		    GrHLine(0,sx-1,y,GrBuildRGBcolorT(yy,0,0));
		    GrHLine(sx,2*sx-1,y,GrBuildRGBcolorT(0,yy,0));
		    GrHLine(2*sx,3*sx-1,y,GrBuildRGBcolorT(0,0,yy));
		    GrHLine(3*sx,4*sx-1,y,GrBuildRGBcolorT(yy,yy,yy));
		@}
	    @}
	    kbhit();    /* this is here to flush in the X version 8-) */
	    gets(m1);
	@}
	return 0;
@}

@end example

@c -----------------------------------------------------------------------------
@node mousetest.c, pcirctst.c, modetest.c, Test examples
@unnumberedsec mousetest.c

@example
/**
 ** mousetest.c ---- test mouse cursor and mouse/keyboard input
 **
 **
 ** Copyright (c) 1995 Csaba Biegl, 820 Stirrup Dr, Nashville, TN 37221
 ** [e-mail: csaba@@vuse.vanderbilt.edu]
 **
 ** This is a test/demo file of the GRX graphics library.
 ** You can use GRX test/demo files as you want.
 **
 ** The GRX graphics library is free software; you can redistribute it
 ** and/or modify it under some conditions; see the "copying.grx" file
 ** for details.
 **
 ** This library is distributed in the hope that it will be useful,
 ** but WITHOUT ANY WARRANTY; without even the implied warranty of
 ** MERCHANTABILITY or FITNESS FOR A PARTICULAR PURPOSE.
 **
 **/

#include <string.h>
#include <stdio.h>
#include <ctype.h>
#include "test.h"   /* @pxref{test.h}  */

TESTFUNC(mousetest)
@{
	GrMouseEvent evt;
	GrColor bgc = GrAllocColor(0,0,128);
	GrColor fgc = GrAllocColor(255,255,0);
	int  testmotion = 0;
	int  ii,mode;

	if(GrMouseDetect()) @{
	    GrMouseEventMode(1);
	    GrMouseInit();
	    GrMouseSetColors(GrAllocColor(255,0,0),GrBlack());
	    GrMouseDisplayCursor();
	    GrClearScreen(bgc);
	    ii = 0;
	    mode = GR_M_CUR_NORMAL;
	    GrTextXY(
		10,(GrScreenY() - 20),
		"Commands: 'N' -- next mouse mode, 'Q' -- exit",
		GrWhite(),
		bgc
	    );
	    for( ; ; ) @{
		char msg[200];
		drawing(ii,ii,(GrSizeX() - 20),(GrSizeY() - 20),((fgc ^ bgc) | GrXOR),GrNOCOLOR);
		GrMouseGetEventT(GR_M_EVENT,&evt,0L);
		if(evt.flags & (GR_M_KEYPRESS | GR_M_BUTTON_CHANGE | testmotion)) @{
		    strcpy(msg,"Got event(s): ");
#                   define mend (&msg[strlen(msg)])
		    if(evt.flags & GR_M_MOTION)      strcpy( mend,"[moved] ");
		    if(evt.flags & GR_M_LEFT_DOWN)   strcpy( mend,"[left down] ");
		    if(evt.flags & GR_M_MIDDLE_DOWN) strcpy( mend,"[middle down] ");
		    if(evt.flags & GR_M_RIGHT_DOWN)  strcpy( mend,"[right down] ");
		    if(evt.flags & GR_M_LEFT_UP)     strcpy( mend,"[left up] ");
		    if(evt.flags & GR_M_MIDDLE_UP)   strcpy( mend,"[middle up] ");
		    if(evt.flags & GR_M_RIGHT_UP)    strcpy( mend,"[right up] ");
		    if(evt.flags & GR_M_KEYPRESS)    sprintf(mend,"[key (0x%03x)] ",evt.key);
		    sprintf(mend,"at X=%d, Y=%d, ",evt.x,evt.y);
		    sprintf(mend,
			"buttons=%c%c%c, ",
			(evt.buttons & GR_M_LEFT)   ? 'L' : 'l',
			(evt.buttons & GR_M_MIDDLE) ? 'M' : 'm',
			(evt.buttons & GR_M_RIGHT)  ? 'R' : 'r'
		    );
		    sprintf(mend,"deltaT=%ld (ms)",evt.dtime);
		    strcpy (mend,"                         ");
		    GrTextXY(10,(GrScreenY() - 40),msg,GrWhite(),bgc);
		    testmotion = evt.buttons ? GR_M_MOTION : 0;
		@}
		if(evt.flags & GR_M_KEYPRESS) @{
		    int key = evt.key;
		    if((key == 'Q') || (key == 'q')) break;
		    if((key != 'N') && (key != 'n')) continue;
		    GrMouseEraseCursor();
		    switch(mode = (mode + 1) & 3) @{
		      case GR_M_CUR_RUBBER:
			GrMouseSetCursorMode(GR_M_CUR_RUBBER,evt.x,evt.y,GrWhite() ^ bgc);
			break;
		      case GR_M_CUR_LINE:
			GrMouseSetCursorMode(GR_M_CUR_LINE,evt.x,evt.y,GrWhite() ^ bgc);
			break;
		      case GR_M_CUR_BOX:
			GrMouseSetCursorMode(GR_M_CUR_BOX,-20,-10,20,10,GrWhite() ^ bgc);
			break;
		      default:
			GrMouseSetCursorMode(GR_M_CUR_NORMAL);
			break;
		    @}
		    GrMouseDisplayCursor();
		@}
		if((ii += 7) > 20) ii -= 20;
	    @}
	    GrMouseUnInit();
	@} else @{
	    GrClearScreen(bgc);
	    ii = 0;
	    mode = GR_M_CUR_NORMAL;
	    GrTextXY(
		(GrScreenX()/3),(GrScreenY() - 20),
		"Sorry, no mouse found !",
		GrWhite(),
		bgc
	    );
	@}
@}

@end example

@c -----------------------------------------------------------------------------
@node pcirctst.c, polytest.c, mousetest.c, Test examples
@unnumberedsec pcirctst.c

@example
/**
 ** pcirctst.c ---- test custom circle and ellipse rendering
 ** Copyright (c) 1995 Csaba Biegl, 820 Stirrup Dr, Nashville, TN 37221
 ** [e-mail: csaba@@vuse.vanderbilt.edu]
 **
 ** This is a test/demo file of the GRX graphics library.
 ** You can use GRX test/demo files as you want.
 **
 ** The GRX graphics library is free software; you can redistribute it
 ** and/or modify it under some conditions; see the "copying.grx" file
 ** for details.
 **
 ** This library is distributed in the hope that it will be useful,
 ** but WITHOUT ANY WARRANTY; without even the implied warranty of
 ** MERCHANTABILITY or FITNESS FOR A PARTICULAR PURPOSE.
 ** 
 **/

#include "test.h"   /* @pxref{test.h}  */
#include <math.h>

static int stop = 0;

static int widths[] = @{ 1, 2, 5, 10, 20, 50, 0 @};

static GrLineOption Solid = @{ 0, 1, 0, NULL @};  /* normal solid */

static GrLineOption *Patterns[] = @{
  &Solid, NULL
@};

void drawellip(int xc,int yc,int xa,int ya,GrColor c1,GrColor c2,GrColor c3)
@{
	double ddx = (double)xa;
	double ddy = (double)ya;
	double R2 = ddx*ddx*ddy*ddy;
	double SQ;
	int x1,x2,y1,y2;
	int dx,dy;
	int *wdt, idx;
	GrLineOption *l;

	for (idx = 0, l = *Patterns; l != NULL; l = Patterns[++idx])
	    for (wdt=widths; *wdt != 0; ++wdt) @{
		GrClearScreen(GrBlack());

		GrFilledBox(xc-xa,yc-ya,xc+xa,yc+ya,c1);
		dx = xa;
		dy = 0;
		GrPlot(xc-dx,yc,c3);
		GrPlot(xc+dx,yc,c3);
		while(++dy <= ya) @{
		    SQ = R2 - (double)dy * (double)dy * ddx * ddx;
		    dx = (int)(sqrt(SQ)/ddy + 0.5);
		    x1 = xc - dx;
		    x2 = xc + dx;
		    y1 = yc - dy;
		    y2 = yc + dy;
		    GrPlot(x1,y1,c3);
		    GrPlot(x2,y1,c3);
		    GrPlot(x1,y2,c3);
		    GrPlot(x2,y2,c3);
		@}

		l->lno_color = c2;
		l->lno_width = *wdt;
		GrCustomEllipse(xc,yc,xa,ya,l);
		if(GrKeyRead() == 'q') @{
		  stop = 1;
		  return;
		@}
	    @}
@}

TESTFUNC(circtest)
@{
	int  xc,yc;
	int  xr,yr;
	GrColor c1,c2,c3;

	c1 = GrAllocColor(64,64,255);
	c2 = GrAllocColor(255,255,64);
	c3 = GrAllocColor(255,64,64);
	xc = GrSizeX() / 2;
	yc = GrSizeY() / 2;
	xr = 1;
	yr = 1;
	while(!stop && ((xr < 1000) || (yr < 1000))) @{
	    drawellip(xc,yc,xr,yr,c1,c2,c3);
	    xr += xr/4+1;
	    yr += yr/4+1;
	@}
	xr = 4;
	yr = 1;
	while(!stop && ((xr < 1000) || (yr < 1000))) @{
	    drawellip(xc,yc,xr,yr,c1,c2,c3);
	    yr += yr/4+1;
	    xr = yr * 4;
	@}
	xr = 1;
	yr = 4;
	while(!stop && ((xr < 1000) || (yr < 1000))) @{
	    drawellip(xc,yc,xr,yr,c1,c2,c3);
	    xr += xr/4+1;
	    yr = xr * 4;
	@}
@}

@end example

@c -----------------------------------------------------------------------------
@node polytest.c, rgbtest.c, pcirctst.c, Test examples
@unnumberedsec polytest.c

@example
/**
 ** polytest.c ---- test polygon rendering
 ** Copyright (c) 1995 Csaba Biegl, 820 Stirrup Dr, Nashville, TN 37221
 ** [e-mail: csaba@@vuse.vanderbilt.edu]
 **
 ** This is a test/demo file of the GRX graphics library.
 ** You can use GRX test/demo files as you want.
 **
 ** The GRX graphics library is free software; you can redistribute it
 ** and/or modify it under some conditions; see the "copying.grx" file
 ** for details.
 **
 ** This library is distributed in the hope that it will be useful,
 ** but WITHOUT ANY WARRANTY; without even the implied warranty of
 ** MERCHANTABILITY or FITNESS FOR A PARTICULAR PURPOSE.
 **/

#include <string.h>
#include <stdio.h>
#include <time.h>

#ifndef  CLK_TCK
#define  CLK_TCK    CLOCKS_PER_SEC
#endif

#include "test.h"  /* @pxref{test.h}  */
static GrColor *EGA;
#define black EGA[0]
#define red   EGA[12]
#define blue  EGA[1]
#define white EGA[15]

static void testpoly(int n,int points[][2],int convex)
@{
	GrClearScreen(black);
	GrPolygon(n,points,white);
	GrFilledPolygon(n,points,(red | GrXOR));
	GrKeyRead();
	if(convex || (n <= 3)) @{
	    GrClearScreen(black);
	    GrFilledPolygon(n,points,white);
	    GrFilledConvexPolygon(n,points,(red | GrXOR));
	    GrKeyRead();
	@}
@}

static void speedtest(void)
@{
	int pts[4][2];
	int ww = GrSizeX() / 10;
	int hh = GrSizeY() / 10;
	int sx = (GrSizeX() - 2*ww) / 32;
	int sy = (GrSizeY() - 2*hh) / 32;
	int  ii,jj;
	GrColor color;
	long t1,t2,t3;

	GrClearScreen(black);
	t1 = clock();
	pts[0][1] = 0;
	pts[1][1] = hh;
	pts[2][1] = 2*hh;
	pts[3][1] = hh;
	color = 0;
	for(ii = 0; ii < 32; ii++) @{
	    pts[0][0] = ww;
	    pts[1][0] = 2*ww;
	    pts[2][0] = ww;
	    pts[3][0] = 0;
	    for(jj = 0; jj < 32; jj++) @{
		GrFilledPolygon(4,pts, EGA[color] | GrXOR);
		color = (color + 1) & 15;
		pts[0][0] += sx;
		pts[1][0] += sx;
		pts[2][0] += sx;
		pts[3][0] += sx;
	    @}
	    pts[0][1] += sy;
	    pts[1][1] += sy;
	    pts[2][1] += sy;
	    pts[3][1] += sy;
	@}
	t2 = clock();
	pts[0][1] = 0;
	pts[1][1] = hh;
	pts[2][1] = 2*hh;
	pts[3][1] = hh;
	color = 0;
	for(ii = 0; ii < 32; ii++) @{
	    pts[0][0] = ww;
	    pts[1][0] = 2*ww;
	    pts[2][0] = ww;
	    pts[3][0] = 0;
	    for(jj = 0; jj < 32; jj++) @{
		GrFilledConvexPolygon(4,pts, EGA[color] | GrXOR);
		color = (color + 1) & 15;
		pts[0][0] += sx;
		pts[1][0] += sx;
		pts[2][0] += sx;
		pts[3][0] += sx;
	    @}
	    pts[0][1] += sy;
	    pts[1][1] += sy;
	    pts[2][1] += sy;
	    pts[3][1] += sy;
	@}
	t3 = clock();
	sprintf(exit_message,
	    "Times to scan 1024 polygons\n"
	    "   with 'GrFilledPolygon': %.2f (s)\n"
	    "   with 'GrFilledConvexPolygon': %.2f (s)\n",
	    (double)(t2 - t1) / (double)CLK_TCK,
	    (double)(t3 - t2) / (double)CLK_TCK
	);
@}

TESTFUNC(ptest)
@{
	char buff[300];
	int  pts[300][2];
	int  ii,collect;
	int  convex;
	FILE *fp;

	fp = fopen("polytest.dat","r");   /* @pxref{polytest.dat}  */
	if(fp == NULL) return;
	EGA = GrAllocEgaColors();
	ii  = collect = convex = 0;
	while(fgets(buff,299,fp) != NULL) @{
	    if(!collect) @{
		if(strncmp(buff,"begin",5) == 0) @{
		    convex  = (buff[5] == 'c');
		    collect = 1;
		    ii      = 0;
		@}
		continue;
	    @}
	    if(strncmp(buff,"end",3) == 0) @{
		testpoly(ii,pts,convex);
		collect = 0;
		continue;
	    @}
	    if(sscanf(buff,"%d %d",&pts[ii][0],&pts[ii][1]) == 2) ii++;
	@}
	fclose(fp);
	speedtest();
@}

@end example

@c -----------------------------------------------------------------------------
@node rgbtest.c, scroltst.c, polytest.c, Test examples
@unnumberedsec rgbtest.c

@example
/**
 ** rgbtest.c ---- show 256 color RGB palette 
 **
 ** Copyright (c) 1995 Csaba Biegl, 820 Stirrup Dr, Nashville, TN 37221
 ** [e-mail: csaba@@vuse.vanderbilt.edu]
 **
 ** This is a test/demo file of the GRX graphics library.
 ** You can use GRX test/demo files as you want.
 **
 ** The GRX graphics library is free software; you can redistribute it
 ** and/or modify it under some conditions; see the "copying.grx" file
 ** for details.
 **
 ** This library is distributed in the hope that it will be useful,
 ** but WITHOUT ANY WARRANTY; without even the implied warranty of
 ** MERCHANTABILITY or FITNESS FOR A PARTICULAR PURPOSE.
 **/

#include "test.h"   /* @pxref{test.h}  */

TESTFUNC(rgbtest)
@{
	int x = GrSizeX();
	int y = GrSizeY();
	int ww = (x-10)/32;
	int wh = (y-10)/8;
	int ii,jj;

	GrSetRGBcolorMode();
	for(ii = 0; ii < 8; ii++) @{
	    for(jj = 0; jj < 32; jj++) @{
		GrFilledBox(5+jj*ww,5+ii*wh,5+jj*ww+ww-1,5+ii*wh+wh-1,ii*32+jj);
	    @}
	@}
	GrKeyRead();
@}

@end example

@c -----------------------------------------------------------------------------
@node scroltst.c, speedtst.c, rgbtest.c, Test examples
@unnumberedsec scroltst.c

@example
/**
 ** scroltst.c ---- test virtual screen set/scroll
 ** Copyright (c) 1995 Csaba Biegl, 820 Stirrup Dr, Nashville, TN 37221
 ** [e-mail: csaba@@vuse.vanderbilt.edu]
 **
 ** This is a test/demo file of the GRX graphics library.
 ** You can use GRX test/demo files as you want.
 **
 ** The GRX graphics library is free software; you can redistribute it
 ** and/or modify it under some conditions; see the "copying.grx" file
 ** for details.
 **
 ** This library is distributed in the hope that it will be useful,
 ** but WITHOUT ANY WARRANTY; without even the implied warranty of
 ** MERCHANTABILITY or FITNESS FOR A PARTICULAR PURPOSE.
 **/

#include "test.h"   /* @pxref{test.h}  */

TESTFUNC(scrolltest)
@{
	int  wdt = GrScreenX();
	int  hgt = GrScreenY();
	GrColor nc  = GrNumColors();
	int  txh = GrDefaultFont.h.height + 2;
	for( ; ; ) @{
	    char buff[100];
	    char *l1 = "Screen resolution: %dx%d";
	    char *l2 = "Virtual resolution: %dx%d";
	    char *l3 = "Current screen start: x=%-4d y=%-4d";
	    char *l4 = "Commands: q -- exit program,";
	    char *l5 = "w W h H -- shrink/GROW screen width or height,";
	    char *l6 = "x X y Y -- decrease/INCREASE screen start position";
	    GrColor bgc = GrAllocColor(0,0,128);
	    GrColor fgc = GrAllocColor(200,200,0);
	    GrColor txc = GrAllocColor(255,0,255);
	    int vw = GrVirtualX();
	    int vh = GrVirtualY();
	    int vx = GrViewportX();
	    int vy = GrViewportY();
	    int x  = (vw / 2) - (strlen(l6) * GrDefaultFont.h.width / 2);
	    int y  = (vh / 2) - (3 * txh);
	    GrClearScreen(bgc);
	    drawing(0,0,vw,vh,fgc,bgc);
	    sprintf(buff,l1,wdt,hgt); GrTextXY(x,y,buff,txc,bgc); y += txh;
	    sprintf(buff,l2,vw, vh ); GrTextXY(x,y,buff,txc,bgc); y += txh;
	    for( ; ; GrSetViewport(vx,vy)) @{
		int yy = y;
		vx = GrViewportX();
		vy = GrViewportY();
		sprintf(buff,l3,vx,vy); GrTextXY(x,yy,buff,txc,bgc); yy += txh;
		GrTextXY(x,yy,l4,txc,bgc); yy += txh;
		GrTextXY(x,yy,l5,txc,bgc); yy += txh;
		GrTextXY(x,yy,l6,txc,bgc); yy += txh;
		switch(GrKeyRead()) @{
		    case 'w': vw -= 8; break;
		    case 'W': vw += 8; break;
		    case 'h': vh -= 8; break;
		    case 'H': vh += 8; break;
		    case 'x': vx--; continue;
		    case 'X': vx++; continue;
		    case 'y': vy--; continue;
		    case 'Y': vy++; continue;
		    case 'q': return;
		    case 'Q': return;
		    default:  continue;
		@}
		GrSetMode(GR_custom_graphics,wdt,hgt,nc,vw,vh);
		break;
	    @}
	@}
@}

@end example

@c -----------------------------------------------------------------------------
@node speedtst.c, textpatt.c, scroltst.c, Test examples
@unnumberedsec speedtst.c

@example
/**
 ** speedtst.c ---- check all available frame drivers speed
 ** Copyright (c) 1995 Csaba Biegl, 820 Stirrup Dr, Nashville, TN 37221
 ** [e-mail: csaba@@vuse.vanderbilt.edu]
 **
 ** This is a test/demo file of the GRX graphics library.
 ** You can use GRX test/demo files as you want.
 **
 ** The GRX graphics library is free software; you can redistribute it
 ** and/or modify it under some conditions; see the "copying.grx" file
 ** for details.
 **
 ** This library is distributed in the hope that it will be useful,
 ** but WITHOUT ANY WARRANTY; without even the implied warranty of
 ** MERCHANTABILITY or FITNESS FOR A PARTICULAR PURPOSE.
 **/

#include <string.h>
#include <stdlib.h>
#include <stdio.h>
#include <ctype.h>
#include <assert.h>
#ifdef __WATCOMC__
/*#include <wcdefs.h>*/
#include <conio.h>
#else
#include <values.h>
#endif
#ifdef __GO32__
#include <conio.h>
#include <pc.h>
#endif
#include <math.h>
#include <time.h>

#include "rand.h"   /* @pxref{rand.h}  */

#include "grx20.h"  
#if GRX_VERSION_API-0 <= 0x0220
#define GrColor unsigned long
#define BLIT_FAIL(gp) ((gp)->fm!=GR_frameVGA8X)
#else
#define BLIT_FAIL(gp)  0
#endif

#define MEASURE_RAM_MODES 1

#define READPIX_loops      (384*1)
#define READPIX_X11_loops  (  4*1)
#define DRAWPIX_loops      (256*1)
#define DRAWLIN_loops      (12*1)
#define DRAWHLIN_loops     (16*1)
#define DRAWVLIN_loops     (12*1)
#define DRAWBLK_loops      (1*1)
#define BLIT_loops         (1*1)

typedef struct @{
    double rate, count;
@} perfm;

typedef struct @{
    GrFrameMode fm;
    int    w,h,bpp;
    int    flags;
    perfm  readpix;
    perfm  drawpix;
    perfm  drawlin;
    perfm  drawhlin;
    perfm  drawvlin;
    perfm  drawblk;
    perfm  blitv2v;
    perfm  blitv2r;
    perfm  blitr2v;
@} gvmode;
#define FLG_measured 0x0001
#define FLG_tagged   0x0002
#define FLG_rammode  0x0004
#define MEASURED(g) (((g)->flags&FLG_measured)!=0)
#define TAGGED(g)   (((g)->flags&FLG_tagged)!=0)
#define RAMMODE(g)  (((g)->flags&FLG_rammode)!=0)
#define SET_MEASURED(g)  (g)->flags |= FLG_measured
#define SET_TAGGED(g)    (g)->flags |= FLG_tagged
#define SET_RAMMODE(g)   (g)->flags |= FLG_rammode
#define TOGGLE_TAGGED(g) (g)->flags ^= FLG_tagged

int  nmodes = 0;
#define MAX_MODES 256
gvmode *grmodes = NULL;
#if MEASURE_RAM_MODES
gvmode *rammodes = NULL;
#endif

/* No of Points [(x,y) pairs]. Must be multiple of 2*3=6 */
#define PAIRS 4200

#define UL(x)  ((unsigned long)(x))
#define DBL(x)  ((double)(x))
#define INT(x) ((int)(x))

#ifndef  CLK_TCK
#define  CLK_TCK    CLOCKS_PER_SEC
#endif

#ifndef min
#define min(a,b) ((a)<(b) ? (a) : (b))
#endif
#ifndef max
#define max(a,b) ((a)>(b) ? (a) : (b))
#endif

typedef struct XYpairs @{
  int x[PAIRS];
  int y[PAIRS];
  int w, h;
  struct XYpairs *nxt;
@} XY_PAIRS;

XY_PAIRS *xyp = NULL;
int *xb = NULL, *yb = NULL; /* need sorted pairs for block operations */
int measured_any = 0;

XY_PAIRS *checkpairs(int w, int h) @{
  XY_PAIRS *res = xyp;
  int i;

  if (xb == NULL) @{
    xb = malloc(sizeof(int) * PAIRS);
    yb = malloc(sizeof(int) * PAIRS);
  @}

  while (res != NULL) @{
    if (res->w == w && res->h == h)
      return res;
    res = res->nxt;
  @}

  SRND(12345);

  res = malloc(sizeof(XY_PAIRS));
  assert(res != NULL);
  res->w = w;
  res->h = h;
  res->nxt = xyp;
  xyp = res;
  for (i=0; i < PAIRS; ++i) @{
    int x = RND() % w;
    int y = RND() % h;
    if (x < 0) x = 0; else
    if (x >=w) x = w-1;
    if (y < 0) y = 0; else
    if (y >=h) y = h-1;
    res->x[i] = x;
    res->y[i] = y;
  @}
  return res;
@}

double SQR(int a, int b) @{
  double r = DBL(a-b);
  return r*r;
@}

double ABS(int a, int b) @{
  double r = DBL(a-b);
  return fabs(r);
@}

char *FrameDriverName(GrFrameMode m) @{

#if GRX_VERSION_API-0 >= 0x0229
  int x11 = GrGetLibrarySystem() == GRX_VERSION_GENERIC_X11;
#else
# define x11 (GRX_VERSION == GRX_VERSION_GENERIC_X11)
#endif

  switch(m) @{
    case GR_frameUndef: return "Undef";
    case GR_frameText : return "Text";
    case GR_frameHERC1: return "HERC1";
    case GR_frameEGAVGA1: return x11 ? "XWIN1" : "EGAVGA1";
    case GR_frameEGA4: return "EGA4";
    case GR_frameSVGA4: return x11 ? "XWIN4" : "SVGA4";
    case GR_frameSVGA8: return x11 ? "XWIN8" : "SVGA8";
    case GR_frameVGA8X: return "VGA8X";
    case GR_frameSVGA16: return x11 ? "XWIN16" : "SVGA16";
    case GR_frameSVGA24: return x11 ? "XWIN24" : "SVGA24";
    case GR_frameSVGA32L: return x11 ? "XWIN32L" : "SVGA32L";
    case GR_frameSVGA32H: return x11 ? "XWIN32H" : "SVGA32H";
    case GR_frameSVGA8_LFB: return "LFB8";
    case GR_frameSVGA16_LFB: return "LFB16";
    case GR_frameSVGA24_LFB: return "LFB24";
    case GR_frameSVGA32L_LFB: return "LFB32L";
    case GR_frameSVGA32H_LFB: return "LFB32H";
    case GR_frameRAM1: return "RAM1";
    case GR_frameRAM4: return "RAM4";
    case GR_frameRAM8: return "RAM8";
    case GR_frameRAM16: return "RAM16";
    case GR_frameRAM24: return "RAM24";
    case GR_frameRAM32L: return "RAM32L";
    case GR_frameRAM32H: return "RAM32H";
    case GR_frameRAM3x8: return "RAM3x8";
  @}
  return "UNKNOWN";
@}

void Message(int disp, char *txt, gvmode *gp) @{
  char msg[200];
  sprintf(msg, "%s: %d x %d x %dbpp",
		FrameDriverName(gp->fm), gp->w, gp->h, gp->bpp);
#if (GRX_VERSION_API-0) >= 0x0229
  if ( GrGetLibrarySystem() == GRX_VERSION_GENERIC_X11)
    fprintf(stderr,"%s\t%s\n", msg, txt);
#elif (GRX_VERSION == GRX_VERSION_GENERIC_X11)
  fprintf(stderr,"%s\t%s\n", msg, txt);
#endif
  if (disp) @{
    GrTextOption to;
    GrContext save;
    GrSaveContext(&save);
    GrSetContext(NULL);
    to.txo_font = &GrFont_PC6x8;
    to.txo_fgcolor.v = GrWhite();
    to.txo_bgcolor.v = GrBlack();
    to.txo_chrtype = GR_BYTE_TEXT;
    to.txo_direct  = GR_TEXT_RIGHT;
    to.txo_xalign  = GR_ALIGN_LEFT;
    to.txo_yalign  = GR_ALIGN_TOP;
    GrDrawString(msg,strlen(msg),0,0,&to);
    GrDrawString(txt,strlen(txt),0,10,&to);
    GrSetContext(&save);
  @}
@}

void printresultheader(FILE *f) @{
  fprintf(f, "Driver               readp drawp line   hline vline  block  v2v    v2r    r2v\n");
@}

void printresultline(FILE *f, gvmode * gp) @{
  fprintf(f, "%-9s %4dx%4d ", FrameDriverName(gp->fm), gp->w, gp->h);
  fprintf(f, "%6.2f", gp->readpix.rate  / (1024.0 * 1024.0));
  fprintf(f, "%6.2f", gp->drawpix.rate  / (1024.0 * 1024.0));
  fprintf(f, "%6.2f", gp->drawlin.rate  / (1024.0 * 1024.0));
  fprintf(f, "%7.2f", gp->drawhlin.rate / (1024.0 * 1024.0));
  fprintf(f, "%6.2f", gp->drawvlin.rate / (1024.0 * 1024.0));
  fprintf(f, "%7.2f", gp->drawblk.rate  / (1024.0 * 1024.0));
  fprintf(f, "%7.2f", gp->blitv2v.rate  / (1024.0 * 1024.0));
  fprintf(f, "%7.2f", gp->blitv2r.rate  / (1024.0 * 1024.0));
  fprintf(f, "%7.2f", gp->blitr2v.rate  / (1024.0 * 1024.0));
  fprintf(f, "\n");
@}

void readpixeltest(gvmode *gp, XY_PAIRS *pairs,int loops) @{
  int i, j;
  long t1,t2;
  double seconds;
  int *x = pairs->x;
  int *y = pairs->y;

  if (!MEASURED(gp)) @{
    gp->readpix.rate  = 0.0;
    gp->readpix.count = DBL(PAIRS) * DBL(loops);
  @}

  t1 = clock();
  for (i=loops; i > 0; --i) @{
    for (j=PAIRS-1; j >= 0; j--)
       GrPixelNC(x[j],y[j]);
  @}
  t2 = clock();
  seconds = DBL(t2 - t1) / DBL(CLK_TCK);
  if (seconds > 0)
    gp->readpix.rate = gp->readpix.count / seconds;
@}

void drawpixeltest(gvmode *gp, XY_PAIRS *pairs) @{
  int i, j;
  GrColor c1 = GrWhite();
  GrColor c2 = GrWhite() | GrXOR;
  GrColor c3 = GrWhite() | GrOR;
  GrColor c4 = GrBlack() | GrAND;
  long t1,t2;
  double seconds;
  int *x = pairs->x;
  int *y = pairs->y;

  if (!MEASURED(gp)) @{
    gp->drawpix.rate  = 0.0;
    gp->drawpix.count = DBL(PAIRS) * DBL(DRAWPIX_loops) * 4.0;
  @}

  t1 = clock();
  for (i=0; i < DRAWPIX_loops; ++i) @{
    for (j=PAIRS-1; j >= 0; j--) GrPlotNC(x[j],y[j],c1);
    for (j=PAIRS-1; j >= 0; j--) GrPlotNC(x[j],y[j],c2);
    for (j=PAIRS-1; j >= 0; j--) GrPlotNC(x[j],y[j],c3);
    for (j=PAIRS-1; j >= 0; j--) GrPlotNC(x[j],y[j],c4);
  @}
  t2 = clock();
  seconds = DBL(t2 - t1) / DBL(CLK_TCK);
  if (seconds > 0)
    gp->drawpix.rate = gp->drawpix.count / seconds;
@}

void drawlinetest(gvmode *gp, XY_PAIRS *pairs) @{
  int i, j;
  int *x = pairs->x;
  int *y = pairs->y;
  GrColor c1 = GrWhite();
  GrColor c2 = GrWhite() | GrXOR;
  GrColor c3 = GrWhite() | GrOR;
  GrColor c4 = GrBlack() | GrAND;
  long t1,t2;
  double seconds;

  if (!MEASURED(gp)) @{
    gp->drawlin.rate  = 0.0;
    gp->drawlin.count = 0.0;
    for (j=0; j < PAIRS; j+=2)
      gp->drawlin.count += sqrt(SQR(x[j],x[j+1])+SQR(y[j],y[j+1]));
    gp->drawlin.count *= 4.0 * DRAWLIN_loops;
  @}

  t1 = clock();
  for (i=0; i < DRAWLIN_loops; ++i) @{
    for (j=PAIRS-2; j >= 0; j-=2)
	GrLineNC(x[j],y[j],x[j+1],y[j+1],c1);
    for (j=PAIRS-2; j >= 0; j-=2)
	GrLineNC(x[j],y[j],x[j+1],y[j+1],c2);
    for (j=PAIRS-2; j >= 0; j-=2)
	GrLineNC(x[j],y[j],x[j+1],y[j+1],c3);
    for (j=PAIRS-2; j >= 0; j-=2)
	GrLineNC(x[j],y[j],x[j+1],y[j+1],c4);
  @}
  t2 = clock();
  seconds = DBL(t2 - t1) / DBL(CLK_TCK);
  if (seconds > 0)
    gp->drawlin.rate = gp->drawlin.count / seconds;
@}

void drawhlinetest(gvmode *gp, XY_PAIRS *pairs) @{
  int  i, j;
  int *x = pairs->x;
  int *y = pairs->y;
  GrColor c1 = GrWhite();
  GrColor c2 = GrWhite() | GrXOR;
  GrColor c3 = GrWhite() | GrOR;
  GrColor c4 = GrBlack() | GrAND;
  long t1,t2;
  double seconds;

  if (!MEASURED(gp)) @{
    gp->drawhlin.rate = 0.0;
    gp->drawhlin.count = 0.0;
    for (j=0; j < PAIRS; j+=2)
      gp->drawhlin.count += ABS(x[j],x[j+1]);
    gp->drawhlin.count *= 4.0 * DRAWHLIN_loops;
  @}

  t1 = clock();
  for (i=0; i < DRAWHLIN_loops; ++i) @{
    for (j=PAIRS-2; j >= 0; j-=2)
      GrHLineNC(x[j],x[j+1],y[j],c1);
    for (j=PAIRS-2; j >= 0; j-=2)
      GrHLineNC(x[j],x[j+1],y[j],c2);
    for (j=PAIRS-2; j >= 0; j-=2)
      GrHLineNC(x[j],x[j+1],y[j],c3);
    for (j=PAIRS-2; j >= 0; j-=2)
      GrHLineNC(x[j],x[j+1],y[j],c4);
  @}
  t2 = clock();
  seconds = DBL(t2 - t1) / DBL(CLK_TCK);
  if (seconds > 0)
    gp->drawhlin.rate = gp->drawhlin.count / seconds;
@}

void drawvlinetest(gvmode *gp, XY_PAIRS *pairs) @{
  int i, j;
  int *x = pairs->x;
  int *y = pairs->y;
  GrColor c1 = GrWhite();
  GrColor c2 = GrWhite() | GrXOR;
  GrColor c3 = GrWhite() | GrOR;
  GrColor c4 = GrBlack() | GrAND;
  long t1,t2;
  double seconds;

  if (!MEASURED(gp)) @{
    gp->drawvlin.rate = 0.0;
    gp->drawvlin.count = 0.0;
    for (j=0; j < PAIRS; j+=2)
      gp->drawvlin.count += ABS(y[j],y[j+1]);
    gp->drawvlin.count *= 4.0 * DRAWVLIN_loops;
  @}

  t1 = clock();
  for (i=0; i < DRAWVLIN_loops; ++i) @{
    for (j=PAIRS-2; j >= 0; j-=2)
       GrVLineNC(x[j],y[j],y[j+1],c1);
    for (j=PAIRS-2; j >= 0; j-=2)
       GrVLineNC(x[j],y[j],y[j+1],c2);
    for (j=PAIRS-2; j >= 0; j-=2)
       GrVLineNC(x[j],y[j],y[j+1],c3);
    for (j=PAIRS-2; j >= 0; j-=2)
       GrVLineNC(x[j],y[j],y[j+1],c4);
  @}
  t2 = clock();
  seconds = DBL(t2 - t1) / DBL(CLK_TCK);
  if (seconds > 0)
    gp->drawvlin.rate = gp->drawvlin.count / seconds;
@}

void drawblocktest(gvmode *gp, XY_PAIRS *pairs) @{
  int i, j;
  GrColor c1 = GrWhite();
  GrColor c2 = GrWhite() | GrXOR;
  GrColor c3 = GrWhite() | GrOR;
  GrColor c4 = GrBlack() | GrAND;
  long t1,t2;
  double seconds;

  if (xb == NULL || yb == NULL) return;

  for (j=0; j < PAIRS; j+=2) @{
    xb[j]   = min(pairs->x[j],pairs->x[j+1]);
    xb[j+1] = max(pairs->x[j],pairs->x[j+1]);
    yb[j]   = min(pairs->y[j],pairs->y[j+1]);
    yb[j+1] = max(pairs->y[j],pairs->y[j+1]);
  @}

  if (!MEASURED(gp)) @{
    gp->drawblk.rate = 0.0;
    gp->drawblk.count = 0.0;
    for (j=0; j < PAIRS; j+=2)
      gp->drawblk.count += ABS(xb[j],xb[j+1]) * ABS(yb[j],yb[j+1]);
    gp->drawblk.count *= 4.0 * DRAWBLK_loops;
  @}

  t1 = clock();
  for (i=0; i < DRAWBLK_loops; ++i) @{
    for (j=PAIRS-2; j >= 0; j-=2)
      GrFilledBoxNC(xb[j],yb[j],xb[j+1],yb[j+1],c1);
    for (j=PAIRS-2; j >= 0; j-=2)
      GrFilledBoxNC(xb[j],yb[j],xb[j+1],yb[j+1],c2);
    for (j=PAIRS-2; j >= 0; j-=2)
      GrFilledBoxNC(xb[j],yb[j],xb[j+1],yb[j+1],c3);
    for (j=PAIRS-2; j >= 0; j-=2)
      GrFilledBoxNC(xb[j],yb[j],xb[j+1],yb[j+1],c4);
  @}
  t2 = clock();
  seconds = DBL(t2 - t1) / DBL(CLK_TCK);
  if (seconds > 0)
    gp->drawblk.rate = gp->drawblk.count / seconds;
@}

void xor_draw_blocks(GrContext *c) @{
  GrContext save;
  int i;

  GrSaveContext(&save);
  GrSetContext(c);
  GrClearContext(GrBlack());
  for (i=28; i > 1; --i)
    GrFilledBox(GrMaxX()/i,GrMaxY()/i,
		(i-1)*GrMaxX()/i,(i-1)*GrMaxY()/i,GrWhite()|GrXOR);
  GrSetContext(&save);
@}

void blit_measure(gvmode *gp, perfm *p,
		  int *xb, int *yb,
		  GrContext *dst,GrContext *src) @{
  int i, j;
  long t1,t2;
  double seconds;
  GrContext save;

  GrSaveContext(&save);
  if (dst != src) @{
    GrSetContext(dst);
    GrClearContext(GrBlack());
  @}
  xor_draw_blocks(src);
  GrSetContext(&save);

  if (dst != NULL) @{
    char *s = src != NULL ? "ram" : "video";
    char *d = dst != NULL ? "ram" : "video";
    char txt[50];
    sprintf(txt, "blit test: %s -> %s", s, d);
    Message(1,txt, gp);
  @}

  t1 = clock();
  for (i=0; i < BLIT_loops; ++i) @{
    for (j=PAIRS-3; j >= 0; j-=3)
      GrBitBlt(dst,xb[j+2],yb[j+2],src,xb[j+1],yb[j+1],xb[j],yb[j],GrWRITE);
    for (j=PAIRS-3; j >= 0; j-=3)
      GrBitBlt(dst,xb[j+2],yb[j+2],src,xb[j+1],yb[j+1],xb[j],yb[j],GrXOR);
    for (j=PAIRS-3; j >= 0; j-=3)
      GrBitBlt(dst,xb[j+2],yb[j+2],src,xb[j+1],yb[j+1],xb[j],yb[j],GrOR);
    for (j=PAIRS-3; j >= 0; j-=3)
      GrBitBlt(dst,xb[j+2],yb[j+2],src,xb[j+1],yb[j+1],xb[j],yb[j],GrAND);
  @}
  t2 = clock();
  seconds = DBL(t2 - t1) / DBL(CLK_TCK);
  if (seconds > 0)
    p->rate = p->count / seconds;
@}

void blittest(gvmode *gp, XY_PAIRS *pairs, int ram) @{
  int j;

  if (xb == NULL || yb == NULL) return;

  for (j=0; j < PAIRS; j+=3) @{
    int wh;
    xb[j]   = max(pairs->x[j],pairs->x[j+1]);
    xb[j+1] = min(pairs->x[j],pairs->x[j+1]);
    xb[j+2] = pairs->x[j+2];
    wh      = xb[j]-xb[j+1];
    if (xb[j+2]+wh >= gp->w) xb[j+2] = gp->w - wh - 1;
    yb[j]   = max(pairs->y[j],pairs->y[j+1]);
    yb[j+1] = min(pairs->y[j],pairs->y[j+1]);
    yb[j+2] = pairs->y[j+2];
    wh      = yb[j]-yb[j+1];
    if (yb[j+2]+wh >= gp->h) yb[j+2] = gp->h - wh - 1;
  @}

  if (!MEASURED(gp)) @{
    double count = 0.0;
    for (j=0; j < PAIRS; j+=3)
      count += ABS(xb[j],xb[j+1]) * ABS(yb[j],yb[j+1]);
    gp->blitv2v.count =
    gp->blitr2v.count =
    gp->blitv2r.count = count * 4.0 * BLIT_loops;
    gp->blitv2v.rate  =
    gp->blitr2v.rate  =
    gp->blitv2r.rate  = 0.0;
  @}

#if BLIT_loops-0
  blit_measure(gp, &gp->blitv2v, xb, yb,
	       (GrContext *)(RAMMODE(gp) ? GrCurrentContext() : NULL),
	       (GrContext *)(RAMMODE(gp) ? GrCurrentContext() : NULL));
  if (!BLIT_FAIL(gp) && !ram) @{
    GrContext rc;
    GrContext *rcp = GrCreateContext(gp->w,gp->h,NULL,&rc);
    if (rcp) @{
      blit_measure(gp, &gp->blitv2r, xb, yb, rcp, NULL);
      blit_measure(gp, &gp->blitr2v, xb, yb, NULL, rcp);
      GrDestroyContext(rcp);
    @}
  @}
#endif
@}


void measure_one(gvmode *gp, int ram) @{
  XY_PAIRS *pairs;

  if (MEASURED(gp)) return;
  pairs = checkpairs(gp->w, gp->h);
  GrFilledBox( 0, 0, gp->w-1, gp->h-1, GrBlack());
  Message(RAMMODE(gp),"read pixel test", gp);
  @{ int rd_loops = READPIX_loops;
#if (GRX_VERSION_API-0) >= 0x0229
    if ( GrGetLibrarySystem() == GRX_VERSION_GENERIC_X11) @{
      if (!RAMMODE(gp)) rd_loops = READPIX_X11_loops;
    @}
#elif (GRX_VERSION == GRX_VERSION_GENERIC_X11)
    if (!RAMMODE(gp)) rd_loops = READPIX_X11_loops;
#endif
    readpixeltest(gp,pairs,rd_loops);
  @}
  GrFilledBox( 0, 0, gp->w-1, gp->h-1, GrBlack());
  Message(RAMMODE(gp),"draw pixel test", gp);
  drawpixeltest(gp,pairs);
  GrFilledBox( 0, 0, gp->w-1, gp->h-1, GrBlack());
  Message(RAMMODE(gp),"draw line test ", gp);
  drawlinetest(gp,pairs);
  GrFilledBox( 0, 0, gp->w-1, gp->h-1, GrBlack());
  Message(RAMMODE(gp),"draw hline test", gp);
  drawhlinetest(gp,pairs);
  GrFilledBox( 0, 0, gp->w-1, gp->h-1, GrBlack());
  Message(RAMMODE(gp),"draw vline test", gp);
  drawvlinetest(gp,pairs);
  GrFilledBox( 0, 0, gp->w-1, gp->h-1, GrBlack());
  Message(RAMMODE(gp),"draw block test", gp);
  drawblocktest(gp,pairs);
  GrFilledBox( 0, 0, gp->w-1, gp->h-1, GrBlack());
  blittest(gp, pairs, ram);
  GrFilledBox( 0, 0, gp->w-1, gp->h-1, GrBlack());
  SET_MEASURED(gp);
  measured_any = 1;
@}

#if MEASURE_RAM_MODES
int identical_measured(gvmode *tm) @{
  int i;
  for (i=0; i < nmodes; ++i) @{
    if (tm      != &rammodes[i]    &&
	tm->fm  == rammodes[i].fm  &&
	tm->w   == rammodes[i].w   &&
	tm->h   == rammodes[i].h   &&
	tm->bpp == rammodes[i].bpp &&
	MEASURED(&rammodes[i])        ) return (1);
  @}
  return 0;
@}
#endif

void speedcheck(gvmode *gp, int wait) @{
  char m[40];
  static int first = 1;
  gvmode *rp = NULL;

  if (first) @{
    printf(
      "speedtest may take some time to process.\n"
      "Now press <CR> to continue..."
    );
    fflush(stdout);
    gets(m);
  @}

  GrSetMode(
      GR_width_height_color_graphics,
      gp->w, gp->h, 1UL<<gp->bpp
  );

  if (first) @{
    /* xor_draw_blocks(NULL);
       getch(); */
    first = 0;
  @}

  if ( GrScreenFrameMode() != gp->fm) @{
    GrFrameMode act = GrScreenFrameMode();
    GrSetMode(GR_default_text);
    printf("Setup failed : %s != %s\n",
    FrameDriverName(act),
    FrameDriverName(gp->fm));
    getch();
    return;
  @}

  if (!MEASURED(gp))
    measure_one(gp, 0);

#if MEASURE_RAM_MODES
  rp = &rammodes[(unsigned)(gp-grmodes)];
  rp->fm = GrCoreFrameMode();
  if (!MEASURED(rp) && !identical_measured(rp)) @{
    GrContext rc;
    if (GrCreateFrameContext(rp->fm,gp->w,gp->h,NULL,&rc)) @{
      GrSetContext(&rc);
      measure_one(rp, 1);
      GrDestroyContext(&rc);
      GrSetContext(NULL);
    @}
  @}
#endif

  GrSetMode(GR_default_text);
  if (wait) @{
    printf("Results: \n");
    printresultheader(stdout);
    printresultline(stdout, gp);
    if (rp)
      printresultline(stdout, rp);
    kbhit();    /* this is here to flush in the X version 8-) */
    gets(m);
  @}
@}

int collectmodes(const GrVideoDriver *drv)
@{
	gvmode *gp = grmodes;
	GrFrameMode fm;
	const GrVideoMode *mp;
	for(fm =GR_firstGraphicsFrameMode;
	      fm <= GR_lastGraphicsFrameMode; fm++) @{
	    for(mp = GrFirstVideoMode(fm); mp; mp = GrNextVideoMode(mp)) @{
		gp->fm    = fm;
		gp->w     = mp->width;
		gp->h     = mp->height;
		gp->bpp   = mp->bpp;
		gp->flags = 0;
		gp++;
		if (gp-grmodes >= MAX_MODES) return MAX_MODES;
	    @}
	@}
	return(int)(gp-grmodes);
@}

int vmcmp(const void *m1,const void *m2)
@{
	gvmode *md1 = (gvmode *)m1;
	gvmode *md2 = (gvmode *)m2;
	if(md1->bpp != md2->bpp) return(md1->bpp - md2->bpp);
	if(md1->w   != md2->w  ) return(md1->w   - md2->w  );
	if(md1->h   != md2->h  ) return(md1->h   - md2->h  );
	return(0);
@}

#define LINES   20
#define COLUMNS 80

void ModeText(int i, int shrt,char *mdtxt) @{
	char *flg;

	if (MEASURED(&grmodes[i])) flg = " #"; else
	if (TAGGED(&grmodes[i]))   flg = " *"; else
				   flg = ") ";
	switch (shrt) @{
	  case 2 : sprintf(mdtxt,"%2d%s %dx%d ", i+1, flg, grmodes[i].w, grmodes[i].h);
		   break;
	  case 1 : sprintf(mdtxt,"%2d%s %4dx%-4d ", i+1, flg, grmodes[i].w, grmodes[i].h);
		   break;
	  default: sprintf(mdtxt,"  %2d%s  %4dx%-4d ", i+1, flg, grmodes[i].w, grmodes[i].h);
		   break;
	@}
	mdtxt += strlen(mdtxt);

	if (grmodes[i].bpp > 20)
	  sprintf(mdtxt, "%ldM", 1L << (grmodes[i].bpp-20));
	else  if (grmodes[i].bpp > 10)
	  sprintf(mdtxt, "%ldk", 1L << (grmodes[i].bpp-10));
	else
	  sprintf(mdtxt, "%ld", 1L << grmodes[i].bpp);
	switch (shrt) @{
	  case 2 : break;
	  case 1 : strcat(mdtxt, " col"); break;
	  default: strcat(mdtxt, " colors"); break;
	@}
@}

int ColsCheck(int cols, int ml, int sep) @{
  int len;

  len = ml * cols + (cols-1) * sep + 1;
  return len <= COLUMNS;
@}

void PrintModes(void) @{
	char mdtxt[100];
	unsigned int maxlen;
	int i, n, shrt, c, cols;

	cols = (nmodes+LINES-1) / LINES;
	do @{
	  for (shrt = 0; shrt <= 2; ++shrt) @{
	    maxlen = 0;
	    for (i = 0; i < nmodes; ++i) @{
	      ModeText(i,shrt,mdtxt);
	      if (strlen(mdtxt) > maxlen) maxlen = strlen(mdtxt);
	    @}
	    n = 2;
	    if (cols>1 || shrt<2) @{
	      if (!ColsCheck(cols, maxlen, n)) continue;
	      while (ColsCheck(cols, maxlen, n+1) && n < 4) ++n;
	    @}
	    c = 0;
	    for (i = 0; i < nmodes; ++i) @{
	      if (++c == cols) c = 0;
	      ModeText(i,shrt,mdtxt);
	      printf("%*s%s", (c ? -maxlen-n : -maxlen), mdtxt, (c || (i+1==nmodes) ? "" : "\n") );
	    @}
	    return;
	  @}
	  --cols;
	@} while (1);
@}

int main(void)
@{
	int  i;

	grmodes = malloc(MAX_MODES*sizeof(gvmode));
	assert(grmodes!=NULL);
#if MEASURE_RAM_MODES
	rammodes = malloc(MAX_MODES*sizeof(gvmode));
	assert(rammodes!=NULL);
#endif

	GrSetDriver(NULL);
	if(GrCurrentVideoDriver() == NULL) @{
	    printf("No graphics driver found\n");
	    exit(1);
	@}

	nmodes = collectmodes(GrCurrentVideoDriver());
	if(nmodes == 0) @{
	    printf("No graphics modes found\n");
	    exit(1);
	@}
	qsort(grmodes,nmodes,sizeof(grmodes[0]),vmcmp);
#if MEASURE_RAM_MODES
	for (i=0; i < nmodes; ++i) @{
	  rammodes[i].fm    = GR_frameUndef;      /* filled in later */
	  rammodes[i].w     = grmodes[i].w;
	  rammodes[i].h     = grmodes[i].h;
	  rammodes[i].bpp   = grmodes[i].bpp;
	  rammodes[i].flags = FLG_rammode;
	@}
#endif

	for( ; ; ) @{
	    char mb[40], *m = mb;
	    int tflag = 0;
	    GrSetMode(GR_default_text);
	    printf(
		"Graphics driver: \"%s\"\t"
		"graphics defaults: %dx%d %ld colors\n",
		GrCurrentVideoDriver()->name,
		GrDriverInfo->defgw,
		GrDriverInfo->defgh,
		(long)GrDriverInfo->defgc
	    );
	    PrintModes();
	    printf("\nEnter #, 't#' toggels tag, 'm' measure tagged and 'q' to quit> ");
	    fflush(stdout);
	    if(!gets(m)) continue;
	    switch (*m) @{
	      case 't':
	      case 'T': tflag = 1;
			++m;
			break;
	      case 'A':
	      case 'a': for (i=0; i < nmodes; ++i)
			  SET_TAGGED(&grmodes[i]);
			break;
	      case 'M':
	      case 'm': for (i=0; i < nmodes; ++i)
			  if (TAGGED(&grmodes[i])) @{
			    speedcheck(&grmodes[i], 0);
			    TOGGLE_TAGGED(&grmodes[i]);
			  @}
			break;
	      case 'Q':
	      case 'q': goto done;
	    @}
	    if ((sscanf(m,"%d",&i) != 1) || (i < 1) || (i > nmodes))
		continue;
	    i--;
	    if (tflag) TOGGLE_TAGGED(&grmodes[i]);
		  else speedcheck(&grmodes[i], 1);
	@}
done:
	if (measured_any) @{
	    int i;
	    FILE *log = fopen("speedtst.log", "a");

	    if (!log) exit(1);

	    fprintf( log, "\nGraphics driver: \"%s\"\n\n",
					       GrCurrentVideoDriver()->name);
	    printf("Results: \n");
	    printresultheader(log);

	    for (i=0; i < nmodes; ++i)
	      if (MEASURED(&grmodes[i]))
		printresultline(log, &grmodes[i]);
#if MEASURE_RAM_MODES
	    for (i=0; i < nmodes; ++i)
	      if (MEASURED(&rammodes[i]))
		printresultline(log, &rammodes[i]);
#endif
	    fclose(log);
	@}
	return(0);
@}


@end example

@c -----------------------------------------------------------------------------
@node textpatt.c, winclip.c, speedtst.c, Test examples
@unnumberedsec textpatt.c

@example

/*textpatt.c TEST
 **
 ** textpatt.c
 **
 ** This is a test/demo file of the GRX graphics library.
 ** You can use GRX test/demo files as you want.
 **
 ** The GRX graphics library is free software; you can redistribute it
 ** and/or modify it under some conditions; see the "copying.grx" file
 ** for details.
 **
 ** This library is distributed in the hope that it will be useful,
 ** but WITHOUT ANY WARRANTY; without even the implied warranty of
 ** MERCHANTABILITY or FITNESS FOR A PARTICULAR PURPOSE.
 **
*/

#include <stdio.h>
#include <string.h>
#include "grx20.h"         
#include "grxkeys.h"

#define FONT "../fonts/tms38b.fnt"

int main(void)
@{
  char bits[] = @{0, 76, 50, 0, 0, 76, 60, 0@};
  GrPattern *p1, *p2;
  GrFont *font;
  GrTextOption opt;
  int fail_p1, fail_p2, fail_font;

  GrSetMode(GR_width_height_color_graphics, 320, 200, (GrColor)256);
  p1 = GrBuildPixmapFromBits(bits, 8, 8, 11,  3);
  p2 = GrBuildPixmapFromBits(bits, 8, 8,  3, 11);
  font = GrLoadFont(FONT);
  if (font && p1 && p2) @{
    memset(&opt, 0, sizeof(GrTextOption));
    opt.txo_font   = font;
    opt.txo_xalign = 0;
    opt.txo_yalign = 0;
    opt.txo_direct = GR_TEXT_RIGHT;
    opt.txo_fgcolor.v = GrNOCOLOR;
    opt.txo_bgcolor.v = GrNOCOLOR;
    GrPatternFilledBox(0, 0, GrMaxX(), GrMaxY(), p1);
    GrKeyRead();
    GrPatternDrawString(" Hello world !", 15, 40, 10, &opt, p1);
    GrPatternDrawString(" Hello world !", 15, 44, 50, &opt, p2);
    GrPatternDrawStringExt(" Hello world !!", 16, 40, 100, &opt, p1);
    GrPatternDrawStringExt(" Hello world !!", 16, 44, 140, &opt, p2);
    GrKeyRead();
    opt.txo_bgcolor.v = GrBlack();
    GrPatternDrawString(" Hello world !", 15, 40, 10, &opt, p1);
    GrPatternDrawString(" Hello world !", 15, 44, 50, &opt, p2);
    GrPatternDrawStringExt(" Hello world !!", 16, 40, 100, &opt, p1);
    GrPatternDrawStringExt(" Hello world !!", 16, 44, 140, &opt, p2);
    GrKeyRead();
  @}
  fail_p1 = p1 == NULL;
  if (p1)   GrDestroyPattern(p1);
  fail_p2 = p2 == NULL;
  if (p2)   GrDestroyPattern(p2);
  fail_font = font == NULL;
  if (font) GrUnloadFont(font);
  GrSetMode(GR_default_text);
  if (fail_p1) fprintf(stderr, "Couldn't create first pattern\n");
  if (fail_p2) fprintf(stderr, "Couldn't create second pattern\n");
  if (fail_font) fprintf(stderr, "Couldn't load font %s\n", FONT);

  return 0;
@}

@end example

@c -----------------------------------------------------------------------------
@node winclip.c, wintest.c, textpatt.c, Test examples
@unnumberedsec winclip.c

@example
/**
 ** winclip.c ---- clip a drawing to various windows (contexts)
 ** Copyright (c) 1995 Csaba Biegl, 820 Stirrup Dr, Nashville, TN 37221
 ** [e-mail: csaba@@vuse.vanderbilt.edu]
 **
 ** This is a test/demo file of the GRX graphics library.
 ** You can use GRX test/demo files as you want.
 **
 ** The GRX graphics library is free software; you can redistribute it
 ** and/or modify it under some conditions; see the "copying.grx" file
 ** for details.
 **
 ** This library is distributed in the hope that it will be useful,
 ** but WITHOUT ANY WARRANTY; without even the implied warranty of
 ** MERCHANTABILITY or FITNESS FOR A PARTICULAR PURPOSE.
 **
 **/

#include "test.h"   /* @pxref{test.h}  */

TESTFUNC(winclip)
@{
	int  x = GrSizeX();
	int  y = GrSizeY();
	int  ww = (x / 2) - 10;
	int  wh = (y / 2) - 10;
	GrColor c;
	GrContext *w1 = GrCreateSubContext(5,5,ww+4,wh+4,NULL,NULL);
	GrContext *w2 = GrCreateSubContext(15+ww,5,ww+ww+14,wh+4,NULL,NULL);
	GrContext *w3 = GrCreateSubContext(5,15+wh,ww+4,wh+wh+14,NULL,NULL);
	GrContext *w4 = GrCreateSubContext(15+ww,15+wh,ww+ww+14,wh+wh+14,NULL,NULL);

	GrSetContext(w1);
	c = GrAllocColor(200,100,100);
	drawing(0,0,ww,wh,c,GrBlack());
	GrBox(0,0,ww-1,wh-1,c);

	GrSetContext(w2);
	c = GrAllocColor(100,200,200);
	drawing(-ww/4,ww/3,ww,wh,c,GrBlack());
	GrBox(0,0,ww-1,wh-1,c);

	GrSetContext(w3);
	c = GrAllocColor(200,200,0);
	drawing(ww/2,-wh/2,ww,wh,c,GrBlack());
	GrBox(0,0,ww-1,wh-1,c);

	GrSetContext(w4);
	c = GrAllocColor(0,100,200);
	drawing(-ww/2,-wh/2,ww*2,wh*2,c,GrBlack());
	GrBox(0,0,ww-1,wh-1,c);

	GrKeyRead();
@}

@end example

@c -----------------------------------------------------------------------------
@node wintest.c, drawing.h, winclip.c, Test examples
@unnumberedsec wintest.c

@example
/**
 ** wintest.c ---- test window (context) mapping
 ** Copyright (c) 1995 Csaba Biegl, 820 Stirrup Dr, Nashville, TN 37221
 ** [e-mail: csaba@@vuse.vanderbilt.edu]
 **
 ** This is a test/demo file of the GRX graphics library.
 ** You can use GRX test/demo files as you want.
 **
 ** The GRX graphics library is free software; you can redistribute it
 ** and/or modify it under some conditions; see the "copying.grx" file
 ** for details.
 **
 ** This library is distributed in the hope that it will be useful,
 ** but WITHOUT ANY WARRANTY; without even the implied warranty of
 ** MERCHANTABILITY or FITNESS FOR A PARTICULAR PURPOSE.
 **/

#include "test.h"  /* @pxref{test.h}  */

TESTFUNC(wintest)
@{
	int  x = GrSizeX();
	int  y = GrSizeY();
	int  ww = (x / 2) - 10;
	int  wh = (y / 2) - 10;
	GrColor c;
	GrContext *w1 = GrCreateSubContext(5,5,ww+4,wh+4,NULL,NULL);
	GrContext *w2 = GrCreateSubContext(15+ww,5,ww+ww+14,wh+4,NULL,NULL);
	GrContext *w3 = GrCreateSubContext(5,15+wh,ww+4,wh+wh+14,NULL,NULL);
	GrContext *w4 = GrCreateSubContext(15+ww,15+wh,ww+ww+14,wh+wh+14,NULL,NULL);

	GrSetContext(w1);
	c = GrAllocColor(200,100,100);
	drawing(0,0,ww,wh,c,GrBlack());
	GrBox(0,0,ww-1,wh-1,c);

	GrSetContext(w2);
	c = GrAllocColor(100,200,200);
	drawing(0,0,ww,wh,c,GrBlack());
	GrBox(0,0,ww-1,wh-1,c);

	GrSetContext(w3);
	c = GrAllocColor(200,200,0);
	drawing(0,0,ww,wh,c,GrBlack());
	GrBox(0,0,ww-1,wh-1,c);

	GrSetContext(w4);
	c = GrAllocColor(0,100,200);
	drawing(0,0,ww,wh,c,GrBlack());
	GrBox(0,0,ww-1,wh-1,c);

	GrKeyRead();
@}


@end example

@c -----------------------------------------------------------------------------
@node drawing.h, rand.h, wintest.c, Test examples
@unnumberedsec drawing.h

@example
/**
 ** DRAWING.H ---- a stupid little drawing used all over in test programs
 **
 ** Copyright (c) 1995 Csaba Biegl, 820 Stirrup Dr, Nashville, TN 37221
 ** [e-mail: csaba@@vuse.vanderbilt.edu] See "doc/copying.cb" for details.
 **/

#include "rand.h"   /* @pxref{rand.h}  */

void drawing(int xpos,int ypos,int xsize,int ysize,long fg,long bg)
@{
#   define XP(x)   (int)((((long)(x) * (long)xsize) / 100L) + xpos)
#   define YP(y)   (int)((((long)(y) * (long)ysize) / 100L) + ypos)
	int ii;
	if(bg != GrNOCOLOR) @{
		GrFilledBox(xpos,ypos,xpos+xsize-1,ypos+ysize-1,bg);
	@}
	GrLine(XP(10),YP(10),XP(40),YP(40),fg);
	GrLine(XP(40),YP(10),XP(10),YP(40),fg);
	GrLine(XP(35),YP(10),XP(65),YP(40),fg);
	GrLine(XP(35),YP(40),XP(65),YP(10),fg);
	GrLine(XP(70),YP(10),XP(90),YP(40),fg);
	GrLine(XP(70),YP(40),XP(90),YP(10),fg);
	for(ii = 0; ii < 5; ii++) @{
		GrBox(XP(70+2*ii),YP(10+3*ii),XP(90-2*ii),YP(40-3*ii),fg);
	@}
	GrFilledBox(XP(10),YP(50),XP(60),YP(90),fg);
	GrBox(XP(70),YP(50),XP(90),YP(90),fg);
	for(ii = 0; ii < 100; ii++) @{
		GrPlot(XP((RND() % 20U) + 70),YP((RND() % 40U) + 50),fg);
	@}
@}

#undef XP
#undef YP

@end example

@c -----------------------------------------------------------------------------
@node rand.h, test.h, drawing.h, Test examples
@unnumberedsec rand.h

@example
/**
 ** rand.h ---- a very simple random number generator
 **             (from "Numerical recipies")
 **
 ** This is a test/demo file of the GRX graphics library.
 ** You can use GRX test/demo files as you want.
 **
 ** The GRX graphics library is free software; you can redistribute it
 ** and/or modify it under some conditions; see the "copying.grx" file
 ** for details.
 **
 ** This library is distributed in the hope that it will be useful,
 ** but WITHOUT ANY WARRANTY; without even the implied warranty of
 ** MERCHANTABILITY or FITNESS FOR A PARTICULAR PURPOSE.
 **
 **/

#ifndef __RAND_H_INCLUDED
#define __RAND_H_INCLUDED

#define _IA   16807
#define _IM   2147483647L
#define _IQ   127773L
#define _IR   2836
#define _MASK 123459876UL

static long _idum = 0;

unsigned long ran0(void) @{
  long k;
  _idum ^= _MASK;
  k = _idum / _IQ;
  _idum = _IA * (_idum - k * _IQ) - _IR * k;
  if (_idum < 0) _idum += _IM;
  return (unsigned long) _idum;
@}

#define sran0(x) do _idum = (x); while(0)

#define RND()    ran0()
#define SRND(x)  sran0(x)
#define RND_MAX  (_MASK)

#endif

@end example

@c -----------------------------------------------------------------------------
@node test.h, arctest.dat, rand.h, Test examples
@unnumberedsec test.h

@example
/**
 ** test.h ---- common declarations for test programs
 **
 ** Copyright (c) 1995 Csaba Biegl, 820 Stirrup Dr, Nashville, TN 37221
 ** [e-mail: csaba@@vuse.vanderbilt.edu]
 **
 ** This is a test/demo file of the GRX graphics library.
 ** You can use GRX test/demo files as you want.
 **
 ** The GRX graphics library is free software; you can redistribute it
 ** and/or modify it under some conditions; see the "copying.grx" file
 ** for details.
 **
 ** This library is distributed in the hope that it will be useful,
 ** but WITHOUT ANY WARRANTY; without even the implied warranty of
 ** MERCHANTABILITY or FITNESS FOR A PARTICULAR PURPOSE.
 **/

#ifndef __TEST_H_INCLUDED__
#define __TEST_H_INCLUDED__

#include <stdio.h>
#include <stdlib.h>
#include <string.h>

#ifdef  __TURBOC__
#include <conio.h>
#endif

#ifdef  __WATCOMC__
#include <conio.h>
#endif

#ifdef  __GNUC__
extern  int getch(void);
extern  int kbhit(void);
#endif

#include "grx20.h"
#include "grxkeys.h"
#include "drawing.h"    /* @pxref{drawing.h}  */

extern void (*testfunc)(void);
char   exit_message[2000] = @{ "" @};
int    Argc;
char **Argv;

#define TESTFUNC(name)      \
void name(void);        \
void (*testfunc)(void) = name;  \
void name(void)

int main(int argc,char **argv)
@{
	int  x = 0;
	int  y = 0;
	long c = 0;

	Argc = argc - 1;
	Argv = argv + 1;
	if((Argc >= 2) &&
	   (sscanf(Argv[0],"%d",&x) == 1) && (x >= 320) &&
	   (sscanf(Argv[1],"%d",&y) == 1) && (y >= 200)) @{
		Argc -= 2;
		Argv += 2;
		if (Argc > 0) @{
		   char *endp;
		   c = strtol(Argv[0], &endp, 0);
		   switch (*endp) @{
		     case 'k':
		     case 'K': c <<= 10; break;
		     case 'm':
		     case 'M': c <<= 20; break;
		   @}
		   Argc--;
		   Argv++;
		@}
	@}
	if(c >= 2)
		GrSetMode(GR_width_height_color_graphics,x,y,c);
	else if((x >= 320) && (y >= 200))
		GrSetMode(GR_width_height_graphics,x,y);
	else GrSetMode(GR_default_graphics);
	(*testfunc)();
	GrSetMode(GR_default_text);
	if(strlen(exit_message) > 0) @{
		puts(exit_message);
		getch();
	@}
	return(0);
@}

#endif /* _TEST_H_ */
@end example

@node arctest.dat,polytest.dat ,test.h, Test examples
@unnumberedsec arctest.dat
@example
arc xc=300 yc=200 xa=50  ya=50  start=10   end=40
arc xc=300 yc=200 xa=250 ya=150 start=10   end=200
arc xc=300 yc=200 xa=250 ya=150 start=10   end=2000
arc xc=300 yc=200 xa=250 ya=150 start=1000 end=200
arc xc=300 yc=200 xa=25  ya=15  start=3500 end=800
arc xc=300 yc=200 xa=25  ya=15  start=10   end=100
arc xc=300 yc=200 xa=25  ya=15  start=3500 end=10
arc xc=300 yc=200 xa=25  ya=15  start=0    end=900
@end example

@node polytest.dat, Includes,arctest.dat, Test examples
@unnumberedsec polytest.dat
@example
beginc
300 200
400 400
200 400
end

beginc
300 200
400 400
150 470
200 400
end

beginc
300 200
400 400
150 470
120 330
end

beginc
300 200
400 400
050 470
020 330
end

beginc
300 -100
400 400
050 870
020 330
end

beginc
300 20
400 100
050 100
end

beginc
400 500
050 500
200 560
end

beginc
400 500
250 495
050 500
200 560
end

beginc
100 500
400 500
300 550
end

beginc
150 150
300 150
250 250
300 400
120 444
end

begin
250 150
200 450
350 250
150 250
400 450
end

beginc
150 150
400 200
400 202
end

begin
-10 0
100 -10
200 200
-10 200
end
@end example

@node Includes, grx20.h,polytest.dat , A User Manual For GRX2
@unnumberedsec Includes
@menu
* grx20.h::
* grxkeys.h::
@end menu

@c -----------------------------------------------------------------------------
@node grx20.h, grxkeys.h, Includes, Includes
@unnumberedsec grx20.h

@example
/**
 ** grx20.h ---- GRX 2.x API functions and data structure declarations
 **
 ** Copyright (c) 1995 Csaba Biegl, 820 Stirrup Dr, Nashville, TN 37221
 ** [e-mail: csaba@@vuse.vanderbilt.edu]
 **
 ** This file is part of the GRX graphics library.
 **
 ** The GRX graphics library is free software; you can redistribute it
 ** and/or modify it under some conditions; see the "copying.grx" file
 ** for details.
 **
 ** This library is distributed in the hope that it will be useful,
 ** but WITHOUT ANY WARRANTY; without even the implied warranty of
 ** MERCHANTABILITY or FITNESS FOR A PARTICULAR PURPOSE.
 **
 **/

#ifndef __GRX20_H_INCLUDED__
#define __GRX20_H_INCLUDED__

/* ================================================================== */
/*       COMPILER -- CPU -- SYSTEM SPECIFIC VERSION STUFF             */
/* ================================================================== */

/* Version of GRX API
**
** usage:
**    #include <grx20.h>
**    #ifndef GRX_VERSION_API
**    #ifdef  GRX_VERSION
**    #define GRX_VERSION_API 0x0200
**    #else
**    #define GRX_VERSION_API 0x0103
**    #endif
**    #endif
*/
#define GRX_VERSION_API 0x0234

/* these are the supported configurations: */
#define GRX_VERSION_TCC_8086_DOS        1       /* also works with BCC */
#define GRX_VERSION_GCC_386_GO32        2       /* DJGPP */
#define GRX_VERSION_GCC_386_LINUX       3       /* the real stuff */
#define GRX_VERSION_GENERIC_X11         4       /* generic X11 version */
#define GRX_VERSION_WATCOM_DOS4GW       5       /* GS - Watcom C++ 11.0 32 Bit */
/*#define GRX_VERSION_WATCOM_REAL_MODE  6*/     /* GS - Watcom C++ 11.0 16 Bit - TODO! */
#define GRX_VERSION_GCC_386_WIN32       7       /* WIN32 using Mingw32 */

#ifdef  __TURBOC__
#define GRX_VERSION     GRX_VERSION_TCC_8086_DOS
#endif

#ifdef  __GNUC__
#ifdef  __GO32__
#define GRX_VERSION     GRX_VERSION_GCC_386_GO32
#endif
#if defined(__linux__) && defined(__i386__)
#define GRX_VERSION     GRX_VERSION_GCC_386_LINUX
#endif
#ifdef  __WIN32__
#define GRX_VERSION     GRX_VERSION_GCC_386_WIN32
#endif
#endif

#ifdef  __WATCOMC__     /* GS - Watcom C++ 11.0 */
#ifdef  __DOS__
#ifdef  __386__
#define GRX_VERSION     GRX_VERSION_WATCOM_DOS4GW
#else
/* #define GRX_VERSION GRX_VERSION_WATCOM_REAL_MODE  - I haven't tested GRX in 16 bit*/
#endif /* __386__ */
#endif /* __DOS__ */
#endif /* __WATCOMC__ */

#ifndef GRX_VERSION
#if defined(unix) || defined(__unix) || defined(__unix__) || defined(_AIX)
#define GRX_VERSION     GRX_VERSION_GENERIC_X11
#endif
#endif

#ifndef GRX_VERSION
#error  GRX is not supported on your COMPILER/CPU/OPERATING SYSTEM!
#endif

#if (GRX_VERSION==GRX_VERSION_WATCOM_DOS4GW)
#define near
#define far
#define huge
#endif

#if !defined(__TURBOC__) && (GRX_VERSION!=GRX_VERSION_WATCOM_REAL_MODE)
#ifndef near            /* get rid of these stupid keywords */
#define near
#endif
#ifndef far
#define far
#endif
#ifndef huge
#define huge
#endif
#endif

#ifdef __cplusplus
extern "C" @{
#endif

/* a couple of forward declarations ... */
typedef struct _GR_frameDriver  GrFrameDriver;
typedef struct _GR_videoDriver  GrVideoDriver;
typedef struct _GR_videoMode    GrVideoMode;
typedef struct _GR_videoModeExt GrVideoModeExt;
typedef struct _GR_frame        GrFrame;
typedef struct _GR_context      GrContext;

/* ================================================================== */
/*                        SYSTEM TYPE DEF's                           */
/* ================================================================== */

/* need unsigned 32 bit integer for color stuff */
#ifdef __TURBOC__
/* TCC && BCC are 16 bit compilers */
typedef unsigned long int GrColor;
#else
/* all other platforms (GCC on i386 and ALPHA) have 32 bit ints */
typedef unsigned int GrColor;
#endif

/* ================================================================== */
/*                           MODE SETTING                             */
/* ================================================================== */

/*
 * available video modes (for 'GrSetMode')
 */
typedef enum _GR_graphicsModes @{
	GR_unknown_mode = (-1),             /* initial state */
	/* ============= modes which clear the video memory ============= */
	GR_80_25_text = 0,                  /* Extra parameters for GrSetMode: */
	GR_default_text,
	GR_width_height_text,               /* int w,int h */
	GR_biggest_text,
	GR_320_200_graphics,
	GR_default_graphics,
	GR_width_height_graphics,           /* int w,int h */
	GR_biggest_noninterlaced_graphics,
	GR_biggest_graphics,
	GR_width_height_color_graphics,     /* int w,int h,GrColor nc */
	GR_width_height_color_text,         /* int w,int h,GrColor nc */
	GR_custom_graphics,                 /* int w,int h,GrColor nc,int vx,int vy */
	/* ==== equivalent modes which do not clear the video memory ==== */
	GR_NC_80_25_text,
	GR_NC_default_text,
	GR_NC_width_height_text,            /* int w,int h */
	GR_NC_biggest_text,
	GR_NC_320_200_graphics,
	GR_NC_default_graphics,
	GR_NC_width_height_graphics,        /* int w,int h */
	GR_NC_biggest_noninterlaced_graphics,
	GR_NC_biggest_graphics,
	GR_NC_width_height_color_graphics,  /* int w,int h,GrColor nc */
	GR_NC_width_height_color_text,      /* int w,int h,GrColor nc */
	GR_NC_custom_graphics,              /* int w,int h,GrColor nc,int vx,int vy */
	/* ==== plane instead of color based modes ==== */
	/* colors = 1 << bpp  >>> resort enum for GRX3 <<< */
	GR_width_height_bpp_graphics,       /* int w,int h,int bpp */
	GR_width_height_bpp_text,           /* int w,int h,int bpp */
	GR_custom_bpp_graphics,             /* int w,int h,int bpp,int vx,int vy */
	GR_NC_width_height_bpp_graphics,    /* int w,int h,int bpp */
	GR_NC_width_height_bpp_text,        /* int w,int h,int bpp */
	GR_NC_custom_bpp_graphics           /* int w,int h,int bpp,int vx,int vy */
@} GrGraphicsMode;

/*
 * Available frame modes (video memory layouts)
 */
typedef enum _GR_frameModes @{
	/* ====== video frame buffer modes ====== */
	GR_frameUndef,                      /* undefined */
	GR_frameText,                       /* text modes */
	GR_frameHERC1,                      /* Hercules mono */
	GR_frameEGAVGA1,                    /* EGA VGA mono */
	GR_frameEGA4,                       /* EGA 16 color */
	GR_frameSVGA4,                      /* (Super) VGA 16 color */
	GR_frameSVGA8,                      /* (Super) VGA 256 color */
	GR_frameVGA8X,                      /* VGA 256 color mode X */
	GR_frameSVGA16,                     /* Super VGA 32768/65536 color */
	GR_frameSVGA24,                     /* Super VGA 16M color */
	GR_frameSVGA32L,                    /* Super VGA 16M color padded #1 */
	GR_frameSVGA32H,                    /* Super VGA 16M color padded #2 */
	/* ==== modes provided by the X11 driver ===== */
	GR_frameXWIN1   = GR_frameEGAVGA1,
	GR_frameXWIN4   = GR_frameSVGA4,
	GR_frameXWIN8   = GR_frameSVGA8,
	GR_frameXWIN16  = GR_frameSVGA16,
	GR_frameXWIN24  = GR_frameSVGA24,
	GR_frameXWIN32L = GR_frameSVGA32L,
	GR_frameXWIN32H = GR_frameSVGA32H,
	/* ==== linear frame buffer modes  ====== */
	GR_frameSVGA8_LFB,                  /* (Super) VGA 256 color */
	GR_frameSVGA16_LFB,                 /* Super VGA 32768/65536 color */
	GR_frameSVGA24_LFB,                 /* Super VGA 16M color */
	GR_frameSVGA32L_LFB,                /* Super VGA 16M color padded #1 */
	GR_frameSVGA32H_LFB,                /* Super VGA 16M color padded #2 */
	/* ====== system RAM frame buffer modes ====== */
	GR_frameRAM1,                       /* mono */
	GR_frameRAM4,                       /* 16 color planar */
	GR_frameRAM8,                       /* 256 color */
	GR_frameRAM16,                      /* 32768/65536 color */
	GR_frameRAM24,                      /* 16M color */
	GR_frameRAM32L,                     /* 16M color padded #1 */
	GR_frameRAM32H,                     /* 16M color padded #2 */
	GR_frameRAM3x8,                     /* 16M color planar (image mode) */
	/* ====== markers for scanning modes ====== */
	GR_firstTextFrameMode     = GR_frameText,
	GR_lastTextFrameMode      = GR_frameText,
	GR_firstGraphicsFrameMode = GR_frameHERC1,
	GR_lastGraphicsFrameMode  = GR_frameSVGA32H_LFB,
	GR_firstRAMframeMode      = GR_frameRAM1,
	GR_lastRAMframeMode       = GR_frameRAM3x8
@} GrFrameMode;

/*
 * supported video adapter types
 */
typedef enum _GR_videoAdapters @{
	GR_UNKNOWN = (-1),                  /* not known (before driver set) */
	GR_VGA,                             /* VGA adapter */
	GR_EGA,                             /* EGA adapter */
	GR_HERC,                            /* Hercules mono adapter */
	GR_8514A,                           /* 8514A or compatible */
	GR_S3,                              /* S3 graphics accelerator */
	GR_XWIN,                            /* X11 driver */
	GR_MEM                              /* memory only driver */
@} GrVideoAdapter;

/*
 * The video driver descriptor structure
 */
struct _GR_videoDriver @{
	char   *name;                       /* driver name */
	enum   _GR_videoAdapters adapter;   /* adapter type */
	struct _GR_videoDriver  *inherit;   /* inherit video modes from this */
	struct _GR_videoMode    *modes;     /* table of supported modes */
	int     nmodes;                     /* number of modes */
	int   (*detect)(void);
	int   (*init)(char *options);
	void  (*reset)(void);
	GrVideoMode * (*selectmode)(GrVideoDriver *drv,int w,int h,int bpp,
					int txt,unsigned int *ep);
	unsigned  drvflags;
@};
/* bits in the drvflags field: */
#define GR_DRIVERF_USER_RESOLUTION 1
  /* set if driver supports user setable arbitrary resolution */


/*
 * Video driver mode descriptor structure
 */
struct _GR_videoMode @{
	char    present;                    /* is it really available? */
	char    bpp;                        /* log2 of # of colors */
	short   width,height;               /* video mode geometry */
	short   mode;                       /* BIOS mode number (if any) */
	int     lineoffset;                 /* scan line length */
	int     privdata;                   /* driver can use it for anything */
	struct _GR_videoModeExt *extinfo;   /* extra info (maybe shared) */
@};

/*
 * Video driver mode descriptor extension structure. This is a separate
 * structure accessed via a pointer from the main mode descriptor. The
 * reason for this is that frequently several modes can share the same
 * extended info.
 */
struct _GR_videoModeExt @{
	enum   _GR_frameModes   mode;       /* frame driver for this video mode */
	struct _GR_frameDriver *drv;        /* optional frame driver override */
	char    far *frame;                 /* frame buffer address */
	char    cprec[3];                   /* color component precisions */
	char    cpos[3];                    /* color component bit positions */
	int     flags;                      /* mode flag bits; see "grdriver.h" */
	int   (*setup)(GrVideoMode *md,int noclear);
	int   (*setvsize)(GrVideoMode *md,int w,int h,GrVideoMode *result);
	int   (*scroll)(GrVideoMode *md,int x,int y,int result[2]);
	void  (*setbank)(int bk);
	void  (*setrwbanks)(int rb,int wb);
	void  (*loadcolor)(int c,int r,int g,int b);
	int     LFB_Selector;
@};

/*
 * The frame driver descriptor structure.
 */
struct _GR_frameDriver @{
    enum    _GR_frameModes mode;         /* supported frame access mode */
    enum    _GR_frameModes rmode;        /* matching RAM frame (if video) */
    int      is_video;                   /* video RAM frame driver ? */
    int      row_align;                  /* scan line size alignment */
    int      num_planes;                 /* number of planes */
    int      bits_per_pixel;             /* bits per pixel */
    long     max_plane_size;             /* maximum plane size in bytes */
    int      (*init)(GrVideoMode *md);
    GrColor  (*readpixel)(GrFrame *c,int x,int y);
    void     (*drawpixel)(int x,int y,GrColor c);
    void     (*drawline)(int x,int y,int dx,int dy,GrColor c);
    void     (*drawhline)(int x,int y,int w,GrColor c);
    void     (*drawvline)(int x,int y,int h,GrColor c);
    void     (*drawblock)(int x,int y,int w,int h,GrColor c);
    void     (*drawbitmap)(int x,int y,int w,int h,char far *bmp,int pitch,int start,GrColor fg,GrColor bg);
    void     (*drawpattern)(int x,int y,int w,char patt,GrColor fg,GrColor bg);
    void     (*bitblt)(GrFrame *dst,int dx,int dy,GrFrame *src,int x,int y,int w,int h,GrColor op);
    void     (*bltv2r)(GrFrame *dst,int dx,int dy,GrFrame *src,int x,int y,int w,int h,GrColor op);
    void     (*bltr2v)(GrFrame *dst,int dx,int dy,GrFrame *src,int x,int y,int w,int h,GrColor op);
    /* new functions in v2.3 */
    GrColor far *(*getindexedscanline)(GrFrame *c,int x, int y, int w, int *indx);
      /* will return an array of pixel values pv[] read from frame   */
      /*    if indx == NULL: pv[i=0..w-1] = readpixel(x+i,y)         */
      /*    else             pv[i=0..w-1] = readpixel(x+indx[i],y)   */
    void     (*putscanline)(int x, int y, int w,const GrColor far *scl, GrColor op);
      /** will draw scl[i=0..w-1] to frame:                          */
      /*    if (scl[i] != skipcolor) drawpixel(x+i,y,(scl[i] | op))  */
@};

/*
 * driver and mode info structure
 */
extern const struct _GR_driverInfo @{
	struct _GR_videoDriver  *vdriver;   /* the current video driver */
	struct _GR_videoMode    *curmode;   /* current video mode pointer */
	struct _GR_videoMode     actmode;   /* copy of above, resized if virtual */
	struct _GR_frameDriver   fdriver;   /* frame driver for the current context */
	struct _GR_frameDriver   sdriver;   /* frame driver for the screen */
	struct _GR_frameDriver   tdriver;   /* a dummy driver for text modes */
	enum   _GR_graphicsModes mcode;     /* code for the current mode */
	int     deftw,defth;                /* default text mode size */
	int     defgw,defgh;                /* default graphics mode size */
	GrColor deftc,defgc;                /* default text and graphics colors */
	int     vposx,vposy;                /* current virtual viewport position */
	int     errsfatal;                  /* if set, exit upon errors */
	int     moderestore;                /* restore startup video mode if set */
	int     splitbanks;                 /* indicates separate R/W banks */
	int     curbank;                    /* currently mapped bank */
	void  (*mdsethook)(void);           /* callback for mode set */
	void  (*setbank)(int bk);           /* banking routine */
	void  (*setrwbanks)(int rb,int wb); /* split banking routine */
@} * const GrDriverInfo;

/*
 * setup stuff
 */
int  GrSetDriver(char *drvspec);
int  GrSetMode(GrGraphicsMode which,...);
int  GrSetViewport(int xpos,int ypos);
void GrSetModeHook(void (*hookfunc)(void));
void GrSetModeRestore(int restoreFlag);
void GrSetErrorHandling(int exitIfError);
void GrSetEGAVGAmonoDrawnPlane(int plane);
void GrSetEGAVGAmonoShownPlane(int plane);

unsigned GrGetLibraryVersion(void);
unsigned GrGetLibrarySystem(void);

/*
 * inquiry stuff ---- many of these are actually macros (see below)
 */
GrGraphicsMode GrCurrentMode(void);
GrVideoAdapter GrAdapterType(void);
GrFrameMode    GrCurrentFrameMode(void);
GrFrameMode    GrScreenFrameMode(void);
GrFrameMode    GrCoreFrameMode(void);

const GrVideoDriver *GrCurrentVideoDriver(void);
const GrVideoMode   *GrCurrentVideoMode(void);
const GrVideoMode   *GrVirtualVideoMode(void);
const GrFrameDriver *GrCurrentFrameDriver(void);
const GrFrameDriver *GrScreenFrameDriver(void);
const GrVideoMode   *GrFirstVideoMode(GrFrameMode fmode);
const GrVideoMode   *GrNextVideoMode(const GrVideoMode *prev);

int  GrScreenX(void);
int  GrScreenY(void);
int  GrVirtualX(void);
int  GrVirtualY(void);
int  GrViewportX(void);
int  GrViewportY(void);

int  GrScreenIsVirtual(void);

/*
 * RAM context geometry and memory allocation inquiry stuff
 */
int  GrFrameNumPlanes(GrFrameMode md);
int  GrFrameLineOffset(GrFrameMode md,int width);
long GrFramePlaneSize(GrFrameMode md,int w,int h);
long GrFrameContextSize(GrFrameMode md,int w,int h);

int  GrNumPlanes(void);
int  GrLineOffset(int width);
long GrPlaneSize(int w,int h);
long GrContextSize(int w,int h);

/*
 * inline implementation for some of the above
 */
#ifndef GRX_SKIP_INLINES
#define GrAdapterType()         (GrDriverInfo->vdriver ? GrDriverInfo->vdriver->adapter : GR_UNKNOWN)
#define GrCurrentMode()         (GrDriverInfo->mcode)
#define GrCurrentFrameMode()    (GrDriverInfo->fdriver.mode)
#define GrScreenFrameMode()     (GrDriverInfo->sdriver.mode)
#define GrCoreFrameMode()       (GrDriverInfo->sdriver.rmode)

#define GrCurrentVideoDriver()  ((const GrVideoDriver *)( GrDriverInfo->vdriver))
#define GrCurrentVideoMode()    ((const GrVideoMode   *)( GrDriverInfo->curmode))
#define GrVirtualVideoMode()    ((const GrVideoMode   *)(&GrDriverInfo->actmode))
#define GrCurrentFrameDriver()  ((const GrFrameDriver *)(&GrDriverInfo->fdriver))
#define GrScreenFrameDriver()   ((const GrFrameDriver *)(&GrDriverInfo->sdriver))

#define GrIsFixedMode()      (!(  GrCurrentVideoDriver()->drvflags \
				   & GR_DRIVERF_USER_RESOLUTION))

#define GrScreenX()             (GrCurrentVideoMode()->width)
#define GrScreenY()             (GrCurrentVideoMode()->height)
#define GrVirtualX()            (GrVirtualVideoMode()->width)
#define GrVirtualY()            (GrVirtualVideoMode()->height)
#define GrViewportX()           (GrDriverInfo->vposx)
#define GrViewportY()           (GrDriverInfo->vposy)

#define GrScreenIsVirtual()     ((GrScreenX() + GrScreenY()) < (GrVirtualX() + GrVirtualY()))

#define GrNumPlanes()           GrFrameNumPlanes(GrCoreFrameMode())
#define GrLineOffset(w)         GrFrameLineOffset(GrCoreFrameMode(),w)
#define GrPlaneSize(w,h)        GrFramePlaneSize(GrCoreFrameMode(),w,h)
#define GrContextSize(w,h)      GrFrameContextSize(GrCoreFrameMode(),w,h)
#endif  /* GRX_SKIP_INLINES */


/* ================================================================== */
/*              FRAME BUFFER, CONTEXT AND CLIPPING STUFF              */
/* ================================================================== */

struct _GR_frame @{
	char    far *gf_baseaddr[4];        /* base address of frame memory */
	short   gf_selector;                /* frame memory segment selector */
	char    gf_onscreen;                /* is it in video memory ? */
	char    gf_memflags;                /* memory allocation flags */
	int     gf_lineoffset;              /* offset to next scan line in bytes */
	struct _GR_frameDriver *gf_driver;  /* frame access functions */
@};

struct _GR_context @{
	struct _GR_frame    gc_frame;       /* frame buffer info */
	struct _GR_context *gc_root;        /* context which owns frame */
	int    gc_xmax;                     /* max X coord (width  - 1) */
	int    gc_ymax;                     /* max Y coord (height - 1) */
	int    gc_xoffset;                  /* X offset from root's base */
	int    gc_yoffset;                  /* Y offset from root's base */
	int    gc_xcliplo;                  /* low X clipping limit */
	int    gc_ycliplo;                  /* low Y clipping limit */
	int    gc_xcliphi;                  /* high X clipping limit */
	int    gc_ycliphi;                  /* high Y clipping limit */
	int    gc_usrxbase;                 /* user window min X coordinate */
	int    gc_usrybase;                 /* user window min Y coordinate */
	int    gc_usrwidth;                 /* user window width  */
	int    gc_usrheight;                /* user window height */
#   define gc_baseaddr                  gc_frame.gf_baseaddr
#   define gc_selector                  gc_frame.gf_selector
#   define gc_onscreen                  gc_frame.gf_onscreen
#   define gc_memflags                  gc_frame.gf_memflags
#   define gc_lineoffset                gc_frame.gf_lineoffset
#   define gc_driver                    gc_frame.gf_driver
@};

extern const struct _GR_contextInfo @{
	struct _GR_context current;         /* the current context */
	struct _GR_context screen;          /* the screen context */
@} * const GrContextInfo;

GrContext *GrCreateContext(int w,int h,char far *memory[4],GrContext *where);
GrContext *GrCreateFrameContext(GrFrameMode md,int w,int h,char far *memory[4],GrContext *where);
GrContext *GrCreateSubContext(int x1,int y1,int x2,int y2,const GrContext *parent,GrContext *where);
GrContext *GrSaveContext(GrContext *where);

const GrContext *GrCurrentContext(void);
const GrContext *GrScreenContext(void);

void  GrDestroyContext(GrContext *context);
void  GrResizeSubContext(GrContext *context,int x1,int y1,int x2,int y2);
void  GrSetContext(const GrContext *context);

void  GrSetClipBox(int x1,int y1,int x2,int y2);
void  GrSetClipBoxC(GrContext *c,int x1,int y1,int x2,int y2);
void  GrGetClipBox(int *x1p,int *y1p,int *x2p,int *y2p);
void  GrGetClipBoxC(const GrContext *c,int *x1p,int *y1p,int *x2p,int *y2p);
void  GrResetClipBox(void);
void  GrResetClipBoxC(GrContext *c);

int   GrMaxX(void);
int   GrMaxY(void);
int   GrSizeX(void);
int   GrSizeY(void);
int   GrLowX(void);
int   GrLowY(void);
int   GrHighX(void);
int   GrHighY(void);

#ifndef GRX_SKIP_INLINES
#define GrCreateContext(w,h,m,c) (GrCreateFrameContext(GrCoreFrameMode(),w,h,m,c))
#define GrCurrentContext()       ((const GrContext *)(&GrContextInfo->current))
#define GrScreenContext()        ((const GrContext *)(&GrContextInfo->screen))
#define GrMaxX()                 (GrCurrentContext()->gc_xmax)
#define GrMaxY()                 (GrCurrentContext()->gc_ymax)
#define GrSizeX()                (GrMaxX() + 1)
#define GrSizeY()                (GrMaxY() + 1)
#define GrLowX()                 (GrCurrentContext()->gc_xcliplo)
#define GrLowY()                 (GrCurrentContext()->gc_ycliplo)
#define GrHighX()                (GrCurrentContext()->gc_xcliphi)
#define GrHighY()                (GrCurrentContext()->gc_ycliphi)
#define GrGetClipBoxC(C,x1p,y1p,x2p,y2p) do @{           \
	*(x1p) = (C)->gc_xcliplo;                           \
	*(y1p) = (C)->gc_ycliplo;                           \
	*(x2p) = (C)->gc_xcliphi;                           \
	*(y2p) = (C)->gc_ycliphi;                           \
@} while(0)
#define GrGetClipBox(x1p,y1p,x2p,y2p) do @{              \
	*(x1p) = GrLowX();                                  \
	*(y1p) = GrLowY();                                  \
	*(x2p) = GrHighX();                                 \
	*(y2p) = GrHighY();                                 \
@} while(0)
#endif  /* GRX_SKIP_INLINES */

/* ================================================================== */
/*                            COLOR STUFF                             */
/* ================================================================== */

/*
 * Flags to 'OR' to colors for various operations
 */
#define GrWRITE         0UL             /* write color */
#define GrXOR           0x01000000UL    /* to "XOR" any color to the screen */
#define GrOR            0x02000000UL    /* to "OR" to the screen */
#define GrAND           0x03000000UL    /* to "AND" to the screen */
#define GrIMAGE         0x04000000UL    /* BLIT: write, except given color */
#define GrCVALUEMASK    0x00ffffffUL    /* color value mask */
#define GrCMODEMASK     0xff000000UL    /* color operation mask */
#define GrNOCOLOR       (GrXOR | 0)     /* GrNOCOLOR is used for "no" color */

GrColor GrColorValue(GrColor c);
GrColor GrColorMode(GrColor c);
GrColor GrWriteModeColor(GrColor c);
GrColor GrXorModeColor(GrColor c);
GrColor GrOrModeColor(GrColor c);
GrColor GrAndModeColor(GrColor c);
GrColor GrImageModeColor(GrColor c);

/*
 * color system info structure (all [3] arrays are [r,g,b])
 */
extern const struct _GR_colorInfo @{
	GrColor       ncolors;              /* number of colors */
	GrColor       nfree;                /* number of unallocated colors */
	GrColor       black;                /* the black color */
	GrColor       white;                /* the white color */
	unsigned int  RGBmode;              /* set when RGB mode */
	unsigned int  prec[3];              /* color field precisions */
	unsigned int  pos[3];               /* color field positions */
	unsigned int  mask[3];              /* masks for significant bits */
	unsigned int  round[3];             /* add these for rounding */
	unsigned int  shift[3];             /* shifts for (un)packing color */
	unsigned int  norm;                 /* normalization for (un)packing */
	struct @{                            /* color table for non-RGB modes */
		unsigned char r,g,b;            /* loaded components */
		unsigned int  defined:1;        /* r,g,b values are valid if set */
		unsigned int  writable:1;       /* can be changed by 'GrSetColor' */
		unsigned long int nused;        /* usage count */
	@} ctable[256];
@} * const GrColorInfo;

void    GrResetColors(void);
void    GrSetRGBcolorMode(void);
void    GrRefreshColors(void);

GrColor GrNumColors(void);
GrColor GrNumFreeColors(void);

GrColor GrBlack(void);
GrColor GrWhite(void);

GrColor GrBuildRGBcolorT(int r,int g,int b);
GrColor GrBuildRGBcolorR(int r,int g,int b);
int     GrRGBcolorRed(GrColor c);
int     GrRGBcolorGreen(GrColor c);
int     GrRGBcolorBlue(GrColor c);

GrColor GrAllocColor(int r,int g,int b);   /* shared, read-only */
GrColor GrAllocColorID(int r,int g,int b); /* potentially inlined version */
GrColor GrAllocCell(void);                 /* unshared, read-write */

GrColor *GrAllocEgaColors(void);           /* shared, read-only standard EGA colors */

void    GrSetColor(GrColor c,int r,int g,int b);
void    GrFreeColor(GrColor c);
void    GrFreeCell(GrColor c);

void    GrQueryColor(GrColor c,int *r,int *g,int *b);
void    GrQueryColorID(GrColor c,int *r,int *g,int *b);

int     GrColorSaveBufferSize(void);
void    GrSaveColors(void *buffer);
void    GrRestoreColors(void *buffer);

#ifndef GRX_SKIP_INLINES
#define GrColorValue(c)         ((GrColor)(c) & GrCVALUEMASK)
#define GrColorMode(c)          ((GrColor)(c) & GrCMODEMASK)
#define GrWriteModeColor(c)     (GrColorValue(c) | GrWRITE)
#define GrXorModeColor(c)       (GrColorValue(c) | GrXOR)
#define GrOrModeColor(c)        (GrColorValue(c) | GrOR)
#define GrAndModeColor(c)       (GrColorValue(c) | GrAND)
#define GrImageModeColor(c)     (GrColorValue(c) | GrIMAGE)
#define GrNumColors()           (GrColorInfo->ncolors)
#define GrNumFreeColors()       (GrColorInfo->nfree)
#define GrBlack() (                                                            \
	(GrColorInfo->black == GrNOCOLOR) ?                                    \
	(GrBlack)() :                                                          \
	GrColorInfo->black                                                     \
)
#define GrWhite() (                                                            \
	(GrColorInfo->white == GrNOCOLOR) ?                                    \
	(GrWhite)() :                                                          \
	GrColorInfo->white                                                     \
)
#define GrBuildRGBcolorT(r,g,b) ((                                             \
	((GrColor)((int)(r) & GrColorInfo->mask[0]) << GrColorInfo->shift[0]) |\
	((GrColor)((int)(g) & GrColorInfo->mask[1]) << GrColorInfo->shift[1]) |\
	((GrColor)((int)(b) & GrColorInfo->mask[2]) << GrColorInfo->shift[2])  \
	) >> GrColorInfo->norm                                                 \
)
#define GrBuildRGBcolorR(r,g,b) GrBuildRGBcolorT(                              \
	(((unsigned int)(r)) > GrColorInfo->mask[0]) ? 255 : (unsigned int)(r) + GrColorInfo->round[0], \
	(((unsigned int)(g)) > GrColorInfo->mask[1]) ? 255 : (unsigned int)(g) + GrColorInfo->round[1], \
	(((unsigned int)(b)) > GrColorInfo->mask[2]) ? 255 : (unsigned int)(b) + GrColorInfo->round[2]  \
)
#define GrRGBcolorRed(c) (                                                     \
	(int)(((GrColor)(c) << GrColorInfo->norm) >> GrColorInfo->shift[0]) &  \
	(GrColorInfo->mask[0])                                                 \
)
#define GrRGBcolorGreen(c) (                                                   \
	(int)(((GrColor)(c) << GrColorInfo->norm) >> GrColorInfo->shift[1]) &  \
	(GrColorInfo->mask[1])                                                 \
)
#define GrRGBcolorBlue(c) (                                                    \
	(int)(((GrColor)(c) << GrColorInfo->norm) >> GrColorInfo->shift[2]) &  \
	(GrColorInfo->mask[2])                                                 \
)
#define GrAllocColorID(r,g,b) (GrColorInfo->RGBmode ?                          \
	GrBuildRGBcolorR(r,g,b) :                                              \
	GrAllocColor(r,g,b)                                                    \
)
#define GrQueryColorID(c,r,g,b) do @{                                           \
	if(GrColorInfo->RGBmode) @{                                             \
	*(r) = GrRGBcolorRed(c);                                               \
	*(g) = GrRGBcolorGreen(c);                                             \
	*(b) = GrRGBcolorBlue(c);                                              \
	break;                                                                 \
	@}                                                                      \
	if(((GrColor)(c) < GrColorInfo->ncolors) &&                            \
	   (GrColorInfo->ctable[(GrColor)(c)].defined)) @{                      \
	*(r) = GrColorInfo->ctable[(GrColor)(c)].r;                            \
	*(g) = GrColorInfo->ctable[(GrColor)(c)].g;                            \
	*(b) = GrColorInfo->ctable[(GrColor)(c)].b;                            \
	break;                                                                 \
	@}                                                                      \
	*(r) = *(g) = *(b) = 0;                                                \
@} while(0)
#endif  /* GRX_SKIP_INLINES */

/*
 * color table (for primitives using several colors):
 *   it is an array of colors with the first element being
 *   the number of colors in the table
 */
typedef GrColor *GrColorTableP;

#define GR_CTABLE_SIZE(table) (                                                \
	(table) ? (unsigned int)((table)[0]) : 0U                              \
)
#define GR_CTABLE_COLOR(table,index) (                                         \
	((unsigned)(index) < GR_CTABLE_SIZE(table)) ?                          \
	(table)[((unsigned)(index)) + 1] :                                     \
	GrNOCOLOR                                                              \
)
#define GR_CTABLE_ALLOCSIZE(ncolors)    ((ncolors) + 1)

/* ================================================================== */
/*                       GRAPHICS PRIMITIVES                          */
/* ================================================================== */

#ifdef  __TURBOC__
/* this is for GRX compiled with SMALL_STACK: */
#define GR_MAX_POLYGON_POINTS   (8192)
#define GR_MAX_ELLIPSE_POINTS   (1024 + 5)
/* old values without SMALL_STACK: */
/* #define GR_MAX_POLYGON_POINTS   (512) */
/* #define GR_MAX_ELLIPSE_POINTS   (256 + 5) */
#else
#define GR_MAX_POLYGON_POINTS   (1000000)
#define GR_MAX_ELLIPSE_POINTS   (1024 + 5)
#endif
#define GR_MAX_ANGLE_VALUE      (3600)
#define GR_ARC_STYLE_OPEN       0
#define GR_ARC_STYLE_CLOSE1     1
#define GR_ARC_STYLE_CLOSE2     2

typedef struct @{                        /* framed box colors */
	GrColor fbx_intcolor;
	GrColor fbx_topcolor;
	GrColor fbx_rightcolor;
	GrColor fbx_bottomcolor;
	GrColor fbx_leftcolor;
@} GrFBoxColors;

void GrClearScreen(GrColor bg);
void GrClearContext(GrColor bg);
void GrClearClipBox(GrColor bg);
void GrPlot(int x,int y,GrColor c);
void GrLine(int x1,int y1,int x2,int y2,GrColor c);
void GrHLine(int x1,int x2,int y,GrColor c);
void GrVLine(int x,int y1,int y2,GrColor c);
void GrBox(int x1,int y1,int x2,int y2,GrColor c);
void GrFilledBox(int x1,int y1,int x2,int y2,GrColor c);
void GrFramedBox(int x1,int y1,int x2,int y2,int wdt,GrFBoxColors *c);
int  GrGenerateEllipse(int xc,int yc,int xa,int ya,int points[GR_MAX_ELLIPSE_POINTS][2]);
int  GrGenerateEllipseArc(int xc,int yc,int xa,int ya,int start,int end,int points[GR_MAX_ELLIPSE_POINTS][2]);
void GrLastArcCoords(int *xs,int *ys,int *xe,int *ye,int *xc,int *yc);
void GrCircle(int xc,int yc,int r,GrColor c);
void GrEllipse(int xc,int yc,int xa,int ya,GrColor c);
void GrCircleArc(int xc,int yc,int r,int start,int end,int style,GrColor c);
void GrEllipseArc(int xc,int yc,int xa,int ya,int start,int end,int style,GrColor c);
void GrFilledCircle(int xc,int yc,int r,GrColor c);
void GrFilledEllipse(int xc,int yc,int xa,int ya,GrColor c);
void GrFilledCircleArc(int xc,int yc,int r,int start,int end,int style,GrColor c);
void GrFilledEllipseArc(int xc,int yc,int xa,int ya,int start,int end,int style,GrColor c);
void GrPolyLine(int numpts,int points[][2],GrColor c);
void GrPolygon(int numpts,int points[][2],GrColor c);
void GrFilledConvexPolygon(int numpts,int points[][2],GrColor c);
void GrFilledPolygon(int numpts,int points[][2],GrColor c);
void GrBitBlt(GrContext *dst,int x,int y,GrContext *src,int x1,int y1,int x2,int y2,GrColor op);
void GrFloodFill(int x, int y, GrColor border, GrColor c);

GrColor GrPixel(int x,int y);
GrColor GrPixelC(GrContext *c,int x,int y);

const GrColor *GrGetScanline(int x1,int x2,int yy);
const GrColor *GrGetScanlineC(GrContext *ctx,int x1,int x2,int yy);
/* Input   ctx: source context, if NULL the current context is used */
/*         x1 : first x coordinate read                             */
/*         x2 : last  x coordinate read                             */
/*         yy : y coordinate                                        */
/* Output  NULL     : error / no data (clipping occured)            */
/*         else                                                     */
/*           p[0..w]: pixel values read                             */
/*                      (w = |x2-y1|)                               */
/*           Output data is valid until next GRX call !             */

void GrPutScanline(int x1,int x2,int yy,const GrColor *c, GrColor op);
/* Input   x1 : first x coordinate to be set                        */
/*         x2 : last  x coordinate to be set                        */
/*         yy : y coordinate                                        */
/*         c  : c[0..(|x2-x1|] hold the pixel data                  */
/*         op : Operation (GrWRITE/GrXOR/GrOR/GrAND/GrIMAGE)        */
/*                                                                  */
/* Note    c[..] data must fit GrCVALUEMASK otherwise the results   */
/*         are implementation dependend.                            */
/*         => You can't supply operation code with the pixel data!  */


#ifndef GRX_SKIP_INLINES
#define GrGetScanline(x1,x2,yy) \
	GrGetScanlineC(NULL,(x1),(x2),(yy))
#endif

/* ================================================================== */
/*                 NON CLIPPING DRAWING PRIMITIVES                    */
/* ================================================================== */

void GrPlotNC(int x,int y,GrColor c);
void GrLineNC(int x1,int y1,int x2,int y2,GrColor c);
void GrHLineNC(int x1,int x2,int y,GrColor c);
void GrVLineNC(int x,int y1,int y2,GrColor c);
void GrBoxNC(int x1,int y1,int x2,int y2,GrColor c);
void GrFilledBoxNC(int x1,int y1,int x2,int y2,GrColor c);
void GrFramedBoxNC(int x1,int y1,int x2,int y2,int wdt,GrFBoxColors *c);
void GrBitBltNC(GrContext *dst,int x,int y,GrContext *src,int x1,int y1,int x2,int y2,GrColor op);

GrColor GrPixelNC(int x,int y);
GrColor GrPixelCNC(GrContext *c,int x,int y);

#ifndef GRX_SKIP_INLINES
#define GrPlotNC(x,y,c) (                                                      \
	(*GrCurrentFrameDriver()->drawpixel)(                                  \

	((x) + GrCurrentContext()->gc_xoffset),                                \
	((y) + GrCurrentContext()->gc_yoffset),                                \
	((c))                                                                  \
	)                                                                      \
)
#define GrPixelNC(x,y) (                                                       \
	(*GrCurrentFrameDriver()->readpixel)(                                  \
	(GrFrame *)(&GrCurrentContext()->gc_frame),                            \
	((x) + GrCurrentContext()->gc_xoffset),                                \
	((y) + GrCurrentContext()->gc_yoffset)                                 \
	)                                                                      \
)
#define GrPixelCNC(c,x,y) (                                                    \
	(*(c)->gc_driver->readpixel)(                                          \
	(&(c)->gc_frame),                                                      \
	((x) + (c)->gc_xoffset),                                               \
	((y) + (c)->gc_yoffset)                                                \
	)                                                                      \
)
#endif  /* GRX_SKIP_INLINES */

/* ================================================================== */
/*                   FONTS AND TEXT PRIMITIVES                        */
/* ================================================================== */

/*
 * text drawing directions
 */
#define GR_TEXT_RIGHT           0       /* normal */
#define GR_TEXT_DOWN            1       /* downward */
#define GR_TEXT_LEFT            2       /* upside down, right to left */
#define GR_TEXT_UP              3       /* upward */
#define GR_TEXT_DEFAULT         GR_TEXT_RIGHT
#define GR_TEXT_IS_VERTICAL(d)  ((d) & 1)

/*
 * text alignment options
 */
#define GR_ALIGN_LEFT           0       /* X only */
#define GR_ALIGN_TOP            0       /* Y only */
#define GR_ALIGN_CENTER         1       /* X, Y   */
#define GR_ALIGN_RIGHT          2       /* X only */
#define GR_ALIGN_BOTTOM         2       /* Y only */
#define GR_ALIGN_BASELINE       3       /* Y only */
#define GR_ALIGN_DEFAULT        GR_ALIGN_LEFT

/*
 * character types in text strings
 */
#define GR_BYTE_TEXT            0       /* one byte per character */
#define GR_WORD_TEXT            1       /* two bytes per character */
#define GR_ATTR_TEXT            2       /* chr w/ PC style attribute byte */

/*
 * macros to access components of various string/character types
 */
#define GR_TEXTCHR_SIZE(ty)     (((ty) == GR_BYTE_TEXT) ? sizeof(char) : sizeof(short))
#define GR_TEXTCHR_CODE(ch,ty)  (((ty) == GR_WORD_TEXT) ? (unsigned short)(ch) : (unsigned char)(ch))
#define GR_TEXTCHR_ATTR(ch,ty)  (((ty) == GR_ATTR_TEXT) ? ((unsigned short)(ch) >> 8) : 0)
#define GR_TEXTSTR_CODE(pt,ty)  (((ty) == GR_WORD_TEXT) ? ((unsigned short *)(pt))[0] : ((unsigned char *)(pt))[0])
#define GR_TEXTSTR_ATTR(pt,ty)  (((ty) == GR_ATTR_TEXT) ? ((unsigned char *)(pt))[1] : 0)

/*
 * text attribute macros for the GR_ATTR_TEXT type
 */
#define GR_BUILD_ATTR(fg,bg,ul) (((fg) & 15) | (((bg) & 7) << 4) | ((ul) ? 128 : 0))
#define GR_ATTR_FGCOLOR(attr)   (((attr)     ) &  15)
#define GR_ATTR_BGCOLOR(attr)   (((attr) >> 4) &   7)
#define GR_ATTR_UNDERLINE(attr) (((attr)     ) & 128)

/*
 * OR this to the foreground color value for underlined text when
 * using GR_BYTE_TEXT or GR_WORD_TEXT modes.
 */
#define GR_UNDERLINE_TEXT       (GrXOR << 4)

/*
 * Font conversion flags for 'GrLoadConvertedFont'. OR them as desired.
 */
#define GR_FONTCVT_NONE         0       /* no conversion */
#define GR_FONTCVT_SKIPCHARS    1       /* load only selected characters */
#define GR_FONTCVT_RESIZE       2       /* resize the font */
#define GR_FONTCVT_ITALICIZE    4       /* tilt font for "italic" look */
#define GR_FONTCVT_BOLDIFY      8       /* make a "bold"(er) font  */
#define GR_FONTCVT_FIXIFY       16      /* convert prop. font to fixed wdt */
#define GR_FONTCVT_PROPORTION   32      /* convert fixed font to prop. wdt */

/*
 * font structures
 */
typedef struct _GR_fontHeader @{         /* font descriptor */
	char    *name;                      /* font name */
	char    *family;                    /* font family name */
	char     proportional;              /* characters have varying width */
	char     scalable;                  /* derived from a scalable font */
	char     preloaded;                 /* set when linked into program */
	char     modified;                  /* "tweaked" font (resized, etc..) */
	unsigned int  width;                /* width (proportional=>average) */
	unsigned int  height;               /* font height */
	unsigned int  baseline;             /* baseline pixel pos (from top) */
	unsigned int  ulpos;                /* underline pixel pos (from top) */
	unsigned int  ulheight;             /* underline width */
	unsigned int  minchar;              /* lowest character code in font */
	unsigned int  numchars;             /* number of characters in font */
@} GrFontHeader;

typedef struct _GR_fontChrInfo @{        /* character descriptor */
	unsigned int  width;                /* width of this character */
	unsigned int  offset;               /* offset from start of bitmap */
@} GrFontChrInfo;

typedef struct _GR_font @{               /* the complete font */
	struct  _GR_fontHeader  h;          /* the font info structure */
	char     far *bitmap;               /* character bitmap array */
	char     far *auxmap;               /* map for rotated & underline chrs */
	unsigned int  minwidth;             /* width of narrowest character */
	unsigned int  maxwidth;             /* width of widest character */
	unsigned int  auxsize;              /* allocated size of auxiliary map */
	unsigned int  auxnext;              /* next free byte in auxiliary map */
	unsigned int  far      *auxoffs[7]; /* offsets to completed aux chars */
	struct  _GR_fontChrInfo chrinfo[1]; /* character info (not act. size) */
@} GrFont;

extern  GrFont          GrFont_PC6x8;
extern  GrFont          GrFont_PC8x8;
extern  GrFont          GrFont_PC8x14;
extern  GrFont          GrFont_PC8x16;
#define GrDefaultFont   GrFont_PC8x14

GrFont *GrLoadFont(char *name);
GrFont *GrLoadConvertedFont(char *name,int cvt,int w,int h,int minch,int maxch);
GrFont *GrBuildConvertedFont(GrFont *from,int cvt,int w,int h,int minch,int maxch);

void GrUnloadFont(GrFont *font);
void GrDumpFont(GrFont *f,char *CsymbolName,char *fileName);
void GrSetFontPath(char *path_list);

int  GrFontCharPresent(GrFont *font,int chr);
int  GrFontCharWidth(GrFont *font,int chr);
int  GrFontCharHeight(GrFont *font,int chr);
int  GrFontCharBmpRowSize(GrFont *font,int chr);
int  GrFontCharBitmapSize(GrFont *font,int chr);
int  GrFontStringWidth(GrFont *font,void *text,int len,int type);
int  GrFontStringHeight(GrFont *font,void *text,int len,int type);
int  GrProportionalTextWidth(GrFont *font,void *text,int len,int type);

char far *GrBuildAuxiliaryBitmap(GrFont *font,int chr,int dir,int ul);
char far *GrFontCharBitmap(GrFont *font,int chr);
char far *GrFontCharAuxBmp(GrFont *font,int chr,int dir,int ul);

typedef union _GR_textColor @{           /* text color union */
	GrColor       v;                    /* color value for "direct" text */
	GrColorTableP p;                    /* color table for attribute text */
@} GrTextColor;

typedef struct _GR_textOption @{         /* text drawing option structure */
	struct _GR_font     *txo_font;      /* font to be used */
	union  _GR_textColor txo_fgcolor;   /* foreground color */
	union  _GR_textColor txo_bgcolor;   /* background color */
	char    txo_chrtype;                /* character type (see above) */
	char    txo_direct;                 /* direction (see above) */
	char    txo_xalign;                 /* X alignment (see above) */
	char    txo_yalign;                 /* Y alignment (see above) */
@} GrTextOption;

typedef struct @{                        /* fixed font text window desc. */
	struct _GR_font     *txr_font;      /* font to be used */
	union  _GR_textColor txr_fgcolor;   /* foreground color */
	union  _GR_textColor txr_bgcolor;   /* background color */
	void   *txr_buffer;                 /* pointer to text buffer */
	void   *txr_backup;                 /* optional backup buffer */
	int     txr_width;                  /* width of area in chars */
	int     txr_height;                 /* height of area in chars */
	int     txr_lineoffset;             /* offset in buffer(s) between rows */
	int     txr_xpos;                   /* upper left corner X coordinate */
	int     txr_ypos;                   /* upper left corner Y coordinate */
	char    txr_chrtype;                /* character type (see above) */
@} GrTextRegion;

int  GrCharWidth(int chr,GrTextOption *opt);
int  GrCharHeight(int chr,GrTextOption *opt);
void GrCharSize(int chr,GrTextOption *opt,int *w,int *h);
int  GrStringWidth(void *text,int length,GrTextOption *opt);
int  GrStringHeight(void *text,int length,GrTextOption *opt);
void GrStringSize(void *text,int length,GrTextOption *opt,int *w,int *h);

void GrDrawChar(int chr,int x,int y,GrTextOption *opt);
void GrDrawString(void *text,int length,int x,int y,GrTextOption *opt);
void GrTextXY(int x,int y,char *text,GrColor fg,GrColor bg);

void GrDumpChar(int chr,int col,int row,GrTextRegion *r);
void GrDumpText(int col,int row,int wdt,int hgt,GrTextRegion *r);
void GrDumpTextRegion(GrTextRegion *r);

#ifndef GRX_SKIP_INLINES
#define GrFontCharPresent(f,ch) (                                              \
	((unsigned int)(ch) - (f)->h.minchar) < (f)->h.numchars                \
)
#define GrFontCharWidth(f,ch) (                                                \
	GrFontCharPresent(f,ch) ?                                              \
	(int)(f)->chrinfo[(unsigned int)(ch) - (f)->h.minchar].width :         \
	(f)->h.width                                                           \
)
#define GrFontCharHeight(f,ch) (                                               \
	(f)->h.height                                                          \
)
#define GrFontCharBmpRowSize(f,ch) (                                           \
	GrFontCharPresent(f,ch) ?                                              \
	(((f)->chrinfo[(unsigned int)(ch) - (f)->h.minchar].width + 7) >> 3) : \
	0                                                                      \
)
#define GrFontCharBitmapSize(f,ch) (                                           \
	GrFontCharBmpRowSize(f,ch) * (f)->h.height                             \
)
#define GrFontStringWidth(f,t,l,tp) (                                          \
	(f)->h.proportional ?                                                  \
	GrProportionalTextWidth((f),(t),(l),(tp)) :                            \
	(f)->h.width * (l)                                                     \
)
#define GrFontStringHeight(f,t,l,tp) (                                         \
	(f)->h.height                                                          \
)
#define GrFontCharBitmap(f,ch) (                                               \
	GrFontCharPresent(f,ch) ?                                              \
	&(f)->bitmap[(f)->chrinfo[(unsigned int)(ch) - (f)->h.minchar].offset]:\
	(char far *)0                                                          \
)
#define GrFontCharAuxBmp(f,ch,dir,ul) (                                        \
	(((dir) == GR_TEXT_DEFAULT) && !(ul)) ?                                \
	GrFontCharBitmap(f,ch) :                                               \
	GrBuildAuxiliaryBitmap((f),(ch),(dir),(ul))                            \
)
#define GrCharWidth(c,o) (                                                     \
	GR_TEXT_IS_VERTICAL((o)->txo_direct) ?                                 \
	GrFontCharHeight((o)->txo_font,GR_TEXTCHR_CODE(c,(o)->txo_chrtype)) :  \
	GrFontCharWidth( (o)->txo_font,GR_TEXTCHR_CODE(c,(o)->txo_chrtype))    \
)
#define GrCharHeight(c,o) (                                                    \
	GR_TEXT_IS_VERTICAL((o)->txo_direct) ?                                 \
	GrFontCharWidth( (o)->txo_font,GR_TEXTCHR_CODE(c,(o)->txo_chrtype)) :  \
	GrFontCharHeight((o)->txo_font,GR_TEXTCHR_CODE(c,(o)->txo_chrtype))    \
)
#define GrCharSize(c,o,wp,hp) do @{                                             \
	*(wp) = GrCharHeight(c,o);                                             \
	*(hp) = GrCharWidth( c,o);                                             \
@} while(0)
#define GrStringWidth(t,l,o) (                                                 \
	GR_TEXT_IS_VERTICAL((o)->txo_direct) ?                                 \
	GrFontStringHeight((o)->txo_font,(t),(l),(o)->txo_chrtype) :           \
	GrFontStringWidth( (o)->txo_font,(t),(l),(o)->txo_chrtype)             \
)
#define GrStringHeight(t,l,o) (                                                \
	GR_TEXT_IS_VERTICAL((o)->txo_direct) ?                                 \
	GrFontStringWidth( (o)->txo_font,(t),(l),(o)->txo_chrtype) :           \
	GrFontStringHeight((o)->txo_font,(t),(l),(o)->txo_chrtype)             \
)
#define GrStringSize(t,l,o,wp,hp) do @{                                         \
	*(wp) = GrStringWidth( t,l,o);                                         \
	*(hp) = GrStringHeight(t,l,o);                                         \
@} while(0)
#endif /* GRX_SKIP_INLINES */

/* ================================================================== */
/*            THICK AND DASHED LINE DRAWING PRIMITIVES                */
/* ================================================================== */

/*
 * custom line option structure
 *   zero or one dash pattern length means the line is continuous
 *   the dash pattern always begins with a drawn section
 */
typedef struct @{
	GrColor lno_color;                  /* color used to draw line */
	int     lno_width;                  /* width of the line */
	int     lno_pattlen;                /* length of the dash pattern */
	unsigned char *lno_dashpat;         /* draw/nodraw pattern */
@} GrLineOption;

void GrCustomLine(int x1,int y1,int x2,int y2,GrLineOption *o);
void GrCustomBox(int x1,int y1,int x2,int y2,GrLineOption *o);
void GrCustomCircle(int xc,int yc,int r,GrLineOption *o);
void GrCustomEllipse(int xc,int yc,int xa,int ya,GrLineOption *o);
void GrCustomCircleArc(int xc,int yc,int r,int start,int end,int style,GrLineOption *o);
void GrCustomEllipseArc(int xc,int yc,int xa,int ya,int start,int end,int style,GrLineOption *o);
void GrCustomPolyLine(int numpts,int points[][2],GrLineOption *o);
void GrCustomPolygon(int numpts,int points[][2],GrLineOption *o);

/* ================================================================== */
/*             PATTERNED DRAWING AND FILLING PRIMITIVES               */
/* ================================================================== */

/*
 * BITMAP: a mode independent way to specify a fill pattern of two
 *   colors. It is always 8 pixels wide (1 byte per scan line), its
 *   height is user-defined. SET THE TYPE FLAG TO ZERO!!!
 */
typedef struct _GR_bitmap @{
	int     bmp_ispixmap;               /* type flag for pattern union */
	int     bmp_height;                 /* bitmap height */
	char   *bmp_data;                   /* pointer to the bit pattern */
	GrColor bmp_fgcolor;                /* foreground color for fill */
	GrColor bmp_bgcolor;                /* background color for fill */
	int     bmp_memflags;               /* set if dynamically allocated */
@} GrBitmap;

/*
 * PIXMAP: a fill pattern stored in a layout identical to the video RAM
 *   for filling using 'bitblt'-s. It is mode dependent, typically one
 *   of the library functions is used to build it. KEEP THE TYPE FLAG
 *   NONZERO!!!
 */
typedef struct _GR_pixmap @{
	int     pxp_ispixmap;               /* type flag for pattern union */
	int     pxp_width;                  /* pixmap width (in pixels)  */
	int     pxp_height;                 /* pixmap height (in pixels) */
	GrColor pxp_oper;                   /* bitblt mode (SET, OR, XOR, AND, IMAGE) */
	struct _GR_frame pxp_source;        /* source context for fill */
@} GrPixmap;

/*
 * Fill pattern union -- can either be a bitmap or a pixmap
 */
typedef union _GR_pattern @{
	int      gp_ispixmap;               /* nonzero for pixmaps */
	GrBitmap gp_bitmap;                 /* fill bitmap */
	GrPixmap gp_pixmap;                 /* fill pixmap */
@} GrPattern;

#define gp_bmp_data                     gp_bitmap.bmp_data
#define gp_bmp_height                   gp_bitmap.bmp_height
#define gp_bmp_fgcolor                  gp_bitmap.bmp_fgcolor
#define gp_bmp_bgcolor                  gp_bitmap.bmp_bgcolor

#define gp_pxp_width                    gp_pixmap.pxp_width
#define gp_pxp_height                   gp_pixmap.pxp_height
#define gp_pxp_oper                     gp_pixmap.pxp_oper
#define gp_pxp_source                   gp_pixmap.pxp_source

/*
 * Draw pattern for line drawings -- specifies both the:
 *   (1) fill pattern, and the
 *   (2) custom line drawing option
 */
typedef struct @{
	GrPattern     *lnp_pattern;         /* fill pattern */
	GrLineOption  *lnp_option;          /* width + dash pattern */
@} GrLinePattern;

GrPattern *GrBuildPixmap(char *pixels,int w,int h,GrColorTableP colors);
GrPattern *GrBuildPixmapFromBits(char *bits,int w,int h,GrColor fgc,GrColor bgc);
GrPattern *GrConvertToPixmap(GrContext *src);

void GrDestroyPattern(GrPattern *p);

void GrPatternedLine(int x1,int y1,int x2,int y2,GrLinePattern *lp);
void GrPatternedBox(int x1,int y1,int x2,int y2,GrLinePattern *lp);
void GrPatternedCircle(int xc,int yc,int r,GrLinePattern *lp);
void GrPatternedEllipse(int xc,int yc,int xa,int ya,GrLinePattern *lp);
void GrPatternedCircleArc(int xc,int yc,int r,int start,int end,int style,GrLinePattern *lp);
void GrPatternedEllipseArc(int xc,int yc,int xa,int ya,int start,int end,int style,GrLinePattern *lp);
void GrPatternedPolyLine(int numpts,int points[][2],GrLinePattern *lp);
void GrPatternedPolygon(int numpts,int points[][2],GrLinePattern *lp);

void GrPatternFilledPlot(int x,int y,GrPattern *p);
void GrPatternFilledLine(int x1,int y1,int x2,int y2,GrPattern *p);
void GrPatternFilledBox(int x1,int y1,int x2,int y2,GrPattern *p);
void GrPatternFilledCircle(int xc,int yc,int r,GrPattern *p);
void GrPatternFilledEllipse(int xc,int yc,int xa,int ya,GrPattern *p);
void GrPatternFilledCircleArc(int xc,int yc,int r,int start,int end,int style,GrPattern *p);
void GrPatternFilledEllipseArc(int xc,int yc,int xa,int ya,int start,int end,int style,GrPattern *p);
void GrPatternFilledConvexPolygon(int numpts,int points[][2],GrPattern *p);
void GrPatternFilledPolygon(int numpts,int points[][2],GrPattern *p);
void GrPatternFloodFill(int x, int y, GrColor border, GrPattern *p);

void GrPatternDrawChar(int chr,int x,int y,GrTextOption *opt,GrPattern *p);
void GrPatternDrawString(void *text,int length,int x,int y,GrTextOption *opt,GrPattern *p);
void GrPatternDrawStringExt(void *text,int length,int x,int y,GrTextOption *opt,GrPattern *p);

/* ================================================================== */
/*                      IMAGE MANIPULATION                            */
/* ================================================================== */

/*
 *  by Michal Stencl Copyright (c) 1998 for GRX
 *  <e-mail>    - [stenclpmd@@ba.telecom.sk]
 */

#ifndef GrImage
#define GrImage GrPixmap
#endif

/* Flags for GrImageInverse() */

#define GR_IMAGE_INVERSE_LR  0x01  /* inverse left right */
#define GR_IMAGE_INVERSE_TD  0x02  /* inverse top down */

GrImage *GrImageBuild(char *pixels,int w,int h,GrColorTableP colors);
void     GrImageDestroy(GrImage *i);
void     GrImageDisplay(int x,int y, GrImage *i);
void     GrImageDisplayExt(int x1,int y1,int x2,int y2, GrImage *i);
void     GrImageFilledBoxAlign(int xo,int yo,int x1,int y1,int x2,int y2,GrImage *p);
void     GrImageHLineAlign(int xo,int yo,int x,int y,int width,GrImage *p);
void     GrImagePlotAlign(int xo,int yo,int x,int y,GrImage *p);

GrImage *GrImageInverse(GrImage *p,int flag);
GrImage *GrImageStretch(GrImage *p,int nwidth,int nheight);

GrImage *GrImageFromPattern(GrPattern *p);
GrImage *GrImageFromContext(GrContext *c);
GrImage *GrImageBuildUsedAsPattern(char *pixels,int w,int h,GrColorTableP colors);

GrPattern *GrPatternFromImage(GrImage *p);


#ifndef GRX_SKIP_INLINES
#define GrImageFromPattern(p) \
	(((p) && (p)->gp_ispixmap) ? (&(p)->gp_pixmap) : NULL)
#define GrImageFromContext(c) \
	(GrImage *)GrConvertToPixmap(c)
#define GrPatternFromImage(p) \
	(GrPattern *)(p)
#define GrImageBuildUsedAsPattern(pixels,w,h,colors) \
	(GrImage *)GrBuildPixmap(pixels,w,h,colors);
#define GrImageDestroy(i)   \
	  GrDestroyPattern((GrPattern *)(i));
#endif

/* ================================================================== */
/*                        CTX2PNM ROUTINES                            */
/* ================================================================== */

/*
 *  The PBM formats, grx support load/save of
 *  binaries formats (4,5,6) only
 */

#define PLAINPBMFORMAT 1
#define PLAINPGMFORMAT 2
#define PLAINPPMFORMAT 3
#define PBMFORMAT      4
#define PGMFORMAT      5
#define PPMFORMAT      6

/* The ctx2pnm routines */

int GrSaveContextToPbm( GrContext *grc, char *pbmfn, char *docn );
int GrSaveContextToPgm( GrContext *grc, char *pgmfn, char *docn );
int GrSaveContextToPpm( GrContext *grc, char *ppmfn, char *docn );
int GrLoadContextFromPnm( GrContext *grc, char *pnmfn );
int GrQueryPnm( char *pnmfn, int *width, int *height, int *maxval );

/* ================================================================== */
/*               DRAWING IN USER WINDOW COORDINATES                   */
/* ================================================================== */

void GrSetUserWindow(int x1,int y1,int x2,int y2);
void GrGetUserWindow(int *x1,int *y1,int *x2,int *y2);
void GrGetScreenCoord(int *x,int *y);
void GrGetUserCoord(int *x,int *y);

void GrUsrPlot(int x,int y,GrColor c);
void GrUsrLine(int x1,int y1,int x2,int y2,GrColor c);
void GrUsrHLine(int x1,int x2,int y,GrColor c);
void GrUsrVLine(int x,int y1,int y2,GrColor c);
void GrUsrBox(int x1,int y1,int x2,int y2,GrColor c);
void GrUsrFilledBox(int x1,int y1,int x2,int y2,GrColor c);
void GrUsrFramedBox(int x1,int y1,int x2,int y2,int wdt,GrFBoxColors *c);
void GrUsrCircle(int xc,int yc,int r,GrColor c);
void GrUsrEllipse(int xc,int yc,int xa,int ya,GrColor c);
void GrUsrCircleArc(int xc,int yc,int r,int start,int end,int style,GrColor c);
void GrUsrEllipseArc(int xc,int yc,int xa,int ya,int start,int end,int style,GrColor c);
void GrUsrFilledCircle(int xc,int yc,int r,GrColor c);
void GrUsrFilledEllipse(int xc,int yc,int xa,int ya,GrColor c);
void GrUsrFilledCircleArc(int xc,int yc,int r,int start,int end,int style,GrColor c);
void GrUsrFilledEllipseArc(int xc,int yc,int xa,int ya,int start,int end,int style,GrColor c);
void GrUsrPolyLine(int numpts,int points[][2],GrColor c);
void GrUsrPolygon(int numpts,int points[][2],GrColor c);
void GrUsrFilledConvexPolygon(int numpts,int points[][2],GrColor c);
void GrUsrFilledPolygon(int numpts,int points[][2],GrColor c);
void GrUsrFloodFill(int x, int y, GrColor border, GrColor c);

GrColor GrUsrPixel(int x,int y);
GrColor GrUsrPixelC(GrContext *c,int x,int y);

void GrUsrCustomLine(int x1,int y1,int x2,int y2,GrLineOption *o);
void GrUsrCustomBox(int x1,int y1,int x2,int y2,GrLineOption *o);
void GrUsrCustomCircle(int xc,int yc,int r,GrLineOption *o);
void GrUsrCustomEllipse(int xc,int yc,int xa,int ya,GrLineOption *o);
void GrUsrCustomCircleArc(int xc,int yc,int r,int start,int end,int style,GrLineOption *o);
void GrUsrCustomEllipseArc(int xc,int yc,int xa,int ya,int start,int end,int style,GrLineOption *o);
void GrUsrCustomPolyLine(int numpts,int points[][2],GrLineOption *o);
void GrUsrCustomPolygon(int numpts,int points[][2],GrLineOption *o);

void GrUsrPatternedLine(int x1,int y1,int x2,int y2,GrLinePattern *lp);
void GrUsrPatternedBox(int x1,int y1,int x2,int y2,GrLinePattern *lp);
void GrUsrPatternedCircle(int xc,int yc,int r,GrLinePattern *lp);
void GrUsrPatternedEllipse(int xc,int yc,int xa,int ya,GrLinePattern *lp);
void GrUsrPatternedCircleArc(int xc,int yc,int r,int start,int end,int style,GrLinePattern *lp);
void GrUsrPatternedEllipseArc(int xc,int yc,int xa,int ya,int start,int end,int style,GrLinePattern *lp);
void GrUsrPatternedPolyLine(int numpts,int points[][2],GrLinePattern *lp);
void GrUsrPatternedPolygon(int numpts,int points[][2],GrLinePattern *lp);

void GrUsrPatternFilledPlot(int x,int y,GrPattern *p);
void GrUsrPatternFilledLine(int x1,int y1,int x2,int y2,GrPattern *p);
void GrUsrPatternFilledBox(int x1,int y1,int x2,int y2,GrPattern *p);
void GrUsrPatternFilledCircle(int xc,int yc,int r,GrPattern *p);
void GrUsrPatternFilledEllipse(int xc,int yc,int xa,int ya,GrPattern *p);
void GrUsrPatternFilledCircleArc(int xc,int yc,int r,int start,int end,int style,GrPattern *p);
void GrUsrPatternFilledEllipseArc(int xc,int yc,int xa,int ya,int start,int end,int style,GrPattern *p);
void GrUsrPatternFilledConvexPolygon(int numpts,int points[][2],GrPattern *p);
void GrUsrPatternFilledPolygon(int numpts,int points[][2],GrPattern *p);
void GrUsrPatternFloodFill(int x, int y, GrColor border, GrPattern *p);

void GrUsrDrawChar(int chr,int x,int y,GrTextOption *opt);
void GrUsrDrawString(char *text,int length,int x,int y,GrTextOption *opt);
void GrUsrTextXY(int x,int y,char *text,GrColor fg,GrColor bg);

/* ================================================================== */
/*                    GRAPHICS CURSOR UTILITIES                       */
/* ================================================================== */

typedef struct _GR_cursor @{
	struct _GR_context work;                    /* work areas (4) */
	int     xcord,ycord;                        /* cursor position on screen */
	int     xsize,ysize;                        /* cursor size */
	int     xoffs,yoffs;                        /* LU corner to hot point offset */
	int     xwork,ywork;                        /* save/work area sizes */
	int     xwpos,ywpos;                        /* save/work area position on screen */
	int     displayed;                          /* set if displayed */
@} GrCursor;

GrCursor *GrBuildCursor(char far *pixels,int pitch,int w,int h,int xo,int yo,GrColorTableP c);
void GrDestroyCursor(GrCursor *cursor);
void GrDisplayCursor(GrCursor *cursor);
void GrEraseCursor(GrCursor *cursor);
void GrMoveCursor(GrCursor *cursor,int x,int y);

/* ================================================================== */
/*               MOUSE AND KEYBOARD INPUT UTILITIES                   */
/* ================================================================== */

#define GR_M_MOTION         0x001               /* mouse event flag bits */
#define GR_M_LEFT_DOWN      0x002
#define GR_M_LEFT_UP        0x004
#define GR_M_RIGHT_DOWN     0x008
#define GR_M_RIGHT_UP       0x010
#define GR_M_MIDDLE_DOWN    0x020
#define GR_M_MIDDLE_UP      0x040
#define GR_M_BUTTON_DOWN    (GR_M_LEFT_DOWN | GR_M_MIDDLE_DOWN | GR_M_RIGHT_DOWN)
#define GR_M_BUTTON_UP      (GR_M_LEFT_UP   | GR_M_MIDDLE_UP   | GR_M_RIGHT_UP)
#define GR_M_BUTTON_CHANGE  (GR_M_BUTTON_UP | GR_M_BUTTON_DOWN )

#define GR_M_LEFT           1                   /* mouse button index bits */
#define GR_M_RIGHT          2
#define GR_M_MIDDLE         4

#define GR_M_KEYPRESS       0x080               /* other event flag bits */
#define GR_M_POLL           0x100
#define GR_M_NOPAINT        0x200
#define GR_M_EVENT          (GR_M_MOTION | GR_M_KEYPRESS | GR_M_BUTTON_DOWN | GR_M_BUTTON_UP)

#define GR_KB_RIGHTSHIFT    0x01                /* Keybd states: right shift key depressed */
#define GR_KB_LEFTSHIFT     0x02                /* left shift key depressed */
#define GR_KB_CTRL          0x04                /* CTRL depressed */
#define GR_KB_ALT           0x08                /* ALT depressed */
#define GR_KB_SCROLLOCK     0x10                /* SCROLL LOCK active */
#define GR_KB_NUMLOCK       0x20                /* NUM LOCK active */
#define GR_KB_CAPSLOCK      0x40                /* CAPS LOCK active */
#define GR_KB_INSERT        0x80                /* INSERT state active */
#define GR_KB_SHIFT         (GR_KB_LEFTSHIFT | GR_KB_RIGHTSHIFT)

#define GR_M_CUR_NORMAL     0                   /* MOUSE CURSOR modes: just the cursor */
#define GR_M_CUR_RUBBER     1                   /* rectangular rubber band (XOR-d to the screen) */
#define GR_M_CUR_LINE       2                   /* line attached to the cursor */
#define GR_M_CUR_BOX        3                   /* rectangular box dragged by the cursor */

#define GR_M_QUEUE_SIZE     128                 /* default queue size */

typedef struct _GR_mouseEvent @{                 /* mouse event buffer structure */
	int  flags;                                 /* event type flags (see above) */
	int  x,y;                                   /* mouse coordinates */
	int  buttons;                               /* mouse button state */
	int  key;                                   /* key code from keyboard */
	int  kbstat;                                /* keybd status (ALT, CTRL, etc..) */
	long dtime;                                 /* time since last event (msec) */
@} GrMouseEvent;

/*
 * mouse status information
 */
extern const struct _GR_mouseInfo @{
	int   (*block)(GrContext*,int,int,int,int); /* mouse block function */
	void  (*unblock)(int flags);                /* mouse unblock function */
	void  (*uninit)(void);                      /* mouse cleanupt function */
	struct _GR_cursor     *cursor;              /* the mouse cursor */
	struct _GR_mouseEvent *queue;               /* queue of pending input events */
	int     msstatus;                           /* -1:missing, 0:unknown, 1:present, 2:initted */
	int     displayed;                          /* cursor is (generally) drawn */
	int     blockflag;                          /* cursor temp. erase/block flag */
	int     docheck;                            /* need to check before gr. op. to screen */
	int     cursmode;                           /* mouse cursor draw mode */
	int     x1,y1,x2,y2;                        /* auxiliary params for some cursor draw modes */
	GrColor curscolor;                          /* color for some cursor draw modes */
	int     owncursor;                          /* auto generated cursor */
	int     xpos,ypos;                          /* current mouse position */
	int     xmin,xmax;                          /* mouse movement X coordinate limits */
	int     ymin,ymax;                          /* mouse movement Y coordinate limits */
	int     spmult,spdiv;                       /* mouse cursor speed factors */
	int     thresh,accel;                       /* mouse acceleration parameters */
	int     moved;                              /* mouse cursor movement flag */
	int     qsize;                              /* max size of the queue */
	int     qlength;                            /* current # of items in the queue */
	int     qread;                              /* read pointer for the queue */
	int     qwrite;                             /* write pointer for the queue */
@} * const GrMouseInfo;

int  GrMouseDetect(void);
void GrMouseEventMode(int dummy);
void GrMouseInit(void);
void GrMouseInitN(int queue_size);
void GrMouseUnInit(void);
void GrMouseSetSpeed(int spmult,int spdiv);
void GrMouseSetAccel(int thresh,int accel);
void GrMouseSetLimits(int x1,int y1,int x2,int y2);
void GrMouseGetLimits(int *x1,int *y1,int *x2,int *y2);
void GrMouseWarp(int x,int y);
void GrMouseEventEnable(int enable_kb,int enable_ms);
void GrMouseGetEvent(int flags,GrMouseEvent *event);

void GrMouseGetEventT(int flags,GrMouseEvent *event,long timout_msecs);
/* Note:
**       event->dtime is only valid if any event occured (event->flags !=0)
**       otherwise it's set as -1.
**       Additionally event timing is now real world time. (X11 && Linux
**       used clock(), user process time, up to 2.28f)
*/

int  GrMousePendingEvent(void);

GrCursor *GrMouseGetCursor(void);
void GrMouseSetCursor(GrCursor *cursor);
void GrMouseSetColors(GrColor fg,GrColor bg);
void GrMouseSetCursorMode(int mode,...);
void GrMouseDisplayCursor(void);
void GrMouseEraseCursor(void);
void GrMouseUpdateCursor(void);
int  GrMouseCursorIsDisplayed(void);

int  GrMouseBlock(GrContext *c,int x1,int y1,int x2,int y2);
void GrMouseUnBlock(int return_value_from_GrMouseBlock);

#if 0
/* !! old style (before grx v2.26) keyboard interface    !!
   !! old functions still linkable but for compatibility !!
   !! across platforms and with future versions of GRX   !!
   !! one use functions from grkeys.h                    !! */
#ifndef __MSDOS__
int  kbhit(void);
int  getch(void);
#endif
#ifndef __GO32__
int  getkey(void);
int  getxkey(void);
#endif
int  getkbstat(void);
#endif
/* Why this ???
#ifdef __WATCOMC__
int  getxkey(void);
#endif
*/

#ifndef GRX_SKIP_INLINES
#define GrMouseEventMode(x)         /* nothing! */
#define GrMouseGetCursor()          (GrMouseInfo->cursor)
#define GrMouseCursorIsDisplayed()  (GrMouseInfo->displayed)
#define GrMouseInit()               GrMouseInitN(GR_M_QUEUE_SIZE);
#define GrMouseGetEvent(f,ev)       GrMouseGetEventT((f),(ev),(-1L));
#define GrMousePendingEvent() (                                                \
	GrMouseUpdateCursor(),                                                 \
   (GrMouseInfo->qlength > 0)                                                  \
)
#define GrMouseUnInit() do @{                                                   \
	if(GrMouseInfo->uninit) @{                                              \
	(*GrMouseInfo->uninit)();                                              \
	@}                                                                      \
@} while(0)
#define GrMouseGetLimits(x1p,y1p,x2p,y2p) do @{                                 \
	*(x1p) = GrMouseInfo->xmin; *(y1p) = GrMouseInfo->ymin;                \
	*(x2p) = GrMouseInfo->xmax; *(y2p) = GrMouseInfo->ymax;                \
@} while(0)
#define GrMouseBlock(c,x1,y1,x2,y2) (                                          \
	(((c) ? (const GrContext*)(c) : GrCurrentContext())->gc_onscreen &&    \
	 (GrMouseInfo->docheck)) ?                                             \
	(*GrMouseInfo->block)((c),(x1),(y1),(x2),(y2)) :                       \
	0                                                                      \
)
#define GrMouseUnBlock(f) do @{                                                 \
	if((f) && GrMouseInfo->displayed) @{                                    \
	(*GrMouseInfo->unblock)((f));                                          \
	@}                                                                      \
@} while(0)
#endif  /* GRX_SKIP_INLINES */

/* ================================================================== */
/*               MISCELLANEOUS UTILITIY FUNCTIONS                     */
/* ================================================================== */

void GrResizeGrayMap(unsigned char far *map,int pitch,int ow,int oh,int nw,int nh);
int  GrMatchString(const char *pattern,const char *strg);


/* ================================================================== */
/*                            ADDON FUNCTIONS                         */
/*  these functions may not be installed or available on all system   */
/* ================================================================== */

/*
** SaveContextToTiff - Dump a context in a TIFF file
**
** Arguments:
**   cxt:   Context to be saved (NULL -> use current context)
**   tiffn: Name of tiff file
**   compr: Compression method (see tiff.h), 0: automatic selection
**   docn:  string saved in the tiff file (DOCUMENTNAME tag)
**
**  Returns  0 on success
**          -1 on error
**
** requires tifflib by  Sam Leffler (sam@@engr.sgi.com)
**        available at  ftp://ftp.sgi.com/graphics/tiff
*/
int SaveContextToTiff(GrContext *cxt, char *tiffn, unsigned compr, char *docn);


/*
** SaveContextToJpeg - Dump a context in a JPEG file
**
** Arguments:
**   cxt:      Context to be saved (NULL -> use current context)
**   jpegn:    Name of the jpeg file
**   accuracy: Accuracy percentage (100 for no loss of quality)
**
**  Returns  0 on success
**          -1 on error
**
** requires jpeg-6a by  IJG (Independent JPEG Group)
**        available at  ftp.uu.net as graphics/jpeg/jpegsrc.v6a.tar.gz
*/
int SaveContextToJpeg(GrContext *volatile cxt, char *jpegn, int accuracy);


#ifdef __cplusplus
@}
#endif
#endif  /* whole file */

@end example

@node grxkeys.h,, grx20.h, Includes
@c -----------------------------------------------------------------------------
@unnumberedsec grxkeys.h

@example
/**
 ** grxkeys.h ---- platform independent key definitions
 **
 ** Copyright (c) 1997 Hartmut Schirmer
 **/

#ifndef __GRKEYS_H_INCLUDED__
#define __GRKEYS_H_INCLUDED__

/*
** NOTES - some keys may not be available under all systems
**       - key values will be differ on different systems
*/

#ifndef __GRX20_H_INCLUDED__
#include <grx20.h>
#endif

#ifdef __cplusplus
extern "C" @{
#endif

/* all keycodes should fit into 16 bit unsigned */
typedef unsigned short GrKeyType;

/* no key available */
#define GrKey_NoKey                0x0000

/* key typed but code outside 1..GrKey_LastDefinedKeycode */
#define GrKey_OutsideValidRange    0x0100

/* standard ASCII key codes */
#define GrKey_Control_A            0x0001
#define GrKey_Control_B            0x0002
#define GrKey_Control_C            0x0003
#define GrKey_Control_D            0x0004
#define GrKey_Control_E            0x0005
#define GrKey_Control_F            0x0006
#define GrKey_Control_G            0x0007
#define GrKey_Control_H            0x0008
#define GrKey_Control_I            0x0009
#define GrKey_Control_J            0x000a
#define GrKey_Control_K            0x000b
#define GrKey_Control_L            0x000c
#define GrKey_Control_M            0x000d
#define GrKey_Control_N            0x000e
#define GrKey_Control_O            0x000f
#define GrKey_Control_P            0x0010
#define GrKey_Control_Q            0x0011
#define GrKey_Control_R            0x0012
#define GrKey_Control_S            0x0013
#define GrKey_Control_T            0x0014
#define GrKey_Control_U            0x0015
#define GrKey_Control_V            0x0016
#define GrKey_Control_W            0x0017
#define GrKey_Control_X            0x0018
#define GrKey_Control_Y            0x0019
#define GrKey_Control_Z            0x001a
#define GrKey_Control_LBracket     0x001b
#define GrKey_Control_BackSlash    0x001c
#define GrKey_Control_RBracket     0x001d
#define GrKey_Control_Caret        0x001e
#define GrKey_Control_Underscore   0x001f
#define GrKey_Space                0x0020
#define GrKey_ExclamationPoint     0x0021
#define GrKey_DoubleQuote          0x0022
#define GrKey_Hash                 0x0023
#define GrKey_Dollar               0x0024
#define GrKey_Percent              0x0025
#define GrKey_Ampersand            0x0026
#define GrKey_Quote                0x0027
#define GrKey_LParen               0x0028
#define GrKey_RParen               0x0029
#define GrKey_Star                 0x002a
#define GrKey_Plus                 0x002b
#define GrKey_Comma                0x002c
#define GrKey_Dash                 0x002d
#define GrKey_Period               0x002e
#define GrKey_Slash                0x002f
#define GrKey_0                    0x0030
#define GrKey_1                    0x0031
#define GrKey_2                    0x0032
#define GrKey_3                    0x0033
#define GrKey_4                    0x0034
#define GrKey_5                    0x0035
#define GrKey_6                    0x0036
#define GrKey_7                    0x0037
#define GrKey_8                    0x0038
#define GrKey_9                    0x0039
#define GrKey_Colon                0x003a
#define GrKey_SemiColon            0x003b
#define GrKey_LAngle               0x003c
#define GrKey_Equals               0x003d
#define GrKey_RAngle               0x003e
#define GrKey_QuestionMark         0x003f
#define GrKey_At                   0x0040
#define GrKey_A                    0x0041
#define GrKey_B                    0x0042
#define GrKey_C                    0x0043
#define GrKey_D                    0x0044
#define GrKey_E                    0x0045
#define GrKey_F                    0x0046
#define GrKey_G                    0x0047
#define GrKey_H                    0x0048
#define GrKey_I                    0x0049
#define GrKey_J                    0x004a
#define GrKey_K                    0x004b
#define GrKey_L                    0x004c
#define GrKey_M                    0x004d
#define GrKey_N                    0x004e
#define GrKey_O                    0x004f
#define GrKey_P                    0x0050
#define GrKey_Q                    0x0051
#define GrKey_R                    0x0052
#define GrKey_S                    0x0053
#define GrKey_T                    0x0054
#define GrKey_U                    0x0055
#define GrKey_V                    0x0056
#define GrKey_W                    0x0057
#define GrKey_X                    0x0058
#define GrKey_Y                    0x0059
#define GrKey_Z                    0x005a
#define GrKey_LBracket             0x005b
#define GrKey_BackSlash            0x005c
#define GrKey_RBracket             0x005d
#define GrKey_Caret                0x005e
#define GrKey_UnderScore           0x005f
#define GrKey_BackQuote            0x0060
#define GrKey_a                    0x0061
#define GrKey_b                    0x0062
#define GrKey_c                    0x0063
#define GrKey_d                    0x0064
#define GrKey_e                    0x0065
#define GrKey_f                    0x0066
#define GrKey_g                    0x0067
#define GrKey_h                    0x0068
#define GrKey_i                    0x0069
#define GrKey_j                    0x006a
#define GrKey_k                    0x006b
#define GrKey_l                    0x006c
#define GrKey_m                    0x006d
#define GrKey_n                    0x006e
#define GrKey_o                    0x006f
#define GrKey_p                    0x0070
#define GrKey_q                    0x0071
#define GrKey_r                    0x0072
#define GrKey_s                    0x0073
#define GrKey_t                    0x0074
#define GrKey_u                    0x0075
#define GrKey_v                    0x0076
#define GrKey_w                    0x0077
#define GrKey_x                    0x0078
#define GrKey_y                    0x0079
#define GrKey_z                    0x007a
#define GrKey_LBrace               0x007b
#define GrKey_Pipe                 0x007c
#define GrKey_RBrace               0x007d
#define GrKey_Tilde                0x007e
#define GrKey_Control_Backspace    0x007f

/* extended key codes as defined in DJGPP */
#define GrKey_Alt_Escape           0x0101
#define GrKey_Control_At           0x0103
#define GrKey_Alt_Backspace        0x010e
#define GrKey_BackTab              0x010f
#define GrKey_Alt_Q                0x0110
#define GrKey_Alt_W                0x0111
#define GrKey_Alt_E                0x0112
#define GrKey_Alt_R                0x0113
#define GrKey_Alt_T                0x0114
#define GrKey_Alt_Y                0x0115
#define GrKey_Alt_U                0x0116
#define GrKey_Alt_I                0x0117
#define GrKey_Alt_O                0x0118
#define GrKey_Alt_P                0x0119
#define GrKey_Alt_LBracket         0x011a
#define GrKey_Alt_RBracket         0x011b
#define GrKey_Alt_Return           0x011c
#define GrKey_Alt_A                0x011e
#define GrKey_Alt_S                0x011f
#define GrKey_Alt_D                0x0120
#define GrKey_Alt_F                0x0121
#define GrKey_Alt_G                0x0122
#define GrKey_Alt_H                0x0123
#define GrKey_Alt_J                0x0124
#define GrKey_Alt_K                0x0125
#define GrKey_Alt_L                0x0126
#define GrKey_Alt_Semicolon        0x0127
#define GrKey_Alt_Quote            0x0128
#define GrKey_Alt_Backquote        0x0129
#define GrKey_Alt_Backslash        0x012b
#define GrKey_Alt_Z                0x012c
#define GrKey_Alt_X                0x012d
#define GrKey_Alt_C                0x012e
#define GrKey_Alt_V                0x012f
#define GrKey_Alt_B                0x0130
#define GrKey_Alt_N                0x0131
#define GrKey_Alt_M                0x0132
#define GrKey_Alt_Comma            0x0133
#define GrKey_Alt_Period           0x0134
#define GrKey_Alt_Slash            0x0135
#define GrKey_Alt_KPStar           0x0137
#define GrKey_F1                   0x013b
#define GrKey_F2                   0x013c
#define GrKey_F3                   0x013d
#define GrKey_F4                   0x013e
#define GrKey_F5                   0x013f
#define GrKey_F6                   0x0140
#define GrKey_F7                   0x0141
#define GrKey_F8                   0x0142
#define GrKey_F9                   0x0143
#define GrKey_F10                  0x0144
#define GrKey_Home                 0x0147
#define GrKey_Up                   0x0148
#define GrKey_PageUp               0x0149
#define GrKey_Alt_KPMinus          0x014a
#define GrKey_Left                 0x014b
#define GrKey_Center               0x014c
#define GrKey_Right                0x014d
#define GrKey_Alt_KPPlus           0x014e
#define GrKey_End                  0x014f
#define GrKey_Down                 0x0150
#define GrKey_PageDown             0x0151
#define GrKey_Insert               0x0152
#define GrKey_Delete               0x0153
#define GrKey_Shift_F1             0x0154
#define GrKey_Shift_F2             0x0155
#define GrKey_Shift_F3             0x0156
#define GrKey_Shift_F4             0x0157
#define GrKey_Shift_F5             0x0158
#define GrKey_Shift_F6             0x0159
#define GrKey_Shift_F7             0x015a
#define GrKey_Shift_F8             0x015b
#define GrKey_Shift_F9             0x015c
#define GrKey_Shift_F10            0x015d
#define GrKey_Control_F1           0x015e
#define GrKey_Control_F2           0x015f
#define GrKey_Control_F3           0x0160
#define GrKey_Control_F4           0x0161
#define GrKey_Control_F5           0x0162
#define GrKey_Control_F6           0x0163
#define GrKey_Control_F7           0x0164
#define GrKey_Control_F8           0x0165
#define GrKey_Control_F9           0x0166
#define GrKey_Control_F10          0x0167
#define GrKey_Alt_F1               0x0168
#define GrKey_Alt_F2               0x0169
#define GrKey_Alt_F3               0x016a
#define GrKey_Alt_F4               0x016b
#define GrKey_Alt_F5               0x016c
#define GrKey_Alt_F6               0x016d
#define GrKey_Alt_F7               0x016e
#define GrKey_Alt_F8               0x016f
#define GrKey_Alt_F9               0x0170
#define GrKey_Alt_F10              0x0171
#define GrKey_Control_Print        0x0172
#define GrKey_Control_Left         0x0173
#define GrKey_Control_Right        0x0174
#define GrKey_Control_End          0x0175
#define GrKey_Control_PageDown     0x0176
#define GrKey_Control_Home         0x0177
#define GrKey_Alt_1                0x0178
#define GrKey_Alt_2                0x0179
#define GrKey_Alt_3                0x017a
#define GrKey_Alt_4                0x017b
#define GrKey_Alt_5                0x017c
#define GrKey_Alt_6                0x017d
#define GrKey_Alt_7                0x017e
#define GrKey_Alt_8                0x017f
#define GrKey_Alt_9                0x0180
#define GrKey_Alt_0                0x0181
#define GrKey_Alt_Dash             0x0182
#define GrKey_Alt_Equals           0x0183
#define GrKey_Control_PageUp       0x0184
#define GrKey_F11                  0x0185
#define GrKey_F12                  0x0186
#define GrKey_Shift_F11            0x0187
#define GrKey_Shift_F12            0x0188
#define GrKey_Control_F11          0x0189
#define GrKey_Control_F12          0x018a
#define GrKey_Alt_F11              0x018b
#define GrKey_Alt_F12              0x018c
#define GrKey_Control_Up           0x018d
#define GrKey_Control_KPDash       0x018e
#define GrKey_Control_Center       0x018f
#define GrKey_Control_KPPlus       0x0190
#define GrKey_Control_Down         0x0191
#define GrKey_Control_Insert       0x0192
#define GrKey_Control_Delete       0x0193
#define GrKey_Control_Tab          0x0194
#define GrKey_Control_KPSlash      0x0195
#define GrKey_Control_KPStar       0x0196
#define GrKey_Alt_KPSlash          0x01a4
#define GrKey_Alt_Tab              0x01a5
#define GrKey_Alt_Enter            0x01a6

/* some additional codes not in DJGPP */
#define GrKey_Alt_LAngle           0x01b0
#define GrKey_Alt_RAngle           0x01b1
#define GrKey_Alt_At               0x01b2
#define GrKey_Alt_LBrace           0x01b3
#define GrKey_Alt_Pipe             0x01b4
#define GrKey_Alt_RBrace           0x01b5
#define GrKey_Print                0x01b6
#define GrKey_Shift_Insert         0x01b7
#define GrKey_Shift_Home           0x01b8
#define GrKey_Shift_End            0x01b9
#define GrKey_Shift_PageUp         0x01ba
#define GrKey_Shift_PageDown       0x01bb
#define GrKey_Alt_Up               0x01bc
#define GrKey_Alt_Left             0x01bd
#define GrKey_Alt_Center           0x01be
#define GrKey_Alt_Right            0x01c0
#define GrKey_Alt_Down             0x01c1
#define GrKey_Alt_Insert           0x01c2
#define GrKey_Alt_Delete           0x01c3
#define GrKey_Alt_Home             0x01c4
#define GrKey_Alt_End              0x01c5
#define GrKey_Alt_PageUp           0x01c6
#define GrKey_Alt_PageDown         0x01c7
#define GrKey_Shift_Up             0x01c8
#define GrKey_Shift_Down           0x01c9
#define GrKey_Shift_Right          0x01ca
#define GrKey_Shift_Left           0x01cb

/* this may be usefull for table allocation ... */
#define GrKey_LastDefinedKeycode   GrKey_Shift_Left

/* some well known synomyms */
#define GrKey_BackSpace            GrKey_Control_H
#define GrKey_Tab                  GrKey_Control_I
#define GrKey_LineFeed             GrKey_Control_J
#define GrKey_Escape               GrKey_Control_LBracket
#define GrKey_Return               GrKey_Control_M

/*
** new functions to replace the old style
**   kbhit / getch / getkey / getxkey / getkbstat
** keyboard interface
*/
extern int GrKeyPressed(void);
extern GrKeyType GrKeyRead(void);
extern int GrKeyStat(void);

/* some compatibility interface here ?? */
/* eg., #define kbhit() GrKeyPressed() ?? */

#ifdef __cplusplus
@}
#endif

#endif /* whole file */

@end example

@c -----------------------------------------------------------------------------
